\chapter{Giới thiệu đề tài}

\section{Thực trạng chung}

Trong những năm gần đây, lĩnh vực trí tuệ nhân tạo ngày càng phát triển nhanh chóng. Việc ứng dụng thành quả của trí tuệ nhân tạo ngày càng được phổ biến rộng rãi, không chỉ trong đời sống hằng ngày mà cả trong công việc. Khi các ứng dụng học máy không còn bị giới hạn trong phòng thí nghiệm nữa để được ứng dụng vào trong đời sống trong các lĩnh vực sản xuất như bảo trì thông minh và kiểm soát chất lượng. Khi đó, các câu hỏi liên quan đến độ tin cậy và độ bền liên tục của chúng nảy sinh.

Các tập dữ liệu tĩnh được sử dụng để huấn luyện các mô hình học máy chỉ có thể nắm bắt được một phần nhỏ các điều kiện có thể xảy ra trong thế giới thực. Các trường hợp trôi dạt, chẳng hạn như thay đổi điều kiện môi trường, thiết bị và vận hành có thể, theo thời gian, làm giảm đáng kể hiệu suất của các mô hình học máy, gây ảnh hưởng đến sự an toàn, độ tin cậy của mô hình và kinh tế nếu không được giải quyết đúng cách. Do đó, cần phải (1) phát hiện sự trôi dạt sớm nhất có thể và (2) điều chỉnh hiệu quả mô hình theo các điều kiện thay đổi động.

\section{Tổng quan về bài toán}

Trong lĩnh vực xử lý dữ liệu, việc xử lý và phân tích dữ liệu streaming đang trở thành một thách thức lớn do tính phức tạp và sự đặc biệt của dữ liệu này. Một thách thức lớn trong quá trình xử lý dữ liệu streaming là việc xử lý hiện tượng trôi dạt - một hiện tượng xảy ra làm ảnh hưởng đến phân phối dữ liệu đầu vào hoặc ý nghĩa của mối quan hệ của các tính chất dữ liệu đầu vào bị thay đổi - điều này ảnh hưởng đến hiệu suất của các mô hình học máy và các hệ thống phân tích dữ liệu.

Đối với nhiều ứng dụng và hệ thống, việc phát hiện trôi dạt trong thời gian thực hoặc gần thời gian thực là một yêu cầu quan trọng để đảm bảo chất lượng và độ tin cậy của kết quả phân tích, từ đó đưa ra các quyết định chính xác. Tuy nhiên, việc thực hiện điều này trong môi trường streaming đòi hỏi phải xử lý dữ liệu nhanh chóng và hiệu quả, với khả năng phát hiện và cảnh báo kịp thời, từ đó thực hiện các biện pháp điều chỉnh cần thiết cho mô hình học máy để thích nghi với dữ liệu thay đổi mới.

Trong đề tài này, luận văn tập trung vào việc nghiên cứu và phát triển phương pháp phát hiện trôi dạt sử dụng tiếp cận không giám sát để giải quyết bài toán phát hiện trôi dạt trong dữ liệu streaming và cập nhật mô hình thích ứng. Điều này đòi hỏi sự kết hợp giữa sức mạnh của phương pháp trong việc phát hiện hiện tượng và chiến thuật cập nhật mô hình thích ứng một cách hợp lý và nhanh chóng.

\section{Mục tiêu đề tài}

\subsection{Nghiên cứu và phân tích các phương pháp phát hiện trôi dạt}

Mục tiêu đầu tiên của đề tài là tiến hành một nghiên cứu chi tiết và phân tích sâu về phương pháp các phương pháp phát hiện trôi dạt hiện nay - bao gồm các phương pháp truyền thống và các tiếp cận hiện đại sau này. Luận văn sẽ xem xét cơ sở lý thuyết của các phương pháp, bao gồm cách tiếp cận các phương pháp triển khai, cũng như tính linh hoạt của chúng. Bằng cách này, luận văn sẽ làm rõ hơn về sức mạnh và giới hạn của các phương pháp để làm cơ sở cho việc lựa chọn và ứng dụng phương pháp phù hợp nhất đối với bài toán đặt ra.

\subsection{Ứng dụng phương pháp được chọn vào phát hiện trôi dạt trong dữ liệu streaming}

Sau khi hiểu rõ về cơ sở lý thuyết của phương pháp đó, luận văn sẽ tiến hành ứng dụng với phương pháp được chọn cho việc phát hiện trôi dạt trong dữ liệu streaming. Phương pháp này sẽ được thiết kế để đảm bảo khả năng phát hiện nhanh chóng và chính xác, cũng như mức độ trôi đối với nhiều kiểu dữ liệu phức tạp khác nhau.


\subsection{Tìm hiểu các chiến lược cập nhật mô hình thích ứng}

Sau khi phát hiện được hiện tượng trôi dạt xảy ra, tùy thuộc vào loại trôi dạt nào cũng như là mức độ ảnh hưởng của nó đến mô hình, luận văn sẽ nghiên cứu và đề xuất các chiến lược cập nhật mô hình thích ứng phù hợp. Điều này có thể bao gồm việc điều chỉnh các tham số của mô hình, thay đổi cấu trúc của mô hình, hoặc áp dụng các kỹ thuật học máy mới để cải thiện khả năng phát hiện và thích ứng với dữ liệu mới.


\subsection{Kết hợp phương pháp phát hiện trôi dạt với chiến lược cập nhật mô hình}

Luận văn sẽ tiến hành nghiêng cứu và phát triển một khung tổng thể kết hợp giữa phương pháp phát hiện trôi dạt và chiến lược cập nhật mô hình thích ứng. Khung này sẽ được thiết kế để đảm bảo tính linh hoạt và khả năng mở rộng, cho phép nó áp dụng hiệu quả trong các môi trường dữ liệu streaming có cấu trúc đa dạng và phức tạp.

\subsection{Tùy chỉnh và tối ưu hóa mô hình}

Dựa trên kết quả nghiên cứu và thử nghiệm, luận văn sẽ tiến hành tùy chỉnh và tối ưu hóa mô hình kết hợp giữa để phù hợp với các đặc điểm riêng biệt của dữ liệu streaming và các loại trôi dạt khác nhau. Quá trình này sẽ bao gồm việc điều chỉnh các tham số, cải thiện thuật toán, và phát triển các chiến lược thích ứng để đạt được hiệu suất tối ưu trong việc phát hiện.

\subsection{Triển khai và đánh giá}

Cuối cùng, luận văn sẽ triển khai mô hình đã được tối ưu hóa và tiến hành đánh giá toàn diện trên các tập dữ liệu synthetic và thực tế. Quá trình đánh giá sẽ bao gồm việc so sánh hiệu suất với các phương pháp hiện có, phân tích độ chính xác, thời gian phát hiện, và khả năng thích ứng của mô hình trong các điều kiện khác nhau.

\section{Giới hạn đề tài và đối tượng nghiên cứu}

\subsection{Phạm vi đề tài}

Nghiên cứu này tập trung vào các phạm vi cụ thể sau:

\textbf{Phạm vi về phương pháp:}
\begin{itemize}
    \item Nghiên cứu sâu về thuật toán Shape Drift Detector (ShapeDD) dựa trên Maximum Mean Discrepancy (MMD)
    \item Tập trung vào phát hiện concept drift trong môi trường dữ liệu streaming
    \item Không bao gồm các phương pháp adaptation sau khi phát hiện drift
\end{itemize}

\textbf{Phạm vi về dữ liệu:}
\begin{itemize}
    \item Chủ yếu sử dụng tập dữ liệu synthetic có kiểm soát để đánh giá hiệu suất
    \item Tập trung vào dữ liệu số với các loại concept drift: đột ngột, tăng dần, và tuần hoàn
    \item Không bao gồm dữ liệu phi cấu trúc như text, image hoặc audio
\end{itemize}

\textbf{Phạm vi về đánh giá:}
\begin{itemize}
    \item Đánh giá hiệu suất phát hiện drift về mặt độ chính xác, thời gian phát hiện và tỷ lệ false alarm
    \item So sánh với các baseline methods cơ bản
    \item Không bao gồm đánh giá về computational complexity chi tiết hoặc scalability trên big data
\end{itemize}

\subsection{Đối tượng nghiên cứu}

\textbf{Đối tượng chính:} Thuật toán Shape Drift Detector (ShapeDD) và ứng dụng của nó trong phát hiện concept drift

\textbf{Các đối tượng cụ thể bao gồm:}
\begin{itemize}
    \item Cơ sở lý thuyết Maximum Mean Discrepancy (MMD) trong RKHS
    \item Các thành phần của thuật toán ShapeDD: kernel selection, window management, statistical validation
    \item Phương pháp tối ưu hóa tham số cho ShapeDD
    \item Tập dữ liệu synthetic mô phỏng các loại concept drift khác nhau
    \item Metrics đánh giá hiệu suất phát hiện drift
\end{itemize}
