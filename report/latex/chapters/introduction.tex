\chapter{Giới thiệu đề tài}

\section{Thực trạng chung}

Trong những năm gần đây, lĩnh vực trí tuệ nhân tạo ngày càng phát triển nhanh chóng. Việc ứng dụng thành quả của trí tuệ nhân tạo ngày càng được phổ biến rộng rãi, không chỉ trong đời sống hằng ngày mà cả trong công việc. Các ứng dụng học máy không còn bị giới hạn trong phòng thí nghiệm mà đã được triển khai vào đời sống thực tế trong các lĩnh vực sản xuất như bảo trì thông minh và kiểm soát chất lượng. Khi đó, các câu hỏi liên quan đến độ tin cậy và độ bền liên tục của chúng nảy sinh.

Các tập dữ liệu tĩnh được sử dụng để huấn luyện các mô hình học máy chỉ có thể nắm bắt được một phần nhỏ các điều kiện có thể xảy ra trong thế giới thực. Các trường hợp trôi dạt, chẳng hạn như thay đổi điều kiện môi trường, thiết bị và vận hành có thể, theo thời gian, làm giảm đáng kể hiệu suất của các mô hình học máy, gây ảnh hưởng đến sự an toàn, độ tin cậy của mô hình và kinh tế nếu không được giải quyết đúng cách. Do đó, cần phải (1) phát hiện sự trôi dạt sớm nhất có thể và (2) điều chỉnh hiệu quả mô hình theo các điều kiện thay đổi động.

\section{Bối cảnh và động lực nghiên cứu}

Nghiên cứu này được xây dựng dựa trên nền tảng của survey paper \textit{"One or Two Things We Know about Concept Drift -- A Survey on Monitoring Evolving Environments"}~\cite{hinder2024survey_partA,hinder2024survey_partB} được công bố trên Frontiers in Artificial Intelligence năm 2024. Survey này cung cấp một tổng quan toàn diện về các phương pháp phát hiện, định vị và giải thích concept drift, tạo nền tảng lý thuyết vững chắc cho nghiên cứu của nghiên cứu này.

Dựa trên các phát hiện từ survey, luận văn này tập trung vào việc triển khai thực tế một hệ thống hoàn chỉnh kết hợp ba thành phần chính: (1) phát hiện drift sử dụng ShapeDD detector, (2) phân loại loại drift sử dụng phương pháp CDT\_MSW, và (3) thích ứng mô hình với chiến lược phù hợp cho từng loại drift.

\section{Tổng quan về bài toán}

Trong lĩnh vực xử lý dữ liệu streaming, việc xử lý và phân tích dữ liệu đang trở thành một thách thức lớn do tính phức tạp và tính động của dữ liệu này. Một thách thức quan trọng là xử lý hiện tượng \textbf{concept drift} - một hiện tượng xảy ra khi phân phối dữ liệu đầu vào $P(X)$ hoặc mối quan hệ giữa đầu vào và đầu ra $P(y|X)$ thay đổi theo thời gian - điều này ảnh hưởng trực tiếp đến hiệu suất của các mô hình học máy và các hệ thống ra quyết định tự động.

Đối với các ứng dụng production-grade, việc phát hiện drift trong thời gian thực (real-time) hoặc gần thời gian thực là yêu cầu bắt buộc để đảm bảo chất lượng và độ tin cậy của hệ thống. Tuy nhiên, việc xây dựng hệ thống như vậy đặt ra nhiều thách thức: (1) phát hiện drift chính xác với false positive thấp, (2) xác định loại drift để chọn chiến lược thích ứng phù hợp, (3) cập nhật mô hình nhanh chóng mà không làm gián đoạn dịch vụ, và (4) xử lý khối lượng lớn dữ liệu streaming với latency thấp.

Luận văn này giải quyết các thách thức trên bằng cách xây dựng một hệ thống end-to-end kết hợp: (1) phát hiện drift không giám sát sử dụng ShapeDD với kernel-based statistical testing, (2) phân loại loại drift tự động sử dụng CDT\_MSW, (3) adaptation strategies đa dạng cho từng loại drift, và (4) kiến trúc streaming real-time sử dụng Apache Kafka cho việc xử lý event-driven có khả năng mở rộng.

\section{Mục tiêu đề tài}

\subsection{Nghiên cứu và phân tích các phương pháp phát hiện trôi dạt}

Mục tiêu đầu tiên của đề tài là tiến hành một nghiên cứu chi tiết và phân tích sâu về các phương pháp phát hiện trôi dạt hiện nay - bao gồm các phương pháp truyền thống và các tiếp cận hiện đại. Luận văn sẽ xem xét cơ sở lý thuyết của các phương pháp, cách tiếp cận triển khai, cũng như tính linh hoạt của chúng. Bằng cách này, luận văn sẽ làm rõ hơn về ưu điểm và hạn chế của các phương pháp để làm cơ sở cho việc lựa chọn và ứng dụng phương pháp phù hợp nhất đối với bài toán đặt ra.

\subsection{Ứng dụng phương pháp được chọn vào phát hiện trôi dạt trong hệ thống}

Sau khi hiểu rõ về cơ sở lý thuyết của phương pháp được chọn, luận văn sẽ tiến hành ứng dụng phương pháp đó cho việc phát hiện trôi dạt trong dữ liệu streaming. Phương pháp này sẽ được thiết kế để đảm bảo khả năng phát hiện nhanh chóng và chính xác đối với nhiều kiểu dữ liệu phức tạp khác nhau.

\subsection{Tìm hiểu các chiến lược cập nhật mô hình thích ứng}

Sau khi phát hiện được hiện tượng trôi dạt xảy ra, tùy thuộc vào loại trôi dạt cũng như mức độ ảnh hưởng của nó đến mô hình, luận văn sẽ nghiên cứu và đề xuất các chiến lược cập nhật mô hình thích ứng phù hợp. Điều này có thể bao gồm việc điều chỉnh các tham số của mô hình, thay đổi cấu trúc của mô hình, hoặc áp dụng các kỹ thuật học máy mới để cải thiện khả năng phát hiện và thích ứng với dữ liệu mới.

\subsection{Kết hợp phương pháp phát hiện trôi dạt với chiến lược cập nhật mô hình}

Luận văn sẽ tiến hành nghiên cứu và phát triển một khung tổng thể kết hợp giữa phương pháp phát hiện trôi dạt và chiến lược cập nhật mô hình thích ứng. Khung này sẽ được thiết kế để đảm bảo tính linh hoạt và khả năng mở rộng, cho phép áp dụng hiệu quả trong các môi trường dữ liệu streaming có cấu trúc đa dạng và phức tạp.

\subsection{Tùy chỉnh và tối ưu hóa mô hình}

Dựa trên kết quả nghiên cứu và thử nghiệm, luận văn sẽ tiến hành tùy chỉnh và tối ưu hóa mô hình để phù hợp với các đặc điểm riêng biệt của dữ liệu và các loại trôi dạt khác nhau. Quá trình này sẽ bao gồm việc điều chỉnh các tham số, cải thiện thuật toán, và phát triển các chiến lược thích ứng để đạt được hiệu suất tối ưu trong việc phát hiện.

\subsection{Triển khai và đánh giá}

Cuối cùng, luận văn sẽ triển khai mô hình đã được tối ưu hóa và tiến hành đánh giá toàn diện trên các tập dữ liệu tổng hợp và thực tế. Quá trình đánh giá sẽ bao gồm việc so sánh hiệu suất với các phương pháp hiện có, phân tích độ chính xác, thời gian phát hiện, và khả năng thích ứng của mô hình trong các điều kiện khác nhau.

\section{Giới hạn đề tài và đối tượng nghiên cứu}

\subsection{Phạm vi đề tài}

Nghiên cứu này tập trung vào các phạm vi cụ thể sau:

\textbf{Phạm vi về phương pháp:}
\begin{itemize}
    \item Nghiên cứu sâu về phương pháp Shape Drift Detector (ShapeDD) và các phương pháp phát hiện drift khác (ADWIN, DDM, EDDM, etc.)
    \item Phát hiện concept drift trong môi trường dữ liệu streaming real-time sử dụng Apache Kafka
    \item Nghiên cứu phương pháp phân loại loại drift (CDT\_MSW) để nhận diện sudden, incremental, gradual, recurrent và blip drift
    \item Phát triển chiến lược thích ứng mô hình (model adaptation strategies) phù hợp với từng loại drift
    \item Xây dựng hệ thống end-to-end từ phát hiện đến thích ứng tự động
\end{itemize}

\textbf{Phạm vi về dữ liệu:}
\begin{itemize}
    \item Chủ yếu sử dụng tập dữ liệu synthetic có kiểm soát để đánh giá hiệu suất
    \item Tập trung vào dữ liệu số với các loại concept drift: đột ngột, tăng dần, và tuần hoàn
    \item Không bao gồm dữ liệu phi cấu trúc như text, image hoặc audio
\end{itemize}

\textbf{Phạm vi về đánh giá:}
\begin{itemize}
    \item Đánh giá hiệu suất phát hiện drift về mặt độ chính xác, thời gian phát hiện và tỷ lệ false alarm
    \item So sánh với các baseline methods cơ bản
    \item Không bao gồm đánh giá về computational complexity chi tiết hoặc scalability trên big data
\end{itemize}

\subsection{Đối tượng nghiên cứu}

\textbf{Đối tượng chính:} Hệ thống phát hiện và thích ứng concept drift tự động sử dụng ShapeDD detector và chiến lược thích ứng theo loại drift

\textbf{Các đối tượng cụ thể bao gồm:}
\begin{itemize}
    \item Cơ sở lý thuyết Maximum Mean Discrepancy (MMD) trong RKHS và kernel-based drift detection
    \item Các thành phần của thuật toán ShapeDD: kernel selection, window management, statistical validation
    \item Phương pháp CDT\_MSW (Concept Drift Type Identification based on Multi-Sliding Windows) để phân loại loại drift
    \item Chiến lược thích ứng mô hình: full reset (sudden), gradual updates (incremental), weighted updates (gradual), model caching (recurrent), minimal update (blip)
    \item Kiến trúc streaming real-time: Apache Kafka message broker, Producer-Consumer pattern, event-driven architecture
    \item Tập dữ liệu synthetic mô phỏng các loại concept drift khác nhau (SEA dataset variants)
    \item Metrics đánh giá: F1-score, detection delay, recovery rate, accuracy degradation
\end{itemize}
