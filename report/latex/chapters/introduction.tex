\chapter{GIỚI THIỆU CHUNG}

\section{Đặt vấn đề}

Trong những năm gần đây, lĩnh vực trí tuệ nhân tạo ngày càng phát triển nhanh chóng. Việc ứng dụng thành quả của trí tuệ nhân tạo ngày càng được phổ biến rộng rãi, không chỉ trong đời sống hằng ngày mà cả trong công việc. Khi các ứng dụng học máy không còn bị giới hạn trong phòng thí nghiệm nữa để được ứng dụng vào trong đời sống trong các lĩnh vực sản xuất như bảo trì thông minh và kiểm soát chất lượng. Khi đó, các câu hỏi liên quan đến độ tin cậy và độ bền liên tục của chúng nảy sinh. 
Các tập dữ liệu tĩnh được sử dụng để huấn luyện các mô hình học máy chỉ có thể nắm bắt được một phần nhỏ các điều kiện có thể xảy ra trong thế giới thực. Các trường hợp trôi dạt khái niệm (concept drift), chẳng hạn như thay đổi điều kiện môi trường, thiết bị và vận hành có thể, theo thời gian, làm giảm đáng kể hiệu suất của các mô hình học máy, gây ảnh hưởng đến sự an toàn, độ tin cậy của mô hình và kinh tế nếu không được giải quyết đúng cách. Nội dung của đề cương này được trình bày như sau:


\section{Mục tiêu nghiên cứu}

Trong bối cảnh dữ liệu lớn và học máy ngày càng phát triển, các hệ thống học máy thường đối mặt với thách thức khi dữ liệu thay đổi theo thời gian, dẫn đến hiện tượng concept drift. Hiện tượng này xảy ra khi phân phối dữ liệu hoặc mối quan hệ giữa dữ liệu đầu vào và đầu ra thay đổi, làm giảm hiệu suất của các mô hình dự đoán.

Việc phát hiện kịp thời và chính xác concept drift đóng vai trò quan trọng trong việc đảm bảo độ tin cậy và hiệu quả của các hệ thống học máy trong các ứng dụng thực tiễn như phân tích tài chính, y tế, và giám sát hệ thống.

Đề tài này tập trung vào việc xem xét một phương pháp phát hiện concept drift dựa trên đánh giá sự thay đổi phân phối, đánh giá hiệu quả của nó và đề xuất giải pháp tối ưu nhằm nâng cao khả năng thích ứng của các mô hình học máy trong môi trường dữ liệu động.

Về mặt mục tiêu, đề tài này sẽ đề ra 3 mục tiêu cần đạt được:

\begin{enumerate}
    \item Tìm hiểu cơ sở lý thuyết và cách hoạt động của phương pháp ShapeDD
    \item Thử nghiệm phương pháp với tập dữ liệu synthetic
    \item Đánh giá độ chính xác của phương pháp trong các trường hợp trôi dạt khác nhau
\end{enumerate}

\section{Đóng góp của nghiên cứu}

Nghiên cứu này mang lại những đóng góp quan trọng trong lĩnh vực phát hiện concept drift:

\textbf{Đóng góp về lý thuyết:}
\begin{itemize}
    \item Phân tích toàn diện về nền tảng lý thuyết của Shape Drift Detector (ShapeDD), bao gồm nghiên cứu chi tiết về Maximum Mean Discrepancy (MMD) trong không gian Reproducing Kernel Hilbert Space (RKHS)
    \item Hình thức hóa toán học của quy trình phát hiện đa giai đoạn ShapeDD, bao gồm thu thập dữ liệu, xây dựng đặc trưng, tính toán sự khác biệt và xác thực thống kê
    \item Phân tích lý thuyết về chiến lược lựa chọn kernel và quản lý cửa sổ để đạt hiệu suất phát hiện trôi dạt tối ưu
\end{itemize}

\textbf{Đóng góp về phương pháp:}
\begin{itemize}
    \item Triển khai và tối ưu hóa chi tiết thuật toán ShapeDD cho các tình huống trôi dạt khác nhau
    \item Phát triển các phương pháp tạo tập dữ liệu synthetic toàn diện để đánh giá có kiểm soát các mẫu trôi dạt đột ngột và tăng dần
    \item Phân tích độ nhạy tham số và các chiến lược tối ưu hóa cho kích thước cửa sổ, tham số kernel và ngưỡng ý nghĩa thống kê
\end{itemize}

\textbf{Đóng góp về thực nghiệm:}
\begin{itemize}
    \item Đánh giá thực nghiệm mở rộng trên tập dữ liệu synthetic chứng minh hiệu quả của ShapeDD trên các loại trôi dạt khác nhau
    \item Phân tích so sánh hiệu suất ShapeDD trong các điều kiện khác nhau, bao gồm các kích thước cửa sổ và độ lớn trôi dạt khác nhau
    \item Hướng dẫn thực tiễn để triển khai ShapeDD trong các ứng dụng thực tế, bao gồm lựa chọn tham số và các chiến lược tối ưu hiệu suất
\end{itemize}

\section{Cấu trúc luận văn}

Luận văn được tổ chức thành năm chương chính:

\textbf{Chương 1: Giới thiệu chung} trình bày tổng quan về nghiên cứu, bao gồm đặt vấn đề, mục tiêu nghiên cứu, đóng góp của nghiên cứu và cấu trúc luận văn.

\textbf{Chương 2: Tổng quan nghiên cứu liên quan} cung cấp khảo sát toàn diện về nghiên cứu hiện có trong lĩnh vực phát hiện và thích ứng concept drift. Chúng tôi xem xét sự phát triển của lĩnh vực, phân loại các phương pháp hiện có và xác định các hạn chế hiện tại cũng như khoảng trống nghiên cứu.

\textbf{Chương 3: Phương pháp nghiên cứu} trình bày phương pháp nghiên cứu của chúng tôi, bao gồm phát triển phân loại trôi dạt, thiết kế các thuật toán phát hiện mới và khung thực nghiệm được sử dụng để đánh giá.

\textbf{Chương 4: Kết quả và thảo luận} báo cáo các phát hiện từ đánh giá thực nghiệm mở rộng của chúng tôi. Chúng tôi phân tích hiệu suất của các phương pháp khác nhau trên các tình huống trôi dạt khác nhau và thảo luận về ý nghĩa của kết quả.

\textbf{Chương 5: Kết luận và hướng phát triển} tóm tắt các đóng góp chính của nghiên cứu này, thảo luận về các hạn chế và phác thảo các hướng đầy hứa hẹn cho nghiên cứu tương lai.

Luận văn kết thúc với các phụ lục chứa tài liệu bổ sung, bao gồm kết quả thực nghiệm chi tiết, mã giả thuật toán và các chứng minh lý thuyết bổ sung.

 
