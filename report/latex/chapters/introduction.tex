\chapter{Giới thiệu đề tài}

\section{Thực trạng chung}

Trong những năm gần đây, lĩnh vực trí tuệ nhân tạo ngày càng phát triển nhanh chóng. Việc ứng dụng thành quả của trí tuệ nhân tạo ngày càng được phổ biến rộng rãi, không chỉ trong đời sống hằng ngày mà cả trong công việc. Khi các ứng dụng học máy không còn bị giới hạn trong phòng thí nghiệm nữa để được ứng dụng vào trong đời sống trong các lĩnh vực sản xuất như bảo trì thông minh và kiểm soát chất lượng. Khi đó, các câu hỏi liên quan đến độ tin cậy và độ bền liên tục của chúng nảy sinh.

Các tập dữ liệu tĩnh được sử dụng để huấn luyện các mô hình học máy chỉ có thể nắm bắt được một phần nhỏ các điều kiện có thể xảy ra trong thế giới thực. Các trường hợp trôi dạt khái niệm (concept drift), chẳng hạn như thay đổi điều kiện môi trường, thiết bị và vận hành có thể, theo thời gian, làm giảm đáng kể hiệu suất của các mô hình học máy, gây ảnh hưởng đến sự an toàn, độ tin cậy của mô hình và kinh tế nếu không được giải quyết đúng cách. Do đó, cần phải (1) phát hiện sự trôi dạt sớm nhất có thể và (2) điều chỉnh hiệu quả mô hình theo các điều kiện thay đổi động.

\section{Tổng quan về bài toán}

Trong lĩnh vực xử lý dữ liệu, việc xử lý và phân tích dữ liệu streaming đang trở thành một thách thức lớn do tính phức tạp và sự đặc biệt của dữ liệu này. Một thách thức lớn trong quá trình xử lý dữ liệu streaming là việc xử lý hiện tượng trôi dạt - một hiện tượng xảy ra .

Đối với nhiều ứng dụng và hệ thống, việc phát hiện concept drift trong thời gian thực hoặc gần thời gian thực là một yêu cầu quan trọng để đảm bảo chất lượng và độ tin cậy của kết quả phân tích, từ đó đưa ra các quyết định chính xác. Tuy nhiên, việc thực hiện điều này trong môi trường streaming đòi hỏi phải xử lý dữ liệu nhanh chóng và hiệu quả, với khả năng mở rộng và cập nhật liên tục.

Trong đề tài này, luận văn tập trung vào việc nghiên cứu và phát triển phương pháp phát hiện concept drift sử dụng Shape Drift Detector (ShapeDD) dựa trên Maximum Mean Discrepancy (MMD) để giải quyết bài toán phát hiện concept drift trong dữ liệu streaming. Điều này đòi hỏi sự kết hợp giữa sức mạnh của MMD trong việc so sánh phân phối dữ liệu và các kỹ thuật tối ưu hóa để xử lý dữ liệu streaming một cách hiệu quả và linh hoạt.


\section{Mục tiêu đề tài}

\subsection{Nghiên cứu và phân tích Shape Drift Detector}

Mục tiêu đầu tiên của đề tài là tiến hành một nghiên cứu chi tiết và phân tích sâu về phương pháp Shape Drift Detector (ShapeDD) hiện đại. Luận văn sẽ xem xét cơ sở lý thuyết của ShapeDD, bao gồm Maximum Mean Discrepancy (MMD) trong không gian Reproducing Kernel Hilbert Space (RKHS), các phương pháp triển khai, cũng như cấu trúc và tính linh hoạt của chúng trong việc phát hiện concept drift. Bằng cách này, luận văn sẽ làm rõ hơn về sức mạnh và giới hạn của ShapeDD để làm cơ sở cho việc tinh chỉnh, cải tiến mô hình phát hiện.

\subsection{Ứng dụng mô hình ShapeDD cho dữ liệu streaming}

Sau khi hiểu rõ về cơ sở lý thuyết của ShapeDD, luận văn sẽ tiến hành ứng dụng với mô hình ShapeDD được tinh chỉnh đặc biệt cho việc xử lý dữ liệu streaming và phát hiện concept drift. Mô hình này sẽ được thiết kế để đảm bảo khả năng mở rộng, linh hoạt và hiệu suất cao trong môi trường streaming dữ liệu, bao gồm khả năng tự động học và điều chỉnh để thích nghi với sự biến đổi của dữ liệu theo thời gian.

\subsection{Kết hợp các kỹ thuật từ MMD để phát hiện concept drift}

Luận văn sẽ nghiên cứu và áp dụng các kỹ thuật tiên tiến từ Maximum Mean Discrepancy (MMD) để nâng cao khả năng phát hiện concept drift. Điều này bao gồm việc tối ưu hóa các tham số kernel, quản lý cửa sổ trượt, và kết hợp với các phương pháp thống kê để tăng độ chính xác và giảm false alarm trong quá trình phát hiện.

\subsection{Tùy chỉnh và tối ưu hóa mô hình}

Dựa trên kết quả nghiên cứu và thử nghiệm, luận văn sẽ tiến hành tùy chỉnh và tối ưu hóa mô hình ShapeDD để phù hợp với các đặc điểm riêng biệt của dữ liệu streaming và các loại concept drift khác nhau. Quá trình này sẽ bao gồm việc điều chỉnh các tham số, cải thiện thuật toán, và phát triển các chiến lược thích ứng để đạt được hiệu suất tối ưu trong việc phát hiện concept drift.

\subsection{Triển khai và đánh giá}

Cuối cùng, luận văn sẽ triển khai mô hình đã được tối ưu hóa và tiến hành đánh giá toàn diện trên các tập dữ liệu synthetic và thực tế. Quá trình đánh giá sẽ bao gồm việc so sánh hiệu suất với các phương pháp hiện có, phân tích độ chính xác, thời gian phát hiện, và khả năng thích ứng của mô hình trong các điều kiện khác nhau.

\section{Đóng góp của nghiên cứu}

Nghiên cứu này mang lại những đóng góp quan trọng trong lĩnh vực phát hiện concept drift:

\textbf{Đóng góp về lý thuyết:}
\begin{itemize}
    \item Phân tích toàn diện về nền tảng lý thuyết của Shape Drift Detector (ShapeDD), bao gồm nghiên cứu chi tiết về Maximum Mean Discrepancy (MMD) trong không gian Reproducing Kernel Hilbert Space (RKHS)
    \item Hình thức hóa toán học của quy trình phát hiện đa giai đoạn ShapeDD, bao gồm thu thập dữ liệu, xây dựng đặc trưng, tính toán sự khác biệt và xác thực thống kê
    \item Phân tích lý thuyết về chiến lược lựa chọn kernel và quản lý cửa sổ để đạt hiệu suất phát hiện trôi dạt tối ưu
\end{itemize}

\textbf{Đóng góp về phương pháp:}
\begin{itemize}
    \item Triển khai và tối ưu hóa chi tiết thuật toán ShapeDD cho các tình huống trôi dạt khác nhau
    \item Phát triển các phương pháp tạo tập dữ liệu synthetic toàn diện để đánh giá có kiểm soát các mẫu trôi dạt đột ngột và tăng dần
    \item Phân tích độ nhạy tham số và các chiến lược tối ưu hóa cho kích thước cửa sổ, tham số kernel và ngưỡng ý nghĩa thống kê
\end{itemize}

\textbf{Đóng góp về thực nghiệm:}
\begin{itemize}
    \item Đánh giá thực nghiệm mở rộng trên tập dữ liệu synthetic chứng minh hiệu quả của ShapeDD trên các loại trôi dạt khác nhau
    \item Phân tích so sánh hiệu suất ShapeDD trong các điều kiện khác nhau, bao gồm các kích thước cửa sổ và độ lớn trôi dạt khác nhau
    \item Hướng dẫn thực tiễn để triển khai ShapeDD trong các ứng dụng thực tế, bao gồm lựa chọn tham số và các chiến lược tối ưu hiệu suất
\end{itemize}

\section{Giới hạn đề tài và đối tượng nghiên cứu}

\subsection{Phạm vi đề tài}

Nghiên cứu này tập trung vào các phạm vi cụ thể sau:

\textbf{Phạm vi về phương pháp:}
\begin{itemize}
    \item Nghiên cứu sâu về thuật toán Shape Drift Detector (ShapeDD) dựa trên Maximum Mean Discrepancy (MMD)
    \item Tập trung vào phát hiện concept drift trong môi trường dữ liệu streaming
    \item Không bao gồm các phương pháp adaptation sau khi phát hiện drift
\end{itemize}

\textbf{Phạm vi về dữ liệu:}
\begin{itemize}
    \item Chủ yếu sử dụng tập dữ liệu synthetic có kiểm soát để đánh giá hiệu suất
    \item Tập trung vào dữ liệu số với các loại concept drift: đột ngột, tăng dần, và tuần hoàn
    \item Không bao gồm dữ liệu phi cấu trúc như text, image hoặc audio
\end{itemize}

\textbf{Phạm vi về đánh giá:}
\begin{itemize}
    \item Đánh giá hiệu suất phát hiện drift về mặt độ chính xác, thời gian phát hiện và tỷ lệ false alarm
    \item So sánh với các baseline methods cơ bản
    \item Không bao gồm đánh giá về computational complexity chi tiết hoặc scalability trên big data
\end{itemize}

\subsection{Đối tượng nghiên cứu}

\textbf{Đối tượng chính:} Thuật toán Shape Drift Detector (ShapeDD) và ứng dụng của nó trong phát hiện concept drift

\textbf{Các đối tượng cụ thể bao gồm:}
\begin{itemize}
    \item Cơ sở lý thuyết Maximum Mean Discrepancy (MMD) trong RKHS
    \item Các thành phần của thuật toán ShapeDD: kernel selection, window management, statistical validation
    \item Phương pháp tối ưu hóa tham số cho ShapeDD
    \item Tập dữ liệu synthetic mô phỏng các loại concept drift khác nhau
    \item Metrics đánh giá hiệu suất phát hiện drift
\end{itemize}

\section{Cấu trúc luận văn}

Luận văn được tổ chức thành sáu chương chính:

\textbf{Chương 1: Giới thiệu đề tài} trình bày tổng quan về nghiên cứu, bao gồm đặt vấn đề, mục tiêu nghiên cứu, đóng góp của nghiên cứu và cấu trúc luận văn.

\textbf{Chương 2: Các công trình liên quan} cung cấp khảo sát toàn diện về nghiên cứu hiện có trong lĩnh vực phát hiện và thích ứng concept drift. Chúng tôi xem xét sự phát triển của lĩnh vực, phân loại các phương pháp hiện có và xác định các hạn chế hiện tại cũng như khoảng trống nghiên cứu.

\textbf{Chương 3: Cơ sở lý thuyết} trình bày nền tảng lý thuyết của nghiên cứu, bao gồm phân tích chi tiết về Maximum Mean Discrepancy (MMD), thuật toán Shape Drift Detector (ShapeDD) và các khái niệm toán học liên quan.

\textbf{Chương 4: Mô hình đề xuất} trình bày chi tiết về mô hình phát hiện concept drift được đề xuất, bao gồm kiến trúc thuật toán, chiến lược tối ưu hóa tham số và khung triển khai thực tế.

\textbf{Chương 5: Thực nghiệm và đánh giá} báo cáo các kết quả từ đánh giá thực nghiệm mở rộng trên tập dữ liệu synthetic và thực tế. Chúng tôi phân tích hiệu suất của phương pháp đề xuất và so sánh với các phương pháp hiện có.

\textbf{Chương 6: Kết luận và hướng phát triển} tóm tắt các đóng góp chính của nghiên cứu này, thảo luận về các hạn chế và phác thảo các hướng đầy hứa hẹn cho nghiên cứu tương lai.

Luận văn kết thúc với các phụ lục chứa tài liệu bổ sung, bao gồm kết quả thực nghiệm chi tiết, mã giả thuật toán và các chứng minh lý thuyết bổ sung.

 
