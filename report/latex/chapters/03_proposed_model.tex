\chapter{Mô hình đề xuất cho hệ thống phát hiện và thích ứng concept drift}
\label{chap:proposed-model}

\section{Tổng quan kiến trúc hệ thống}

Hệ thống phát hiện và thích ứng concept drift được đề xuất trong luận văn này được xây dựng dựa trên kiến trúc streaming thời gian thực sử dụng Apache Kafka làm nền tảng xử lý luồng dữ liệu. Như đã trình bày trong Chapter 2, Apache Kafka cung cấp các đặc tính quan trọng cho hệ thống phát hiện drift: high throughput, low latency, durability, scalability, và replay capability. Chương này trình bày chi tiết cách triển khai hệ thống phát hiện drift sử dụng Kafka và các phương pháp phát hiện đã được nghiên cứu.

\textit{Lưu ý:} Nền tảng lý thuyết về Apache Kafka (kiến trúc, Producer-Consumer model, các đặc điểm kỹ thuật) và các phương pháp phát hiện drift (ShapeDD gốc và SNR-Adaptive) đã được trình bày chi tiết trong Chapter 2. Chapter này tập trung vào triển khai thực tế và tích hợp các thành phần trong hệ thống thời gian thực.

\subsection{Kiến trúc tổng thể}

Hệ thống được thiết kế theo mô hình pipeline với các thành phần độc lập giao tiếp qua Kafka message queue:

\begin{enumerate}
    \item \textbf{Producer (Bộ phát dữ liệu):} Tạo ra luồng dữ liệu liên tục với các điểm drift được kiểm soát, sử dụng hàm \texttt{gen\_random} để sinh dữ liệu tổng hợp.
    
    \item \textbf{Kafka Broker:} Quản lý hai topic chính:
    \begin{itemize}
        \item \texttt{sensor.stream}: Luồng dữ liệu đầu vào
        \item \texttt{drift.results}: Kết quả phát hiện drift
    \end{itemize}
    
    \item \textbf{Consumer - ShapeDD Detector:} Nhận dữ liệu từ Kafka, thực hiện phát hiện drift theo batch, và phân loại loại drift.
    
    \item \textbf{Adaptor (Bộ thích ứng mô hình):} Lắng nghe sự kiện drift, chọn chiến lược thích ứng phù hợp và cập nhật mô hình.
    
    \item \textbf{Real-time Visualization:} Hiển thị trực quan kết quả phát hiện và hiệu suất mô hình theo thời gian thực.
\end{enumerate}

\subsection{Luồng xử lý dữ liệu}

Quy trình xử lý dữ liệu trong hệ thống tuân theo các bước sau:

\begin{enumerate}
    \item Producer tạo dữ liệu với chỉ số drift (drift indicator) và gửi vào topic \texttt{sensor.stream}
    \item Consumer đọc dữ liệu, lưu vào buffer tuần hoàn (circular buffer) với kích thước cấu hình
    \item Khi đủ BUFFER\_SIZE mẫu, Consumer thực hiện phân tích ShapeDD trên toàn bộ batch
    \item Nếu phát hiện drift, hệ thống phân loại loại drift dựa trên phương pháp CDT\_MSW
    \item Kết quả phát hiện (bao gồm vị trí drift, p-value, loại drift) được ghi vào CSV và publish lên topic \texttt{drift.results}
    \item Adaptor nhận sự kiện drift, chọn chiến lược thích ứng dựa trên loại drift
    \item Mô hình được cập nhật và lưu lại cho inference tiếp theo
\end{enumerate}

\subsection{Cấu hình hệ thống}

Các tham số quan trọng của hệ thống:

\begin{table}[H]
\centering
\caption{Tham số cấu hình hệ thống}
\label{tab:system-config}
\begin{tabular}{lll}
\toprule
\textbf{Tham số} & \textbf{Giá trị} & \textbf{Ý nghĩa} \\
\midrule
BUFFER\_SIZE & 1000 & Số mẫu xử lý mỗi batch \\
CHUNK\_SIZE & 250 & Kích thước chunk cho phân tích drift \\
SHAPE\_L1 & 50 & Nửa cửa sổ cho ShapeDD \\
SHAPE\_L2 & 250 & Cửa sổ đầy đủ cho MMD \\
SHAPE\_N\_PERM & 2500 & Số lần hoán vị cho kiểm định \\
DRIFT\_PVALUE & 0.05 & Ngưỡng p-value phát hiện drift \\
\bottomrule
\end{tabular}
\end{table}

\section{Triển khai hệ thống Kafka cho phát hiện drift}

\subsection{Cài đặt môi trường Kafka}

Hệ thống sử dụng Docker Compose để triển khai Kafka cluster, đảm bảo tính nhất quán và dễ dàng tái tạo môi trường. Cấu hình bao gồm:

\begin{itemize}
    \item \textbf{ZooKeeper}: Quản lý metadata và coordination cho Kafka cluster
    \item \textbf{Kafka Broker}: Single broker cho development/testing (có thể scale lên nhiều broker cho production)
    \item \textbf{Network configuration}: Internal network cho communication giữa các container
    \item \textbf{Volume mapping}: Persist data và logs
\end{itemize}

File cấu hình \texttt{docker-compose.yml} định nghĩa các service và dependencies. Kafka được expose trên port 9092 cho external clients và 29092 cho inter-broker communication.

\subsection{Triển khai Producer}

Producer component (\texttt{producer.py}) chịu trách nhiệm sinh dữ liệu streaming và gửi vào Kafka topic \texttt{sensor.stream}. Thiết kế của Producer bao gồm:

\textbf{Khởi tạo Kafka Producer:}
\begin{itemize}
    \item Sử dụng \texttt{kafka-python} library để connect tới Kafka broker
    \item Cấu hình serialization: JSON format cho messages
    \item Retry logic và error handling cho network failures
    \item Batch configuration để tối ưu throughput
\end{itemize}

\textbf{Sinh dữ liệu với drift:}
\begin{itemize}
    \item Sử dụng hàm \texttt{gen\_random} để sinh synthetic data với controlled drift
    \item Mỗi message bao gồm: timestamp, features, drift\_indicator
    \item Drift được inject tại các điểm định sẵn với các loại khác nhau (abrupt, gradual, incremental)
    \item Producer gửi messages với rate cấu hình (ví dụ: 100 messages/second)
\end{itemize}

\textbf{Message format:}
\begin{verbatim}
{
    "timestamp": 1234567890,
    "features": [0.234, 0.567, ...],
    "drift_indicator": 0,  // 0: normal, 1: drift region
    "drift_type": "abrupt"
}
\end{verbatim}

\subsection{Triển khai Consumer với ShapeDD}

Consumer component (\texttt{consumer\_stream.py}) đọc dữ liệu từ Kafka và thực hiện drift detection sử dụng ShapeDD. Kiến trúc consumer:

\textbf{Kafka Consumer configuration:}
\begin{itemize}
    \item Subscribe vào topic \texttt{sensor.stream}
    \item Consumer group: \texttt{drift-detector-group} (cho phép scale horizontal)
    \item Auto-commit offset sau khi xử lý thành công
    \item Deserialize JSON messages thành Python objects
\end{itemize}

\textbf{Buffer-based processing:}
\begin{itemize}
    \item Maintain circular buffer với BUFFER\_SIZE=750 samples
    \item Check drift mỗi CHECK\_FREQUENCY=150 samples
    \item Sliding window với L1=50, L2=150 cho ShapeDD detection
    \item COOLDOWN=75 samples để tránh chattering (multiple detections cho cùng drift event)
\end{itemize}

\textbf{ShapeDD detection logic:}
\begin{itemize}
    \item Khi buffer đầy, gọi \texttt{shape\_snr\_adaptive()} từ module \texttt{shape\_dd.py}
    \item Thuật toán tự động ước lượng SNR và chọn strategy (aggressive/conservative)
    \item Nếu drift detected (p-value < 0.05), publish event lên topic \texttt{drift.results}
    \item Log detection details: timestamp, p-value, strategy used, window statistics
\end{itemize}

\textbf{Event publishing:}
\begin{verbatim}
{
    "detection_time": 1234567890,
    "p_value": 0.012,
    "strategy": "aggressive",
    "snr_estimate": 0.015,
    "drift_location": 5432
}
\end{verbatim}

\subsection{Triển khai Model Adaptor}

Adaptor component (\texttt{adaptor.py}) lắng nghe drift events và trigger model adaptation. Thiết kế của Adaptor:

\textbf{Drift event listener:}
\begin{itemize}
    \item Subscribe vào topic \texttt{drift.results}
    \item Parse drift event để xác định loại drift và severity
    \item Maintain drift history để tránh over-adaptation
\end{itemize}

\textbf{Adaptation strategies:}
\begin{itemize}
    \item \textbf{Abrupt drift}: Full model retrain với recent window
    \item \textbf{Gradual drift}: Incremental update với weighted samples
    \item \textbf{Incremental drift}: Online learning với adaptive learning rate
    \item \textbf{Recurrent drift}: Retrieve previous model từ model repository
\end{itemize}

\textbf{Model management:}
\begin{itemize}
    \item Versioned model storage (MLflow hoặc filesystem)
    \item A/B testing framework để so sánh old vs new model
    \item Gradual rollout của updated model
    \item Rollback mechanism nếu performance degrades
\end{itemize}

\subsection{Communication qua Kafka Topics}

Hệ thống sử dụng multiple Kafka topics để decouple các components:

\begin{table}[H]
\centering
\caption{Kafka topics trong hệ thống}
\begin{tabular}{lll}
\toprule
\textbf{Topic} & \textbf{Producer} & \textbf{Consumer} \\
\midrule
sensor.stream & Data Producer & Drift Detector \\
drift.results & Drift Detector & Model Adaptor \\
model.updates & Model Adaptor & Inference Service \\
metrics.monitoring & All components & Monitoring Dashboard \\
\bottomrule
\end{tabular}
\end{table}

Thiết kế multi-topic này cho phép:
\begin{itemize}
    \item Tách biệt concerns (data ingestion, detection, adaptation, monitoring)
    \item Scale independently các components (horizontal scaling)
    \item Replay data cho debugging hoặc retraining
    \item Multiple consumers có thể subscribe cùng topic (ví dụ: logging, alerting)
\end{itemize}

\subsection{Ưu điểm của kiến trúc Kafka-based}

Việc sử dụng Apache Kafka làm backbone cho hệ thống phát hiện drift mang lại các lợi ích quan trọng:

\begin{enumerate}
    \item \textbf{Decoupling}: Producer, Consumer, Adaptor hoạt động độc lập, có thể develop và deploy riêng
    \item \textbf{Scalability}: Có thể scale từng component dựa trên bottleneck (ví dụ: nhiều detector consumers cho high-volume streams)
    \item \textbf{Fault tolerance}: Kafka replication đảm bảo no data loss khi có failures
    \item \textbf{Replay capability}: Có thể re-process historical data để tune parameters hoặc test new detectors
    \item \textbf{Low latency}: End-to-end latency từ data ingestion đến drift detection < 100ms (với buffer size thích hợp)
    \item \textbf{Observability}: Kafka metrics (lag, throughput, partition status) cung cấp visibility vào system health
\end{enumerate}

Kiến trúc này đã được triển khai và testing trong folder \texttt{drift-monitoring/} của repository, với đầy đủ các component producer, consumer, adaptor, và configuration files.

\section{Phương pháp CDT\_MSW – Nhận diện loại trôi dạt}
\label{sec:cdt-msw}

\subsection{Tổng quan}

Trong khi các thuật toán như DDM, ADWIN hay ShapeDD chỉ xác định \emph{thời điểm xảy ra trôi dạt}, thì CDT\_MSW đi xa hơn bằng cách \textbf{nhận dạng loại trôi dạt} dựa trên hình thái thay đổi của phân phối dữ liệu theo thời gian. Phương pháp \textit{Concept Drift Type Identification based on Multi–Sliding Windows (CDT\_MSW)} được đề xuất bởi Guo và cộng sự~\cite{guo2022cdtmsw} nhằm mở rộng chức năng của các bộ phát hiện trôi dạt thông thường. Phương pháp này đặc biệt hữu ích cho các hệ thống học thích ứng (adaptive learning systems), nơi mà mỗi loại trôi dạt đòi hỏi một chiến lược cập nhật mô hình khác nhau.

\subsection{Nguyên lý hoạt động}
Phương pháp CDT\_MSW sử dụng cấu trúc \textbf{cửa sổ trượt đa mức (multi–sliding windows)} để theo dõi sự thay đổi của phân phối dữ liệu trước và sau điểm trôi dạt. 
Ý tưởng chính là quan sát sự tiến hóa của khoảng cách phân phối khi cửa sổ so sánh mở rộng dần theo thời gian. 
Quá trình nhận diện loại trôi dạt bao gồm ba giai đoạn chính (Hình~\ref{fig:cdt-msw}).

\begin{enumerate}
    \item \textbf{Phát hiện ban đầu (Detection):} 
    Bộ phát hiện trôi dạt (ví dụ ShapeDD) phát hiện một điểm thay đổi tại thời điểm $t_0$. 
    Một cửa sổ tham chiếu $\mathcal{W}_{ref}$ được lấy từ dữ liệu trước thời điểm $t_0$ để biểu diễn khái niệm cũ.
    
    \item \textbf{Giai đoạn mở rộng (Growth process):} 
    Một cửa sổ quan sát $\mathcal{W}_{post}$ bắt đầu từ $t_0$ được mở rộng dần về phía sau dòng dữ liệu. 
    Ở mỗi bước, khoảng cách giữa hai phân phối $P_{ref}$ và $P_{post}$ được tính theo một độ đo thống kê như:
    \begin{equation}
        d_t = D\left(P(X \in \mathcal{W}_{ref}), P(X \in \mathcal{W}_{post}(t))\right),
    \end{equation}
    với $D(\cdot,\cdot)$ là độ đo khoảng cách phân phối, chẳng hạn như Kolmogorov–Smirnov (KS) hoặc Maximum Mean Discrepancy (MMD).
    Khi sai biệt giữa hai bước liên tiếp $\Delta d_t = |d_t - d_{t-1}|$ giảm xuống dưới ngưỡng ổn định $\delta$ trong $p$ bước liên tục, quá trình mở rộng dừng lại. 
    Độ dài vùng chuyển tiếp $L_d = t_{stable} - t_0$ được xem là \emph{độ dài trôi dạt (drift length)}.

    \item \textbf{Theo dõi (Tracking process):}
    Sau khi xác định được $L_d$, thuật toán tiếp tục trượt cửa sổ $\mathcal{W}_{post}$ để theo dõi mức độ khác biệt $r(t)$ giữa phân phối hiện tại và phân phối trước trôi dạt:
    \begin{equation}
        r(t) = D\left(P(X \in \mathcal{W}_{ref}), P(X \in \mathcal{W}_{post}(t))\right),
    \end{equation}
    giúp xác định xem khái niệm mới có ổn định, dao động, hay quay lại trạng thái cũ.
\end{enumerate}

% TODO: Tạo sơ đồ CDT_MSW framework (3 giai đoạn: Detection – Growth – Tracking)
% Có thể sử dụng TikZ hoặc draw.io để vẽ workflow diagram
% \begin{figure}[H]
%     \centering
%     \includegraphics[width=0.9\linewidth]{figures/cdt_msw_framework.pdf}
%     \caption{Sơ đồ nguyên lý của phương pháp CDT\_MSW với ba giai đoạn: Detection – Growth – Tracking.}
%     \label{fig:cdt-msw}
% \end{figure}

\subsection{Quy tắc phân loại loại trôi dạt}
Dựa trên độ dài trôi dạt $L_d$ và hình dạng của đường cong $r(t)$, CDT\_MSW phân loại các loại trôi dạt theo các quy tắc sau:

\begin{itemize}
    \item \textbf{Trôi dạt đột ngột (Sudden drift):} 
    $L_d < \theta_{sudden}$, giá trị $r(t)$ tăng nhanh và đạt trạng thái ổn định trong thời gian ngắn.
    \item \textbf{Trôi dạt dần dần (Gradual drift):} 
    $L_d > \theta_{sudden}$, $r(t)$ dao động quanh ngưỡng ổn định và thể hiện sự xen kẽ giữa hai khái niệm.
    \item \textbf{Trôi dạt tăng dần (Incremental drift):} 
    Dạng đặc biệt của gradual drift, trong đó $r(t)$ thay đổi đơn điệu (ít dao động ngược chiều).
    \item \textbf{Trôi dạt lặp lại (Recurring drift):} 
    Sau khi khái niệm mới ổn định, $r(t)$ giảm xuống dưới ngưỡng tương đồng $\epsilon_{recur}$ và duy trì trong thời gian đủ dài $T_{recur}$, biểu thị sự quay lại của khái niệm cũ.
    \item \textbf{Trôi dạt tạm thời (Blip drift):} 
    Sự thay đổi ngắn hạn với $L_d < \theta_{blip}$, khái niệm nhanh chóng quay lại trạng thái ban đầu.
\end{itemize}

\subsection{Các tham số của CDT\_MSW}
Phương pháp CDT\_MSW không yêu cầu đặt ngưỡng cố định cho tất cả các luồng dữ liệu; 
các tham số được điều chỉnh linh hoạt dựa trên tốc độ thay đổi của luồng và độ nhiễu của dữ liệu. 
Bảng~\ref{tab:cdt-msw-params} liệt kê các tham số thường dùng và gợi ý giá trị khởi tạo được đề xuất trong~\cite{guo2022cdtmsw}.

\begin{table}[H]
\centering
\caption{Các tham số chính trong phương pháp CDT\_MSW.}
\label{tab:cdt-msw-params}
\begin{tabular}{lll}
\toprule
\textbf{Ký hiệu} & \textbf{Ý nghĩa} & \textbf{Giá trị gợi ý} \\
\midrule
$w_{ref}$ & Cửa sổ tham chiếu trước drift & 200 mẫu \\
$w_{basic}$ & Cửa sổ so sánh cơ bản & 50 mẫu \\
$\delta$ & Ngưỡng ổn định biến thiên & 0.02 \\
$\theta_{sudden}$ & Ngưỡng phân biệt sudden/progressive & 60 mẫu \\
$\epsilon_{recur}$ & Ngưỡng tương đồng khi khái niệm quay lại & 0.15 \\
$T_{recur}$ & Độ dài tối thiểu để xem là recurrent & 120 mẫu \\
$T_{blip}$ & Độ dài tối đa cho blip drift & 60 mẫu \\
\bottomrule
\end{tabular}
\end{table}

\subsection{Ưu điểm của CDT\_MSW}
So với các phương pháp truyền thống, CDT\_MSW có một số ưu điểm nổi bật:
\begin{itemize}
    \item Không phụ thuộc vào mô hình học máy; hoạt động thuần túy trên phân phối dữ liệu nên áp dụng được cho cả dữ liệu giám sát và phi giám sát.
    \item Có khả năng nhận diện \emph{nhiều loại trôi dạt khác nhau}, bao gồm đột ngột, dần dần, tăng dần, lặp lại và tạm thời.
    \item Cung cấp thông tin định lượng như độ dài trôi dạt và độ ổn định, giúp lựa chọn chiến lược cập nhật mô hình thích hợp.
\end{itemize}

\section{Mô hình học máy và quy trình thích ứng}
\label{sec:model-adaptation}

\subsection{Kiến trúc mô hình}

Hệ thống sử dụng mô hình học máy với ba giai đoạn hoạt động chính:

\textbf{1. Giai đoạn huấn luyện ban đầu (Training Phase):}
\begin{itemize}
    \item Sử dụng scikit-learn để xây dựng pipeline mô hình batch
    \item Cấu trúc: StandardScaler + LogisticRegression
    \item Huấn luyện trên dữ liệu ban đầu (pre-drift data)
    \item Model được lưu dưới dạng pickle file
\end{itemize}

\textbf{2. Giai đoạn triển khai (Deployment Phase):}
\begin{itemize}
    \item Mô hình hoạt động ở chế độ \textit{frozen} (đóng băng)
    \item Không có online learning trong quá trình inference
    \item Chỉ thực hiện prediction trên dữ liệu mới
    \item Theo dõi accuracy để phát hiện suy giảm hiệu suất
\end{itemize}

\textbf{3. Giai đoạn cập nhật (Update Phase):}
\begin{itemize}
    \item Được kích hoạt khi phát hiện drift
    \item Chiến lược cập nhật được chọn dựa trên loại drift
    \item Mô hình mới được huấn luyện hoặc cập nhật
    \item Model được lưu lại và thay thế model cũ
\end{itemize}

\subsection{Chiến lược thích ứng theo loại drift}

Sau khi nhận diện được loại trôi dạt, hệ thống tự động \textbf{chọn chiến lược cập nhật mô hình} phù hợp. 
Mục tiêu của giai đoạn này là đảm bảo mô hình học máy duy trì hiệu suất cao nhất với chi phí cập nhật tối ưu, tránh huấn luyện lại toàn bộ khi không cần thiết. 

Hệ thống triển khai năm chiến lược thích ứng chính dựa trên module \texttt{adaptation\_strategies.py}, được mô tả trong Bảng~\ref{tab:drift-adaptation}.

\begin{table}[H]
\centering
\caption{Chiến lược thích ứng mô hình theo loại trôi dạt trong hệ thống.}
\label{tab:drift-adaptation}
\begin{tabular}{p{3cm}p{5cm}p{6cm}}
\toprule
\textbf{Loại drift} & \textbf{Đặc trưng} & \textbf{Chiến lược triển khai} \\
\midrule
\textbf{Sudden} & 
Thay đổi tức thời, khái niệm cũ không còn hiệu lực &
\textbf{Full model reset:} Tạo mô hình mới hoàn toàn (sklearn pipeline mới), huấn luyện trên dữ liệu post-drift. Hàm: \texttt{adapt\_sudden\_drift()}. \\
\midrule
\textbf{Incremental} & 
Thay đổi đơn điệu, tiến triển liên tục &
\textbf{Gradual online update:} Cập nhật tuần tự từng mẫu bằng River online learning (\texttt{learn\_one}). Hàm: \texttt{adapt\_incremental\_drift()}. \\
\midrule
\textbf{Gradual} & 
Hai khái niệm xen kẽ, dao động không đơn điệu &
\textbf{Weighted update:} Ưu tiên mẫu gần đây hơn (weight = vị trí/tổng), chỉ cập nhật 50\% mẫu cuối. Hàm: \texttt{adapt\_gradual\_drift()}. \\
\midrule
\textbf{Recurrent} & 
Khái niệm cũ quay lại (theo mùa, chu kỳ) &
\textbf{Model cache \& reuse:} Tìm kiếm mô hình đã lưu có phân phối tương tự (KS distance < 0.15), fine-tune nếu tìm thấy. Cache: \texttt{models/cache/}. Hàm: \texttt{adapt\_recurrent\_drift()}. \\
\midrule
\textbf{Blip} & 
Thay đổi tạm thời, ngắn hạn &
\textbf{Minimal update:} Cập nhật bảo thủ với tối đa 5 mẫu, hoặc bỏ qua nếu không có nhãn. Hàm: \texttt{adapt\_blip\_drift()}. \\
\bottomrule
\end{tabular}
\end{table}

\subsection{Cơ chế quyết định tự động}

Trong quá trình vận hành hệ thống, việc lựa chọn chiến lược thích ứng được thực hiện tự động thông qua module \texttt{adaptor.py}. 
Quy trình hoạt động như sau:

\textbf{Bước 1: Lắng nghe sự kiện drift}
\begin{itemize}
    \item Adaptor subscribe vào Kafka topic \texttt{drift.results}
    \item Nhận message chứa thông tin: \texttt{idx}, \texttt{p\_value}, \texttt{drift\_type}, \texttt{window\_path}
    \item Kiểm tra tính hợp lệ của snapshot file
\end{itemize}

\textbf{Bước 2: Load dữ liệu drift window}
\begin{itemize}
    \item Đọc snapshot từ thư mục \texttt{./snapshots/}
    \item Snapshot chứa: ma trận đặc trưng $X$, nhãn $y$ (nếu có), tên features
    \item Kích thước window phụ thuộc vào tham số \texttt{w\_ref} và \texttt{CHUNK\_SIZE}
\end{itemize}

\textbf{Bước 3: Lựa chọn chiến lược thích ứng}

Hệ thống ánh xạ loại drift sang hàm xử lý tương ứng:

    \begin{equation}
    \text{strategy} = 
        \begin{cases}
        \texttt{adapt\_sudden\_drift()} & \text{nếu } \texttt{drift\_type} = \text{``sudden''} \\
        \texttt{adapt\_incremental\_drift()} & \text{nếu } \texttt{drift\_type} = \text{``incremental''} \\
        \texttt{adapt\_gradual\_drift()} & \text{nếu } \texttt{drift\_type} = \text{``gradual''} \\
        \texttt{adapt\_recurrent\_drift()} & \text{nếu } \texttt{drift\_type} = \text{``recurrent''} \\
        \texttt{adapt\_blip\_drift()} & \text{nếu } \texttt{drift\_type} = \text{``blip''} \\
        \texttt{adapt\_incremental\_drift()} & \text{nếu } \texttt{drift\_type} = \text{``undetermined''}
        \end{cases}
    \end{equation}

\textbf{Bước 4: Cập nhật và lưu mô hình}
\begin{itemize}
    \item Thực thi chiến lược đã chọn
    \item Lưu model mới vào \texttt{./models/current\_model.pkl}
    \item Cập nhật version (dựa trên file modification time)
    \item Publish event \texttt{model\_updated} lên topic \texttt{model.updated}
\end{itemize}

\subsection{Ví dụ quy trình thích ứng}

Giả sử tại thời điểm $t = 1504$, ShapeDD phát hiện drift với loại ``sudden'':

\begin{enumerate}
    \item Consumer phát hiện drift, phân loại là ``sudden'', gửi event:
    \begin{verbatim}
    {
      "event": "drift_detected",
      "idx": 1504,
      "p_value": 0.0001,
      "drift_type": "sudden",
      "window_path": "./snapshots/drift_window_1504_xxx.npz"
    }
    \end{verbatim}
    
    \item Adaptor nhận event, load snapshot chứa 251 mẫu (200 pre-drift + 51 post-drift)
    
    \item Gọi \texttt{adapt\_sudden\_drift()}: Tạo model mới, huấn luyện trên dữ liệu post-drift
    
    \item Lưu model mới, publish \texttt{model\_updated} event
\end{enumerate}

\subsection{Ưu điểm của cơ chế tự động}

Cơ chế quyết định tự động mang lại các lợi ích:

\begin{itemize}
    \item \textbf{Tự động hóa hoàn toàn:} Không cần can thiệp thủ công
    \item \textbf{Tối ưu hóa chi phí:} Chọn chiến lược phù hợp tránh lãng phí tài nguyên
    \item \textbf{Phản hồi nhanh:} Thời gian từ phát hiện đến cập nhật < 1 giây
    \item \textbf{Có khả năng mở rộng:} Dễ dàng thêm chiến lược mới
\end{itemize}
