\chapter{Công trình liên quan}

\section{Giới thiệu về hiện tượng trôi dạt}

Thế giới không ngừng thay đổi đặt ra những thách thức cho các hệ thống tự động, ví dụ như các hệ thống liên quan đến cơ sở hạ tầng quan trọng, sản xuất và kiểm soát chất lượng.
Việc vận hành đáng tin cậy các quy trình tự động và thuật toán giám sát đòi hỏi khả năng phát hiện, phản hồi và thích ứng với những thay đổi này (Ditzler và cộng sự, 2015; 
Reppa và cộng sự, 2016; Chen và Boning, 2017; Vrachimis và cộng sự, 2022; Gabbar và cộng sự, 2023).

Về mặt hình thức, những thay đổi trong phân phối tạo dữ liệu được gọi là trôi khái niệm (Gama và cộng sự, 2014). 
Những thay đổi này có thể do các sửa đổi trong quy trình, môi trường hoặc cảm biến thu thập dữ liệu được quan sát. 
Việc phát hiện các bất thường trong quy trình được quan sát là cần thiết để xác định các sản phẩm bị lỗi hoặc các loại lỗi không mong muốn khác. 
Ngược lại, việc phát hiện những thay đổi trong cảm biến và môi trường là rất quan trọng để các quy trình tự động thực hiện các hành động phù hợp, chẳng hạn như thay 
thế cảm biến bị lỗi hoặc sửa đổi hệ thống xử lý dữ liệu thu thập được để phù hợp với một kịch bản mới (Gama và cộng sự, 2004, 2014; Gonçalves và cộng sự, 2014).

Thông thường, độ trôi được nghiên cứu trong các thiết lập luồng, trong đó những thay đổi trong
phân phối dữ liệu cơ bản đòi hỏi phải điều chỉnh mô hình hoặc
cảnh báo người vận hành để có hành động khắc phục (Ditzler và cộng sự,
2015; Lu và cộng sự, 2018; Delange và cộng sự, 2021). Điều này liên quan chặt chẽ
đến sự phát triển của các khái niệm trong học liên tục, một chủ đề phổ biến
trong học sâu, nơi các khái niệm có thể xuất hiện hoặc biến mất. Độ trôi
mở rộng ra ngoài các luồng dữ liệu và xuất hiện trong dữ liệu chuỗi thời gian với
các quan sát phụ thuộc lẫn nhau. Độ trôi như vậy thường biểu hiện dưới dạng
xu hướng, và sự vắng mặt của nó được gọi là tính dừng (Esling và Agon,
2012; Aminikhanghahi và Cook, 2017).
Trong các bối cảnh mà dữ liệu được quan sát theo thời gian, chẳng hạn như
sản xuất và kiểm soát chất lượng, dữ liệu thường được thu thập
trên nhiều địa điểm và được áp dụng các kỹ thuật học liên kết
(Zhang và cộng sự, 2021). Thay vì hợp nhất tất cả dữ liệu trên một máy chủ toàn cầu, xử lý cục bộ được triển khai và kết quả được tích hợp vào một mô hình tổng thể. Tương tự như học theo luồng, điều quan trọng là phải giải quyết sự khác biệt hoặc độ trôi của dữ liệu từ các vị trí khác nhau để xây dựng một mô hình toàn cầu mạnh mẽ (Liu và cộng sự, 2020). Hơn nữa, độ trôi cần được tính đến trong học chuyển giao, một kỹ thuật học sâu (Pan và Yang, 2010), trong đó mô hình được huấn luyện trước trên một tác vụ tương tự với tập dữ liệu mở rộng hơn trước khi được tinh chỉnh trên tác vụ mục tiêu bằng cách sử dụng một tập dữ liệu hạn chế. Mặc dù trọng tâm chính của nghiên cứu này là các luồng dữ liệu, các chiến lược được trình bày ở đây cũng áp dụng cho các tác vụ khác.

Việc xử lý các luồng dữ liệu trôi dạt bao gồm hai nhiệm vụ chính:
thiết lập một mô hình mạnh mẽ cho các tác vụ dự đoán, tức là học trực tuyến hoặc
học theo luồng, và giám sát các hệ thống để phát hiện hành vi bất ngờ.
Trong trường hợp trước, trọng tâm là nhãn và mối quan hệ của nó với các
tính năng khác, trong khi trường hợp sau liên quan đến bất kỳ thay đổi nào cho thấy
hành vi hoặc trạng thái bất ngờ của hệ thống. Do đó, phát hiện trôi dạt
tập trung vào các mục tiêu khác nhau, tương tự như học tập chung được gọi là
có giám sát cho trường hợp trước và không giám sát cho trường hợp sau. Nghiên cứu này
bỏ qua học trực tuyến vì nó đã được khám phá rộng rãi trong các
khảo sát trước đây (Ditzler và cộng sự, 2015; Losing và cộng sự, 2018; Lu và cộng sự, 2018) và
hộp công cụ (Bifet và cộng sự, 2010; Montiel và cộng sự, 2018, 2021).
Thay vào đó, nghiên cứu này tập trung vào việc phát hiện trôi dạt không giám sát và
giám sát các tình huống trong đó trôi dạt được dự đoán do việc sử dụng cảm biến
hoặc nhạy cảm với những thay đổi của môi trường. Cụ thể, trọng tâm là phát hiện trôi dạt không giám sát, một yếu tố quan trọng để giám sát và hiểu rõ hiện tượng trôi dạt. 
Một số ứng dụng điển hình là phát hiện trôi dạt cho các ứng dụng an ninh (Yang và cộng sự, 2021) và sử dụng phát hiện trôi dạt để phát hiện rò rỉ trong mạng lưới phân phối
nước (Vaquet và cộng sự, 2024a,b). Ngoài ra, còn có các kỹ thuật để phân tích trôi dạt sâu hơn (Webb và cộng sự, 2017, 2018; Hinder và cộng sự, 2023a), mà chúng tôi sẽ không đề 
cập chi tiết trong nghiên cứu này. Đối với độc giả quan tâm, chúng tôi cung cấp một phiên bản mở rộng bao gồm các chủ đề này cũng như nội dung của nghiên cứu 
này (Hinder và cộng sự, 2023b). Lưu ý rằng các phương pháp phát hiện trôi dạt không giám sát được thảo luận ở đây khác với các phương pháp được 
thiết kế cho học tập trực tuyến, như đã được thảo luận bởi Gemaque và cộng sự (2020). 
Trong Phần 2.2, chúng tôi mô tả chi tiết hơn về sự tương phản giữa phát hiện trôi dạt có giám sát. Giám sát bao gồm việc
quan sát một hệ thống và cung cấp thông tin cần thiết cho cả người vận hành và các tác vụ tự động để đảm bảo hệ thống hoạt động bình thường. 
Thông tin cần thiết sẽ khác nhau tùy thuộc vào từng tác vụ cụ thể (Goldenberg và Webb, 2019; Verma, 2021). 
Nhìn chung, có những câu hỏi quan trọng cần trả lời liên quan đến trôi dạt (Lu và cộng sự, 2018):

\begin{equation}
  P_t(X,Y) \;\neq\; P_{t+\Delta}(X,Y).
\end{equation}
Hệ quả là một mô hình dự đoán $f$ huấn luyện tại thời điểm trước có thể suy giảm hiệu năng về sau nếu thiếu cơ chế \emph{phát hiện} và \emph{thích ứng} drift \citep{survey_partA_2024,process_monitoring_2022,deep_survey_mdpi_2023}. Dựa trên phân rã Bayes $P(X,Y)=P(Y\!\mid\!X)P(X)$, ta thường phân biệt:
\begin{itemize}
  \item \textbf{Real (concept) drift}: thay đổi trong $P(Y\!\mid\!X)$ --- tức mối quan hệ đầu vào--đầu ra thay đổi.
  \item \textbf{Virtual drift} (prior/covariate shift): thay đổi trong $P(X)$ hoặc $P(Y)$ trong khi hàm ánh xạ mục tiêu không đổi, nhưng phân phối đến khác biệt làm mô hình suy giảm \citep{survey_partA_2024}.
\end{itemize}

\section{Phân loại (taxonomy) concept drift}
\subsection{Theo mẫu thời gian}
\textbf{Đột ngột (abrupt/sudden)}: chuyển pha nhanh quanh thời điểm $t^\star$. \textbf{Dần dần (gradual)}: xác suất gặp khái niệm cũ/mới chuyển đổi từ từ trong một khoảng. \textbf{Tiệm tiến (incremental)}: ranh giới quyết định trượt liên tục theo thời gian (không có pha rõ ràng). \textbf{Lặp lại (recurring/seasonal)}: khái niệm cũ tái xuất theo chu kỳ hay theo bối cảnh \citep{msw_type_2022,survey_partA_2024}. Các định nghĩa có thể lượng hoá bằng cường độ, tốc độ, thời lượng và chu kỳ drift \citep{survey_partA_2024}.

\subsection{Theo mức tác động phân phối}
\textbf{Prior drift} (đổi $P(Y)$), \textbf{covariate shift} (đổi $P(X)$), và \textbf{concept shift} (đổi $P(Y\!\mid\!X)$). Detector dựa lỗi nhạy với concept shift; detector dựa phân phối có thể phát hiện cả prior/covariate \citep{survey_partA_2024}.

\subsection{Theo nội dung phát sinh}
\textbf{Concept evolution}: xuất hiện lớp mới/khái niệm mới; liên quan chặt với phát hiện \textit{novelty}/OOD và bối cảnh bán giám sát \citep{semi_supervised_recurring_2011,ussm_ssl_2021}.

\section{Mục tiêu phát hiện và lượng hoá}
Một hệ phát hiện hữu dụng thường cần: (i) \textbf{định vị} thời điểm/độ dài thay đổi; (ii) \textbf{định cỡ} (severity) để cân nhắc hành động; (iii) \textbf{nhận dạng loại drift} nhằm chọn chiến lược thích ứng phù hợp; và (iv) đảm bảo \textbf{độ tin cậy} (kiểm soát \textit{false alarm} và \textit{detection delay}) trong môi trường có tương quan theo thời gian \citep{survey_partA_2024,deep_survey_ijcai_2022}.

\section{Đánh giá trong bối cảnh dòng dữ liệu}
\subsection{Quy trình \textit{prequential} (test-then-train)}
Tại thời điểm $t$, dự đoán trước rồi mới cập nhật mô hình, cho phép theo dõi độ chính xác/lỗi \emph{prequential} theo thời gian và đo phản ứng của detector/adaptor \citep{survey_partA_2024}.

\subsection{Chỉ số cho detector}
\textbf{MTTD} (Mean Time/Delay to Detection), \textbf{TP/FP/FN} trên tập drift biết trước (synthetic), \textbf{tỷ lệ báo động giả}, \textbf{tỷ lệ bỏ sót}, kèm \textbf{thời gian/ bộ nhớ} \citep{survey_partA_2024,deep_survey_mdpi_2023}.

\section{Gia đình phương pháp (dẫn dắt từ lý thuyết)}

\subsection{Dựa trên lỗi mô hình (supervised error-rate)}
Theo dõi chuỗi lỗi $e_t\in\{0,1\}$ của mô hình nền và phát hiện khi thống kê vượt ngưỡng.
\begin{itemize}
  \item \textbf{DDM} \citep{ddm_2004}: kiểm soát tần suất lỗi $p_t$ và độ lệch $s_t=\sqrt{p_t(1-p_t)/t}$; kích hoạt mức \emph{warning/drift} khi $p_t+s_t$ vượt ngưỡng so với cực trị đã quan sát.
  \item \textbf{EDDM} \citep{eddm_2006}: theo dõi \emph{khoảng cách giữa các lỗi liên tiếp}, nhạy hơn với \emph{gradual drift}.
\end{itemize}
Ưu: đơn giản, sát \emph{hiệu năng}. Nhược: đòi \emph{nhãn tức thời} --- thường khó trong công nghiệp/thực địa \citep{survey_partA_2024}.

\subsection{Kiểm định phân phối/cửa sổ (unsupervised distributional)}
So sánh thống kê giữa \emph{cửa sổ tham chiếu} và \emph{cửa sổ hiện tại}; không cần nhãn.
\begin{itemize}
  \item \textbf{ADWIN} \citep{adwin_2007}: cửa sổ \emph{tự thích ứng}; tách hai bán cửa sổ và dùng bất đẳng thức Hoeffding để kiểm định chênh trung bình; có bảo đảm lý thuyết về báo động giả.
  \item \textbf{FHDDM/FHDDMS} \citep{fhddms_2018}: dùng Hoeffding trên cửa sổ trượt và \emph{xếp chồng đa cửa sổ} để vừa nhạy drift ngắn hạn vừa bền trước nhiễu dài hạn.
  \item \textbf{MDDM} \citep{mddm_2017}: dùng bất đẳng thức \emph{McDiarmid} với \emph{trọng số lớn hơn cho quan sát gần hiện tại}, giảm độ trễ trong khi kiểm soát sai số.
\end{itemize}

\subsection{Phân biệt hai-mẫu (two-sample discriminative)}
Huấn luyện một bộ \emph{phân loại phân biệt} giữa $W_{\text{ref}}$ và $W_{\text{cur}}$; nếu phân biệt dễ (AUC/accuracy cao), coi là drift.
\begin{itemize}
  \item \textbf{D3} \citep{d3_2019}: một đại diện tiêu biểu, áp dụng tốt cho dữ liệu nhiều chiều; cần kiểm soát \emph{overfitting} và phụ thuộc động lực học cửa sổ.
\end{itemize}

\subsection{Độc lập/không tham số}
Liên hệ ``có drift'' $\Leftrightarrow$ ``mất độc lập'' của biến ngẫu nhiên trong quá trình; kiểm định độc lập trên cửa sổ thích ứng.
\begin{itemize}
  \item \textbf{DAWIDD} \citep{dawidd}: thiết lập tương đương toán học và đề xuất detector \emph{không tham số} bất nhãn.
\end{itemize}

\subsection{Hình thái tín hiệu (shape-based) \& đa thang}
\begin{itemize}
  \item \textbf{ShapeDD} \citep{shapedd_2021}: trích \emph{hình dạng} chuỗi thời gian kèm \emph{khử nhiễu}, làm nổi bật thay đổi cấu trúc trong dữ liệu cảm biến ồn.
  \item \textbf{Nhận dạng loại drift} bằng multi-sliding windows (CDT\_MSW) \citep{msw_type_2022}: pipeline phát hiện $\rightarrow$ ước lượng độ dài $\rightarrow$ phân loại \emph{abrupt/gradual/incremental/recurring}.
\end{itemize}

\subsection{Bán giám sát \& tiến hoá khái niệm}
Khi nhãn khan hiếm, các khung \emph{semi-supervised} xử lý đồng thời drift lặp lại và xuất hiện lớp mới (concept evolution). Ví dụ \citep{semi_supervised_recurring_2011,ussm_ssl_2021}.

\section{Bất đẳng thức nòng cốt}
\textbf{Hoeffding} (dùng trong ADWIN/FHDDM/S):
\begin{equation}
  \Pr(\overline{X}-\mu \ge \epsilon) \le \exp(-2n\epsilon^2).
\end{equation}
\textbf{McDiarmid} (dùng trong MDDM):
\begin{equation}
  \Pr(f(X)-\mathbb{E}f(X) \ge \epsilon) \le \exp\!\Big(-\frac{2\epsilon^2}{\sum_i c_i^2}\Big),
\end{equation}
với $c_i$ là biên độ ảnh hưởng khi thay đổi phần tử thứ $i$.

\section{Dòng thời gian phát triển (2004--2025) và ứng dụng miền}
\paragraph{2004--2007:} \textbf{DDM} \citep{ddm_2004}, \textbf{EDDM} \citep{eddm_2006} đặt nền ``giám sát lỗi có nhãn''; \textbf{ADWIN} \citep{adwin_2007} mở kỷ nguyên ``không nhãn với cửa sổ thích ứng''.\\
\paragraph{2017--2019:} \textbf{FHDDMS} \citep{fhddms_2018} (đa cửa sổ) và \textbf{MDDM} \citep{mddm_2017} (McDiarmid + trọng số) giảm độ trễ và tăng ổn định; \textbf{D3} \citep{d3_2019} đại diện hướng phân biệt hai-mẫu.\\
\paragraph{2020--2022:} \textbf{DAWIDD} \citep{dawidd} (độc lập không tham số); \textbf{CDT\_MSW} \citep{msw_type_2022} (nhận dạng loại drift).\\
\paragraph{2021--2025:} \textbf{ShapeDD} \citep{shapedd_2021} cho dữ liệu cảm biến ồn; các \emph{review} về thích ứng trong \emph{Deep Learning} \citep{deep_survey_ijcai_2022,deep_survey_mdpi_2023} và khung ứng dụng: giám sát quy trình \citep{process_monitoring_2022}, IoT \citep{ilstm_2020}, dự báo điện năng với Transformer/meta-learning \citep{transformer_energy_2023}, và giám sát máy công cụ với GMM \citep{gmm_monitoring}.

\section{Tổng hợp \& khoảng trống}
Văn liệu hội tụ vào hai trục: (i) \emph{phát hiện drift tin cậy} trong điều kiện thiếu nhãn (từ kiểm định cửa sổ đến phân biệt hai-mẫu, kiểm định độc lập, shape-based, nhận dạng loại drift), và (ii) \emph{thích ứng mô hình} (retrain/ensemble/uncertainty-triggered updates) trong khung học sâu. Khoảng trống bao gồm: đánh đổi \textbf{độ trễ--báo động giả}, chuẩn hoá \textbf{đánh giá trực tuyến đa tiêu chí} (MTTD, TP/FP/FN, thời gian/bộ nhớ, chi phí nhãn), và thiết kế \textbf{semi-/unsupervised} mạnh cho tình huống drift lặp lại hay tiến hoá \citep{survey_partA_2024,deep_survey_mdpi_2023}.

\section{Các phương pháp không dựa trên cửa sổ}

Ngoài các phương pháp dựa trên cửa sổ, phát hiện concept drift bao gồm năm danh mục chính: kiểm soát quy trình thống kê, dựa trên hiệu suất, dựa trên phân phối dữ liệu, phương pháp ensemble, và phương pháp dựa trên mạng neural.

\subsection{Phương pháp kiểm soát quy trình thống kê}

\textbf{Cumulative Sum (CUSUM) / Page-Hinkley Test:}
\begin{itemize}
    \item Tích lũy độ lệch từ giá trị mục tiêu và kích hoạt khi vượt ngưỡng
    \item Ưu điểm: Phát hiện nhanh thay đổi đột ngột, không cần tham số
    \item Hạn chế: Gặp khó khăn với drift dần dần, nhạy cảm với nhiễu
\end{itemize}

\textbf{Change Finder:}
\begin{itemize}
    \item Sử dụng mô hình Auto Regression (AR) để biểu diễn hành vi chuỗi thời gian
    \item Cập nhật tham số tăng dần với việc giảm trọng số các mẫu quá khứ
    \item Ứng dụng: Phân tích thị trường tài chính, phát hiện bất thường chuỗi thời gian
\end{itemize}

\textbf{Kolmogorov-Smirnov Test-based Windowing (KSWIN):}
\begin{itemize}
    \item Kiểm định thống kê phi tham số so sánh phân phối
    \item Được bao gồm trong các gói phát hiện drift toàn diện gần đây
\end{itemize}

\subsection{Phương pháp dựa trên hiệu suất}

\textbf{Drift Detection Method (DDM) / Early Drift Detection Method (EDDM):}
\begin{itemize}
    \item DDM: Mô hình tỷ lệ lỗi như biến nhị thức, kích hoạt trên độ lệch thống kê
    \item EDDM: Tập trung vào khoảng cách giữa các lỗi liên tiếp để phát hiện drift dần dần
    \item RDDM: Phiên bản cải tiến với độ nhạy được cải thiện
\end{itemize}

\textbf{Hoeffding Drift Detection Methods (HDDM_A, HDDM_W):}
\begin{itemize}
    \item Sử dụng giới hạn Hoeffding để phát hiện thay đổi
    \item HDDM_W kết hợp trọng số cho các quan sát gần đây
\end{itemize}

\subsection{Phương pháp dựa trên phân phối dữ liệu}

\textbf{Các độ đo khoảng cách thống kê:}
\begin{itemize}
    \item Kullback-Leibler divergence
    \item Wasserstein distance  
    \item Maximum Mean Discrepancy (MMD)
    \item Jensen-Shannon divergence
\end{itemize}

\subsection{Phương pháp ensemble và neural network}

\textbf{Các phương pháp tiên tiến hiện tại:}
\begin{itemize}
    \item Deep learning representations (DriftLens)
    \item Adversarial domain adaptation
    \item Uncertainty-based detection (UDD, PUDD)
    \item Hybrid transformer-autoencoder frameworks
\end{itemize}

Các phương pháp tiên tiến này cung cấp độ nhạy vượt trội, khả năng thời gian thực và độ bền vững so với các phương pháp thống kê truyền thống.

\section{Taxonomy phương pháp dựa trên cửa sổ}

Các detector dựa trên cửa sổ vẫn là nền tảng cho giám sát drift thời gian thực. Chúng được chia thành bốn họ chính: cố định, thích ứng, có trọng số, và đa cửa sổ.

\subsection{Cửa sổ trượt kích thước cố định}
\begin{itemize}
    \item Duy trì một cửa sổ duy nhất có độ dài n
    \item So sánh tỷ lệ lỗi hoặc phân phối trong cửa sổ hiện tại với ngưỡng lịch sử
    \item Ví dụ: Sliding Window Change Detector cơ bản, FHDDM
\end{itemize}

\subsection{Cửa sổ thích ứng (kích thước biến đổi)}
\begin{itemize}
    \item Tự động mở rộng hoặc thu nhỏ cửa sổ để duy trì giới hạn tin cậy mong muốn về lỗi
    \item Khi kiểm định thống kê không phát hiện drift, cửa sổ mở rộng
    \item Khi phát hiện drift, dữ liệu cũ nhất bị loại bỏ
    \item Ví dụ: ADWIN (ADaptive WINdow)
\end{itemize}

\subsection{Cửa sổ có trọng số}
\begin{itemize}
    \item Áp dụng trọng số cao hơn cho các instance gần đây
    \item Ví dụ: MDDM với các biến thể MDDM-A, MDDM-G, MDDM-E
    \item Sử dụng bất đẳng thức McDiarmid để so sánh trung bình có trọng số
\end{itemize}

\subsection{Đa cửa sổ và đa cửa sổ trượt}
\begin{itemize}
    \item Duy trì nhiều cửa sổ chồng chéo với độ dài khác nhau
    \item Sử dụng cửa sổ "tham chiếu" vs "phát hiện" 
    \item Nắm bắt hành vi ngắn hạn và dài hạn
    \item Ví dụ: Multi-Sliding Window để nhận dạng loại drift
\end{itemize}

\section{Ghi chú biên tập}
Các mục tài liệu tham khảo dưới đây được rút ra từ các tệp PDF bạn đã cung cấp. Bạn có thể chuyển chúng sang Bib\TeX\ chính thức (với DOI/venue) trong bước biên tập cuối cùng của luận văn.
