\chapter{Thực nghiệm và đánh giá}

\section{Giới thiệu}

Chương này trình bày kết quả thực nghiệm toàn diện của hệ thống phát hiện concept drift sử dụng ShapeDD và các phương pháp so sánh. Thực nghiệm được thiết kế để đánh giá:

\begin{enumerate}
    \item Hiệu suất phát hiện drift của ShapeDD so với 9 detector khác
    \item Khả năng phân loại loại drift (sudden, gradual, incremental, recurrent, blip)
    \item Hiệu quả của chiến lược thích ứng mô hình theo loại drift
    \item Quá trình degradation và recovery của model sau khi drift
\end{enumerate}

Đánh giá được thực hiện trên dữ liệu tổng hợp với drift đột ngột được kiểm soát chặt chẽ, cho phép đo lường chính xác các metric phát hiện và thích ứng.

\section{Thiết lập thực nghiệm}

\subsection{Cấu hình dữ liệu và drift}

Thí nghiệm chính sử dụng cấu hình sau:

\textbf{Đặc tả luồng dữ liệu:}
\begin{itemize}
    \item \textbf{Kích thước stream:} 10,000 mẫu
    \item \textbf{Vị trí drift:} Sample 1,500 (15\% vào stream)
    \item \textbf{Loại drift:} Sudden drift (đột ngột)
    \item \textbf{Nguồn dữ liệu:} River library - SEA dataset variant
    \item \textbf{Random seed:} 42 (đảm bảo tái tạo được)
\end{itemize}

\textbf{Chi tiết concept drift:}
\begin{itemize}
    \item \textbf{Pre-drift concept (0-1500):} SEA variant 0
    \begin{itemize}
        \item Feature means: [5.13, 4.98, 4.92]
        \item Class 0 ratio: 0.31
    \end{itemize}
    
    \item \textbf{Post-drift concept (1500-10000):} SEA variant 3 + transformation
    \begin{itemize}
        \item Transformation: 
        \begin{equation}
        \begin{aligned}
            x_0' &= x_0 \times 1.8 + 5.0 \\
            x_1' &= x_1 \times 1.5 - 3.0 \\
            x_2' &= x_2 \times 2.0 + 8.0
        \end{aligned}
        \end{equation}
        \item Feature means: [14.11, 4.46, 18.05]
        \item Feature shift magnitude: 15.91 (Euclidean distance)
        \item Class 0 ratio: 0.44
    \end{itemize}
\end{itemize}

\subsection{Cấu hình mô hình}

\textbf{Model architecture:}
\begin{itemize}
    \item Pipeline: StandardScaler + LogisticRegression
    \item Framework: scikit-learn (batch learning)
    \item Optimizer: LBFGS với max\_iter=1000
    \item Regularization: L2 (default)
\end{itemize}

\textbf{Training strategy:}
\begin{itemize}
    \item \textbf{Initial training:} 500 mẫu đầu tiên (pre-drift)
    \item \textbf{Warmup evaluation:} 100 mẫu tiếp theo
    \item \textbf{Deployment:} Model frozen (không online learning)
    \item \textbf{Adaptation delay:} 50 samples sau khi phát hiện drift
    \item \textbf{Adaptation window:} 800 samples (post-drift data)
\end{itemize}

\subsection{Các phương pháp phát hiện drift được đánh giá}

Thí nghiệm so sánh hiệu suất của 10 drift detector, chia thành hai nhóm:

\textbf{Window-based detectors (4 phương pháp):}
\begin{itemize}
    \item \textbf{D3:} Distance-based drift detector với ngưỡng 0.5
    \item \textbf{DAWIDD:} Discriminative Adaptive Window với $\alpha = 0.05$
    \item \textbf{ShapeDD:} Shape-based detector (standard version)
    \begin{itemize}
        \item L1 = 50, L2 = 150, n\_perm = 2500, $\alpha = 0.05$
    \end{itemize}
    \item \textbf{ShapeDD\_Improved:} Adaptive version
    \begin{itemize}
        \item L1 = 0.05 × stream\_length (adaptive)
        \item L2 = 2 × L1, n\_perm = 2500, sensitivity = 'none'
    \end{itemize}
\end{itemize}

\textbf{Streaming detectors (6 phương pháp):}
\begin{itemize}
    \item \textbf{ADWIN:} Adaptive Windowing với $\delta = 0.002$
    \item \textbf{DDM:} Drift Detection Method (cơ bản)
    \item \textbf{EDDM:} Early DDM với $\alpha = 0.95$, $\beta = 0.9$
    \item \textbf{HDDM\_A:} Hoeffding's bound (Average)
    \item \textbf{HDDM\_W:} Hoeffding's bound (Weighted)
    \item \textbf{FHDDM:} Fast HDDM với short\_window\_size = 20
\end{itemize}

\textbf{Tham số chung cho window-based:}
\begin{itemize}
    \item Chunk size: 150 samples
    \item Overlap: 100 samples (shift = 50)
    \item Cooldown period: 75 samples
\end{itemize}

\textbf{Tham số chung cho streaming:}
\begin{itemize}
    \item Detection cooldown: 50 samples
    \item Warm start window: 200 samples
    \item Accuracy window: 50 samples
\end{itemize}

\section{Kết quả thực nghiệm}

\subsection{Kết quả phát hiện drift}

Bảng~\ref{tab:detection-results} trình bày kết quả phát hiện drift của 10 detector trên sudden drift tại sample 1500.

\begin{table}[H]
\centering
\caption{Kết quả phát hiện drift (True drift position: 1500)}
\label{tab:detection-results}
\begin{tabular}{lccccccc}
\toprule
\textbf{Detector} & \textbf{Paradigm} & \textbf{Detected} & \textbf{Delay} & \textbf{TP} & \textbf{FP} & \textbf{F1} & \textbf{MTTD} \\
\midrule
D3 & window & Có & - & 0 & 1 & 0.000 & $\infty$ \\
DAWIDD & window & Có & - & 0 & 1 & 0.000 & $\infty$ \\
ShapeDD & window & Có & - & 0 & 1 & 0.000 & $\infty$ \\
\textbf{ShapeDD\_Improved} & window & \textbf{Có} & \textbf{4} & \textbf{1} & \textbf{0} & \textbf{1.000} & \textbf{4.0} \\
\midrule
ADWIN & streaming & Có & 59 & 1 & 0 & 1.000 & 59.0 \\
DDM & streaming & Có & - & 0 & 1 & 0.000 & $\infty$ \\
EDDM & streaming & Có & 51 & 1 & 0 & 1.000 & 51.0 \\
HDDM\_A & streaming & Không & - & 0 & 0 & 0.000 & $\infty$ \\
HDDM\_W & streaming & Có & - & 0 & 1 & 0.000 & $\infty$ \\
FHDDM & streaming & Có & - & 0 & 1 & 0.000 & $\infty$ \\
\bottomrule
\end{tabular}
\end{table}

\textbf{Phân tích kết quả phát hiện:}

\begin{itemize}
    \item \textbf{ShapeDD\_Improved} đạt hiệu suất tốt nhất:
    \begin{itemize}
        \item F1-Score: 1.0 (perfect detection)
        \item Detection delay: 4 samples (nhanh nhất)
        \item Precision = Recall = 1.0 (không false positive, không false negative)
        \item Drift type được phân loại chính xác là "sudden"
    \end{itemize}
    
    \item \textbf{ADWIN và EDDM} cũng phát hiện thành công:
    \begin{itemize}
        \item F1-Score: 1.0
        \item Detection delay: 59 và 51 samples (chậm hơn ShapeDD)
        \item Cả hai đều phân loại đúng là "sudden drift"
    \end{itemize}
    
    \item \textbf{Các detector thất bại:}
    \begin{itemize}
        \item D3, DAWIDD, ShapeDD (standard): Phát hiện sai vị trí (false detection)
        \item DDM, HDDM\_W, FHDDM: Phát hiện sai vị trí
        \item HDDM\_A: Không phát hiện được drift
    \end{itemize}
\end{itemize}

\textbf{Lý do ShapeDD\_Improved vượt trội:}
\begin{itemize}
    \item Adaptive window sizing: L1 = 0.05 × stream\_length = 500
    \item Phù hợp với magnitude của drift trong dữ liệu
    \item Permutation test với 2500 lần hoán vị đảm bảo độ tin cậy thống kê
\end{itemize}

\subsection{Hiệu suất mô hình và quá trình thích ứng}

Phân tích chi tiết hiệu suất mô hình qua các giai đoạn: baseline, degradation và recovery.

\begin{table}[H]
\centering
\caption{Hiệu suất mô hình qua các giai đoạn (chỉ các detector thành công)}
\label{tab:model-performance}
\begin{tabular}{lcccccc}
\toprule
\textbf{Detector} & \textbf{Baseline} & \textbf{Min Acc} & \textbf{Drop} & \textbf{Recovery} & \textbf{Rate (\%)} & \textbf{Time} \\
\midrule
ShapeDD\_Improved & 0.996 & 0.550 & 0.446 & 0.919 & 82.8 & 96 \\
ADWIN & 0.987 & 0.560 & 0.427 & 0.910 & 82.2 & 79 \\
EDDM & 0.989 & 0.560 & 0.429 & 0.904 & 80.3 & 83 \\
\bottomrule
\end{tabular}
\end{table}

\textbf{Giải thích các metric:}
\begin{itemize}
    \item \textbf{Baseline Acc:} Accuracy trung bình trước drift (pre-drift)
    \item \textbf{Min Acc:} Accuracy thấp nhất trong giai đoạn degradation
    \item \textbf{Drop:} Mức độ suy giảm accuracy = Baseline - Min
    \item \textbf{Recovery Acc:} Accuracy trung bình sau adaptation (300 samples)
    \item \textbf{Recovery Rate:} $\frac{\text{Recovery} - \text{Min}}{\text{Drop}} \times 100\%$
    \item \textbf{Recovery Time:} Số samples để đạt 95\% baseline accuracy
\end{itemize}

\textbf{Quan sát chính:}

\begin{enumerate}
    \item \textbf{Degradation phase (1500-1550):}
    \begin{itemize}
        \item Accuracy giảm mạnh từ ~0.99 xuống ~0.55
        \item Model frozen không thể thích ứng với concept mới
        \item Mức độ suy giảm: ~44-45\%
    \end{itemize}
    
    \item \textbf{Adaptation phase (1550-1650):}
    \begin{itemize}
        \item Delay 50 samples sau detection để quan sát
        \item Model được retrain toàn bộ (sudden drift strategy)
        \item Window size: 800 samples post-drift data
    \end{itemize}
    
    \item \textbf{Recovery phase (1650-1950):}
    \begin{itemize}
        \item Accuracy phục hồi lên ~0.91-0.92
        \item Recovery rate: 80-83\% (rất tốt)
        \item Recovery time: 79-96 samples
        \item Không đạt 100\% do transformation phức tạp của drift
    \end{itemize}
\end{enumerate}

\subsection{Phân tích chi phí tính toán}

Đánh giá hiệu quả tính toán của các detector thành công trong việc xử lý 10,000 samples.

\begin{table}[H]
\centering
\caption{Chi phí tính toán và bộ nhớ}
\label{tab:computational-cost}
\begin{tabular}{lcccc}
\toprule
\textbf{Detector} & \textbf{Runtime (s)} & \textbf{ms/sample} & \textbf{Memory (MB)} & \textbf{Throughput} \\
\midrule
ShapeDD\_Improved & 16.15 & 1.615 & 18.76 & 619 samples/s \\
ADWIN & 3.12 & 0.312 & 0.00 & 3205 samples/s \\
EDDM & 2.71 & 0.271 & 0.00 & 3690 samples/s \\
\midrule
D3 & 2.98 & 0.298 & 0.00 & 3356 samples/s \\
DAWIDD & 3.71 & 0.371 & 0.00 & 2695 samples/s \\
ShapeDD (standard) & 19.95 & 1.995 & 1.66 & 501 samples/s \\
\bottomrule
\end{tabular}
\end{table}

\textbf{Phân tích:}

\begin{itemize}
    \item \textbf{Streaming detectors} (ADWIN, EDDM) nhanh hơn đáng kể:
    \begin{itemize}
        \item Runtime: 0.27-0.31 ms/sample
        \item Memory footprint: Gần như 0 (cập nhật online)
        \item Throughput: >3000 samples/second
    \end{itemize}
    
    \item \textbf{Window-based detectors} chậm hơn nhưng chính xác hơn:
    \begin{itemize}
        \item ShapeDD\_Improved: 1.615 ms/sample
        \item Memory: 18.76 MB (lưu trữ window và permutation results)
        \item Trade-off: Chậm hơn 5x nhưng F1 = 1.0
    \end{itemize}
    
    \item \textbf{Bottleneck chính của ShapeDD:}
    \begin{itemize}
        \item Permutation test: 2500 lần hoán vị
        \item MMD computation trên large windows
        \item Có thể tối ưu bằng parallel processing
    \end{itemize}
\end{itemize}

\section{Đánh giá chiến lược thích ứng}

\subsection{So sánh chiến lược sudden drift}

Hệ thống triển khai chiến lược "full model reset" cho sudden drift, được so sánh với các phương pháp baseline:

\textbf{Kết quả thực nghiệm:}
\begin{itemize}
    \item \textbf{Full model reset (hệ thống):} 
    \begin{itemize}
        \item Recovery accuracy: 0.919 (91.9\%)
        \item Recovery rate: 82.8\%
        \item Training time: <1s trên 800 samples
        \item Memory: 45.2 MB
    \end{itemize}
    
    \item \textbf{Online learning (River):}
    \begin{itemize}
        \item Recovery accuracy: 0.87-0.89 (chậm hơn)
        \item Cần nhiều samples hơn để hội tụ
        \item Memory footprint nhỏ hơn (~8 MB)
    \end{itemize}
\end{itemize}

\textbf{Lý do full reset hiệu quả cho sudden drift:}
\begin{enumerate}
    \item Concept cũ hoàn toàn không còn giá trị
    \item Không có lợi ích từ việc giữ lại knowledge cũ
    \item Batch training nhanh hội tụ với dữ liệu đồng nhất
    \item Sklearn LogisticRegression tối ưu tốt cho binary classification
\end{enumerate}

\section{Thảo luận}

\subsection{Kết quả chính}

Thực nghiệm đã chứng minh các kết quả quan trọng:

\textbf{1. Hiệu quả của ShapeDD\_Improved:}
\begin{itemize}
    \item Đạt F1-score hoàn hảo (1.0) trong phát hiện sudden drift
    \item Detection delay chỉ 4 samples - nhanh nhất trong 10 detectors
    \item Adaptive window sizing tự động phù hợp với đặc tính dữ liệu
    \item Trade-off hợp lý: Chậm hơn 5x so với streaming detectors nhưng chính xác tuyệt đối
\end{itemize}

\textbf{2. Frozen model deployment:}
\begin{itemize}
    \item Model frozen phơi bày rõ ràng drift qua accuracy degradation
    \item Suy giảm từ 0.99 xuống 0.55 (~44\%) chứng tỏ drift nghiêm trọng
    \item Cho phép đo lường chính xác impact của drift
    \item Phù hợp với production systems cần stability
\end{itemize}

\textbf{3. Hiệu quả của full model reset:}
\begin{itemize}
    \item Recovery rate 82-83\% - rất tốt cho sudden drift
    \item Training time < 1s - chấp nhận được cho real-time systems
    \item Model mới học được concept mới hiệu quả
    \item Phù hợp khi concept cũ hoàn toàn không còn giá trị
\end{itemize}

\subsection{So sánh với các phương pháp khác}

\textbf{ShapeDD\_Improved vs. Streaming detectors:}
\begin{itemize}
    \item \textbf{Accuracy:} ShapeDD = 1.0 vs. ADWIN/EDDM = 1.0 (ngang bằng)
    \item \textbf{Speed:} Streaming nhanh hơn 5-10x
    \item \textbf{Delay:} ShapeDD nhanh nhất (4 samples vs. 51-59)
    \item \textbf{Memory:} ShapeDD tốn nhiều hơn (18.76 MB vs. ~0 MB)
    \item \textbf{Trade-off:} ShapeDD đổi tài nguyên lấy độ chính xác + tốc độ phát hiện
\end{itemize}

\textbf{Window-based vs. Streaming paradigm:}
\begin{itemize}
    \item Window-based: Batch analysis, chính xác cao, tốn tài nguyên
    \item Streaming: Online update, nhanh, nhẹ, nhưng có thể bỏ lỡ
    \item Lựa chọn phụ thuộc vào yêu cầu hệ thống
\end{itemize}

\subsection{Hạn chế và thách thức}

\textbf{Hạn chế của thí nghiệm:}
\begin{itemize}
    \item Chỉ test trên một loại drift (sudden)
    \item Dữ liệu synthetic - chưa validate trên real-world data
    \item Single drift point - chưa test multiple drifts
    \item Không có ground truth labels trong production
\end{itemize}

\textbf{Thách thức triển khai:}
\begin{itemize}
    \item Parameter tuning: Cần điều chỉnh L1, L2, n\_perm cho từng domain
    \item Computational cost: ShapeDD tốn tài nguyên cho high-throughput streams
    \item Scalability: Chưa test trên high-dimensional data (>100 features)
    \item Interpretability: Khó giải thích tại sao phát hiện tại vị trí cụ thể
\end{itemize}

\subsection{Ý nghĩa thực tiễn}

\textbf{Áp dụng cho production systems:}
\begin{itemize}
    \item Hệ thống monitoring với Kafka + ShapeDD khả thi
    \item Automatic adaptation giảm manual intervention
    \item Real-time visualization giúp debugging và validation
    \item Model versioning cho phép rollback nếu cần
\end{itemize}

\textbf{Lựa chọn detector phù hợp:}
\begin{itemize}
    \item \textbf{Cần accuracy cao:} Chọn ShapeDD\_Improved
    \item \textbf{Cần tốc độ:} Chọn ADWIN hoặc EDDM
    \item \textbf{Limited resources:} Chọn streaming detectors
    \item \textbf{Cần interpretability:} Chọn DDM (rule-based)
\end{itemize}

\section{Tổng kết}

Kết quả thực nghiệm chứng minh hiệu quả của hệ thống phát hiện và thích ứng concept drift:

\textbf{Thành tựu chính:}
\begin{enumerate}
    \item \textbf{Phát hiện drift chính xác:} ShapeDD\_Improved đạt F1 = 1.0 với detection delay chỉ 4 samples
    
    \item \textbf{Phân loại drift tự động:} Hệ thống phân loại đúng sudden drift và chọn chiến lược thích ứng phù hợp
    
    \item \textbf{Model adaptation hiệu quả:} Full model reset cho recovery rate 82.8\% trong 96 samples
    
    \item \textbf{Kiến trúc khả thi:} Kafka streaming + ShapeDD + Adaptor hoạt động tốt trong real-time
\end{enumerate}

\textbf{Số liệu quan trọng:}
\begin{itemize}
    \item 10 drift detectors được đánh giá
    \item 3/10 detectors phát hiện thành công (ShapeDD\_Improved, ADWIN, EDDM)
    \item ShapeDD\_Improved nhanh nhất: delay 4 samples
    \item Model degradation: 44\% accuracy drop
    \item Model recovery: 82.8\% recovery rate
    \item Computational cost: 1.615 ms/sample (acceptable for real-time)
\end{itemize}

\textbf{Bài học kinh nghiệm:}
\begin{itemize}
    \item Adaptive window sizing quan trọng cho accuracy
    \item Frozen model deployment giúp đo lường drift impact
    \item Full model reset hiệu quả cho sudden drift
    \item Trade-off giữa accuracy và speed cần cân nhắc theo use case
\end{itemize}

Chương tiếp theo sẽ tổng kết các đóng góp chính của luận văn và đề xuất hướng phát triển trong tương lai. 
