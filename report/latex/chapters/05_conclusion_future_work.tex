\chapter{Kết luận và Hướng phát triển}
\label{chap:conclusion}

\section{Tổng kết các đóng góp chính}

Luận văn này đã giải quyết bài toán phát hiện và thích ứng với concept drift trong môi trường dữ liệu luồng (streaming data) thông qua việc kết hợp nền tảng lý thuyết vững chắc và triển khai hệ thống thực tế. Ba đóng góp trọng tâm của nghiên cứu bao gồm:

\subsection{Đóng góp về Phương pháp luận (Methodological)}
\begin{itemize}
	\item \textbf{Tích hợp ShapeDD\_OW\_MMD:} Tích hợp thành công phương pháp OW-MMD (Optimally-Weighted MMD)~\cite{bharti2023owmmd} vào ShapeDD detector. Kết quả cho thấy phương pháp này cải thiện thông lượng xử lý gấp \textbf{7 lần} (đạt $\approx$ 50,300 mẫu/giây) so với ShapeDD gốc, trong khi vẫn duy trì độ chính xác phát hiện ổn định (F1 $\approx$ 0.56).
	\item \textbf{Đề xuất SHAPED\_CDT:} Xây dựng phương pháp phân loại drift không giám sát \textbf{SHAPED\_CDT}, tận dụng khả năng định vị chính xác thời điểm drift của ShapeDD để trích xuất cửa sổ dữ liệu cho CDT\_MSW. Phương pháp đạt độ chính xác \textbf{80\%} trong việc phân loại nhóm drift (TCD vs PCD), tạo tiền đề cho chiến lược thích ứng thông minh.
\end{itemize}

\subsection{Đóng góp về Thực nghiệm (Empirical)}
\begin{itemize}
	\item \textbf{Benchmark toàn diện:} Thực hiện đánh giá có hệ thống 8 phương pháp phát hiện drift trên 10 tập dữ liệu tổng hợp (synthetic) đa dạng. Kết quả được kiểm chứng thống kê bằng kiểm định Friedman và Nemenyi (với 30 lần chạy độc lập), cung cấp cái nhìn khách quan về hiệu năng của các thuật toán hiện đại.
	\item \textbf{Phân tích độ trễ và chi phí:} Chứng minh hiệu quả của việc giảm số lượng hoán vị (permutation) nhờ cơ chế giảm phương sai của OW-MMD, giúp giảm độ trễ phát hiện trung bình xuống còn 18 mẫu.
\end{itemize}

\subsection{Đóng góp về Hệ thống (System)}
\begin{itemize}
	\item \textbf{Kiến trúc Real-time Streaming:} Thiết kế và triển khai kiến trúc hệ thống dựa trên \textbf{Apache Kafka}, cho phép xử lý phân tán và mở rộng. Hệ thống tách biệt các module (Detection, Classification, Adaptation) giúp dễ dàng bảo trì và nâng cấp.
	\item \textbf{Khung chiến lược thích ứng:} Hiện thực hóa 5 chiến lược thích ứng tương ứng với các loại drift (Sudden, Incremental, Gradual, Recurrent, Blip). Đặc biệt, chiến lược \textit{Frozen Model Deployment} được đề xuất để đo lường chính xác tác động của drift trong môi trường production.
\end{itemize}

\section{Những phát hiện và hiểu biết chính}

Từ quá trình nghiên cứu và thực nghiệm, luận văn rút ra các bài học quan trọng:

\begin{enumerate}
	\item \textbf{Adaptive Window là yếu tố then chốt:} ShapeDD với kích thước cửa sổ thích ứng (adaptive sizing) vượt trội hoàn toàn so với cửa sổ cố định. Khả năng tự điều chỉnh theo độ lớn drift giúp giảm báo động giả (False Positives) và tăng độ nhạy.

	\item \textbf{Trade-off giữa Tốc độ và Độ chính xác:} Các phương pháp cửa sổ trượt truyền thống chính xác nhưng chậm. Việc tích hợp OW-MMD đã giải quyết nút thắt này, đưa tốc độ xử lý của phương pháp dựa trên kernel tiệm cận với các thuật toán thống kê nhẹ (như ADWIN) mà không hy sinh độ chính xác.

	\item \textbf{Hiệu quả của Full Reset cho Sudden Drift:} Đối với các thay đổi đột ngột (Sudden Drift), việc huấn luyện lại mô hình từ đầu (Full Reset) cho hiệu quả phục hồi cao hơn (82.8\%) so với cập nhật tăng cường (incremental updates), đồng thời tránh được hiện tượng "catastrophic forgetting".

	\item \textbf{Vai trò của Phân loại Drift:} Việc xác định đúng loại drift (ví dụ: Blip vs. Sudden) là cực kỳ quan trọng để tránh phản ứng thái quá với nhiễu (noise). Chiến lược thích ứng dựa trên loại drift giúp hệ thống vận hành ổn định và tiết kiệm tài nguyên.
\end{enumerate}

\section{Hạn chế và Phạm vi nghiên cứu}

Mặc dù đạt được những kết quả khả quan, nghiên cứu vẫn tồn tại một số hạn chế cần được xem xét:

\subsection{Tập trung vào Sudden Drift}
Luận văn tập trung chuyên sâu vào phát hiện và thích ứng với \textbf{Sudden Drift}. Các loại drift khác (Gradual, Incremental) tuy đã được hỗ trợ trong kiến trúc nhưng chưa được đánh giá sâu về mặt lý thuyết (tính chất hình học của tín hiệu MMD) và thực nghiệm rộng rãi. Kết quả hiện tại đảm bảo tính đúng đắn cao nhất cho các kịch bản thay đổi đột ngột.

\subsection{Dữ liệu tổng hợp (Synthetic Data)}
Việc đánh giá chủ yếu dựa trên các tập dữ liệu tổng hợp. Mặc dù chúng cho phép kiểm soát chính xác ground truth (thời điểm drift, loại drift) để đo lường F1-score và độ trễ, nhưng có thể chưa phản ánh hết độ phức tạp, nhiễu tạp và tính bất định của dữ liệu thực tế (real-world traces) trong các hệ thống sản xuất quy mô lớn.

\section{Hướng nghiên cứu tương lai}

Dựa trên nền tảng của luận văn, các hướng phát triển tiếp theo được đề xuất theo thứ tự ưu tiên:

\subsection{Mở rộng sang các loại Drift khác (Ưu tiên hàng đầu)}
\begin{itemize}
	\item \textbf{Gradual Drift:} Nghiên cứu đặc trưng lý thuyết của tín hiệu MMD đối với drift dần dần (không có dạng tam giác nhọn). Cần điều chỉnh ngưỡng hoặc chiến lược cửa sổ để bắt được các thay đổi chậm.
	\item \textbf{Incremental Drift:} Thử nghiệm chiến lược thích ứng liên tục (continuous adaptation) sử dụng framework River, tập trung vào cân bằng giữa tốc độ cập nhật và độ ổn định mô hình.
	\item \textbf{Recurrent Drift:} Tối ưu hóa cơ chế lưu trữ (caching) mô hình cũ và thuật toán so khớp phân phối để tái sử dụng mô hình hiệu quả hơn.
\end{itemize}

\subsection{Tầm nhìn dài hạn}
\begin{itemize}
	\item \textbf{Meta-learning trực tuyến:} Phát triển các thuật toán có khả năng tự động lựa chọn chiến lược thích ứng và tham số tối ưu dựa trên đặc điểm dữ liệu quan sát được, giảm thiểu sự can thiệp thủ công.
	\item \textbf{Deep Learning Integration:} Khám phá việc sử dụng Deep Learning để học các biểu diễn (representations) bất biến với drift, giúp mô hình bền vững hơn trước các thay đổi nhỏ.
	\item \textbf{Benchmark chuẩn hóa:} Xây dựng bộ benchmark chuẩn với dữ liệu thực tế được gán nhãn drift kỹ lưỡng, đóng góp cho cộng đồng nghiên cứu concept drift.
\end{itemize}

\section{Kết luận}

Hành trình giải quyết bài toán concept drift là một quá trình liên tục. Luận văn này đã đóng góp một bước tiến cụ thể thông qua việc (1) tích hợp các tiến bộ lý thuyết mới nhất (OW-MMD) để giải quyết bài toán hiệu năng, và (2) đề xuất hướng tiếp cận phân loại drift không giám sát (SHAPED\_CDT) để định hướng thích ứng.

Kết quả nghiên cứu khẳng định rằng việc chuyển dịch từ các mô hình tĩnh sang các hệ thống thích ứng (adaptive systems) là xu hướng tất yếu. Với kiến trúc đề xuất và những hiểu biết thu được, hệ thống hoàn toàn có khả năng mở rộng để đối mặt với những thách thức phức tạp hơn của dữ liệu luồng trong tương lai, hướng tới các ứng dụng AI vận hành bền vững và tin cậy.