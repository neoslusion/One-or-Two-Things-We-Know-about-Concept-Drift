\chapter{Kết luận và Hướng phát triển}
\label{chap:conclusion}

\section{Tổng kết các đóng góp chính}

Luận văn này đã giải quyết bài toán phát hiện và thích ứng với concept drift trong môi trường dữ liệu luồng (streaming data) thông qua việc kết hợp nền tảng lý thuyết vững chắc và triển khai hệ thống thực tế. Ba đóng góp trọng tâm của nghiên cứu bao gồm:

\subsection{Đóng góp về Phương pháp luận (Methodological)}
\begin{itemize}
	\item \textbf{Tích hợp ShapeDD-IDW:} Tích hợp phương pháp IDW-MMD (Inverse Density-Weighted MMD) --- một heuristic do luận văn đề xuất, lấy động lực từ weighted MMD framework của~\cite{bharti2023owmmd} --- với asymptotic p-value vào ShapeDD detector. Kết quả cho thấy phương pháp này cải thiện thông lượng xử lý gấp \textbf{17--20 lần} (đạt $\approx$ 111,000--131,000 mẫu/giây) so với ShapeDD gốc ($\approx$ 8,000 mẫu/giây), trong khi đạt F1-score = 0.481 với \textbf{0 False Positives} trên detection benchmark. \textit{Lưu ý:} Phương pháp có giới hạn là không phát hiện được virtual drift (chỉ thay đổi $P(Y|X)$ mà không thay đổi $P(X)$), phản ánh đúng bản chất unsupervised của thuật toán.
	\item \textbf{Đề xuất SE-CDT:} Xây dựng phương pháp phân loại drift không giám sát \textbf{SE-CDT}, tận dụng khả năng định vị chính xác thời điểm drift của ShapeDD để phân loại loại drift. Phương pháp đạt độ chính xác \textbf{72.2\%} trong việc phân loại nhóm drift (TCD vs PCD) và \textbf{50.0\%} cho từng loại drift chi tiết, tạo tiền đề cho chiến lược thích ứng thông minh mà không cần nhãn dữ liệu.
\end{itemize}

\subsection{Đóng góp về Thực nghiệm (Empirical)}
\begin{itemize}
	\item \textbf{Benchmark toàn diện:} Thực hiện đánh giá có hệ thống 8 phương pháp phát hiện drift (MMD, IDW-MMD, KS, ShapeDD, ShapeDD-IDW, SE-CDT, D3, DAWIDD) trên 11 tập dữ liệu tổng hợp đa dạng. Kết quả được kiểm chứng thống kê bằng kiểm định Friedman và Nemenyi (với 30 lần chạy độc lập, $\alpha = 0.05$), cung cấp cái nhìn khách quan về hiệu năng của các thuật toán hiện đại.
	\item \textbf{Phân tích độ trễ và chi phí:} Chứng minh hiệu quả của việc loại bỏ permutation test và thay bằng asymptotic p-value nhờ IDW-MMD, giúp giảm độ trễ phát hiện trung bình xuống còn 18 mẫu và tăng throughput 17--20 lần.
	\item \textbf{Xác nhận hành vi đúng:} Benchmark xác nhận SE-CDT/ShapeDD-IDW không phát hiện ``drift'' trên các dataset virtual drift (hyperplane, led\_abrupt, sea) --- hành vi mong đợi vì $P(X)$ không đổi.
\end{itemize}

\subsection{Đóng góp về Hệ thống (System)}
\begin{itemize}
	\item \textbf{Kiến trúc Real-time Streaming:} Thiết kế và triển khai kiến trúc hệ thống dựa trên \textbf{Apache Kafka}, cho phép xử lý phân tán và mở rộng. Hệ thống tách biệt các module (Detection, Classification, Adaptation) giúp dễ dàng bảo trì và nâng cấp.
	\item \textbf{Khung chiến lược thích ứng:} Hiện thực hóa 5 chiến lược thích ứng tương ứng với các loại drift (Sudden, Incremental, Gradual, Recurrent, Blip). Đặc biệt, chiến lược \textit{triển khai mô hình đóng băng (Frozen Model Deployment)} được đề xuất để đo lường chính xác tác động của drift trong môi trường production.
\end{itemize}

\section{Những phát hiện và hiểu biết chính}

Từ quá trình nghiên cứu và thực nghiệm, luận văn rút ra các bài học quan trọng:

\begin{enumerate}
	\item \textbf{Adaptive Window là yếu tố then chốt:} ShapeDD với kích thước cửa sổ thích ứng (adaptive sizing) cho thấy cải thiện đáng kể so với cửa sổ cố định trong điều kiện thử nghiệm. Khả năng tự điều chỉnh theo độ lớn drift giúp giảm báo động giả (False Positives) và tăng độ nhạy.

	\item \textbf{Trade-off giữa Tốc độ và Độ chính xác:} Các phương pháp cửa sổ trượt truyền thống chính xác nhưng chậm. Việc tích hợp IDW-MMD với asymptotic p-value đã giải quyết nút thắt này, đưa throughput từ $\sim$8,000 lên $\sim$131,000 mẫu/giây --- đạt tốc độ cao hơn KS-Test trong khi duy trì độ chính xác (F1 = 0.481 vs 0.289).

	\item \textbf{Hiệu quả của Full Reset cho Sudden Drift:} Đối với các thay đổi đột ngột (Sudden Drift), việc huấn luyện lại mô hình từ đầu (Full Reset) cho hiệu quả phục hồi cao hơn (85.46\%) so với không thích ứng (81.26\%), cải thiện +5.2\%, đồng thời tránh được hiện tượng ``catastrophic forgetting''.

	\item \textbf{Vai trò của Phân loại Drift:} Việc xác định đúng loại drift (ví dụ: Blip vs. Sudden) là cực kỳ quan trọng để tránh phản ứng thái quá với nhiễu (noise). Chiến lược thích ứng dựa trên loại drift giúp hệ thống vận hành ổn định và tiết kiệm tài nguyên.
\end{enumerate}

\section{Hạn chế và Phạm vi nghiên cứu}

Mặc dù đạt được những kết quả khả quan, nghiên cứu vẫn tồn tại một số hạn chế cần được xem xét:

\subsection{Trade-off giữa Detection và Classification}
SE-CDT (Std) đạt EDR = 94.4\% nhưng có số lượng False Positives cao (FP = 1394). SE-CDT (ADW) giảm FP xuống 171 nhưng EDR giảm còn 50.6\%. Trong môi trường production cần cân nhắc trade-off này:
\begin{itemize}
	\item Nếu ưu tiên \textbf{recall cao}: Sử dụng SE-CDT (Std) với threshold thấp
	\item Nếu ưu tiên \textbf{FP thấp}: Sử dụng SE-CDT (ADW) hoặc ShapeDD-IDW (FP = 0 trong detection benchmark)
\end{itemize}

\subsection{Subcategory Accuracy còn hạn chế}
Mặc dù Category Accuracy (TCD vs PCD) đạt 72.2\%, Subcategory Accuracy chỉ đạt 50.0\%. Cụ thể:
\begin{itemize}
	\item Gradual và Incremental dễ nhầm lẫn do tín hiệu MMD tương tự
	\item Recurrent bị phân loại thành chuỗi Sudden do xử lý từng peak độc lập
	\item Blip detection mới được thêm gần đây, cần validation thêm
\end{itemize}

\subsection{Dữ liệu tổng hợp (Synthetic Data)}
Việc đánh giá chủ yếu dựa trên các tập dữ liệu tổng hợp. Mặc dù chúng cho phép kiểm soát chính xác ground truth (thời điểm drift, loại drift) để đo lường F1-score và độ trễ, nhưng có thể chưa phản ánh hết độ phức tạp, nhiễu tạp và tính bất định của dữ liệu thực tế (real-world traces) trong các hệ thống sản xuất quy mô lớn.

\section{Hướng nghiên cứu tương lai}

Dựa trên nền tảng của luận văn, các hướng phát triển tiếp theo được đề xuất theo thứ tự ưu tiên:

\subsection{Mở rộng sang các loại Drift khác (Ưu tiên hàng đầu)}
\begin{itemize}
	\item \textbf{Gradual Drift:} Nghiên cứu đặc trưng lý thuyết của tín hiệu MMD đối với drift dần dần (không có dạng tam giác nhọn). Cần điều chỉnh ngưỡng hoặc chiến lược cửa sổ để bắt được các thay đổi chậm.
	\item \textbf{Incremental Drift:} Thử nghiệm chiến lược thích ứng liên tục (continuous adaptation) sử dụng framework River, tập trung vào cân bằng giữa tốc độ cập nhật và độ ổn định mô hình.
	\item \textbf{Recurrent Drift:} Tối ưu hóa cơ chế lưu trữ (caching) mô hình cũ và thuật toán so khớp phân phối để tái sử dụng mô hình hiệu quả hơn.
\end{itemize}

\subsection{Tầm nhìn dài hạn}
\begin{itemize}
	\item \textbf{Meta-learning trực tuyến:} Phát triển các thuật toán có khả năng tự động lựa chọn chiến lược thích ứng và tham số tối ưu dựa trên đặc điểm dữ liệu quan sát được, giảm thiểu sự can thiệp thủ công.
	\item \textbf{Deep Learning Integration:} Khám phá việc sử dụng Deep Learning để học các biểu diễn (representations) bất biến với drift, giúp mô hình bền vững hơn trước các thay đổi nhỏ.
	\item \textbf{Benchmark chuẩn hóa:} Xây dựng bộ benchmark chuẩn với dữ liệu thực tế được gán nhãn drift kỹ lưỡng, đóng góp cho cộng đồng nghiên cứu concept drift.
\end{itemize}

\section{Kết luận}

Hành trình giải quyết bài toán concept drift là một quá trình liên tục. Luận văn này đã đưa ra các đề xuất cụ thể thông qua:
\begin{enumerate}
	\item \textbf{Cải thiện hiệu năng:} Tích hợp IDW-MMD với asymptotic p-value giúp tăng throughput \textbf{17--20 lần} (từ 8,000 lên 131,000 mẫu/giây), cho thấy tiềm năng áp dụng trong môi trường production.
	\item \textbf{Đảm bảo độ tin cậy:} ShapeDD-IDW và SE-CDT đạt \textbf{0 False Positives} trong detection benchmark (trên các dataset có thay đổi $P(X)$), quan trọng cho môi trường production.
	\item \textbf{Phân loại không giám sát:} SE-CDT đạt \textbf{CAT = 72.2\%} và \textbf{SUB = 50.0\%} mà không cần labels, chứng minh khả năng phân tích hình học của tín hiệu MMD là khả thi.
\end{enumerate}

Kết quả nghiên cứu cho thấy rằng việc chuyển dịch từ các mô hình tĩnh sang các hệ thống thích ứng (adaptive systems) là xu hướng cần thiết. Với kiến trúc đề xuất và những hiểu biết thu được, hệ thống có khả năng mở rộng để đối mặt với những thách thức phức tạp hơn của dữ liệu luồng trong tương lai, hướng tới các ứng dụng AI vận hành bền vững và tin cậy.
