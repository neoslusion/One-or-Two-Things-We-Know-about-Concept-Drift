\chapter{Giới thiệu đề tài}

\section{Thực trạng chung}
Hệ thống phát hiện gian lận của một ngân hàng có thể đạt độ chính xác 99\% 
trong năm đầu vận hành, nhưng sau 18 tháng, tỷ lệ cảnh báo sai tăng đột biến 
lên gấp ba lần. Một mô hình dự báo bảo trì máy móc trong nhà máy thép hoạt 
động ổn định suốt mùa khô, nhưng khi mùa mưa đến, hiệu suất giảm mạnh do 
điều kiện vận hành thay đổi. Đây không phải là lỗi của thuật toán hay của 
kỹ sư xây dựng mô hình --- mà là hiện tượng \textit{trôi dạt khái niệm} 
(concept drift): khi phân phối dữ liệu thực tế thay đổi theo thời gian trong 
khi mô hình vẫn ``đứng yên''~\cite{jourdan2023process}.

Về mặt toán học, concept drift xảy ra khi $P_{train}(X, Y) \neq P_{online,t}(X, Y)$, 
trong đó $P_{train}$ và $P_{online,t}$ lần lượt là phân phối đồng thời của dữ liệu 
đầu vào và nhãn trong giai đoạn huấn luyện và vận hành tại thời điểm $t$~\cite{jourdan2023process}. 
Concept drift có thể xuất hiện dưới nhiều dạng: \textit{sudden drift}, 
\textit{gradual drift}, \textit{incremental drift}, và \textit{recurrent drift}~\cite{yuan2022advances}.

Các mô hình học máy được huấn luyện với giả định rằng dữ liệu trong tương lai 
sẽ tuân theo cùng phân phối với dữ liệu huấn luyện (giả định i.i.d.). Tuy nhiên, 
môi trường sản xuất công nghiệp hiếm khi tĩnh: thiết bị lão hóa, quy trình thay đổi, 
cảm biến xuống cấp, và hành vi người dùng biến đổi~\cite{jourdan2023process}. Kết quả là 
hiệu suất mô hình suy giảm dần theo thời gian --- đôi khi đột ngột --- gây ảnh hưởng 
đến độ tin cậy hệ thống và chi phí vận hành. Để giải quyết vấn đề này, cần có cơ chế 
(1) phát hiện drift càng sớm càng tốt, và (2) thích ứng mô hình một cách hiệu quả 
theo các điều kiện mới.

\section{Tổng quan về bài toán}
Việc xử lý và phân tích dữ liệu đang trở thành một thách thức lớn do tính phức tạp và tính động của dữ liệu trong bối cảnh hiện nay. 
Một thách thức quan trọng là xử lý hiện tượng concept drift - một hiện tượng xảy ra khi phân phối dữ liệu đầu vào $P(X)$ hoặc mối quan hệ giữa đầu vào và đầu ra $P(y|X)$ thay đổi theo thời gian - điều này ảnh 
hưởng trực tiếp đến hiệu suất của các mô hình học máy và các hệ thống ra quyết định tự động.

Đối với các ứng dụng học máy trong môi trường sản xuất công nghiệp như IoT, 
việc phát hiện drift trong thời gian thực hoặc gần thời gian thực là yêu cầu 
bắt buộc để đảm bảo chất lượng và độ tin cậy của hệ thống~\cite{jourdan2023process}. 
Tuy nhiên, việc xây dựng hệ thống như vậy đặt ra nhiều thách thức:
\begin{enumerate}
    \item Phát hiện chính xác thời điểm xảy ra drift với độ trễ thấp 
          (\textit{short detection delay}).
    \item Phân loại loại drift (sudden, gradual, incremental, recurrent) 
          để xác định chiến lược thích ứng phù hợp.
    \item Cập nhật mô hình nhanh chóng mà không làm gián đoạn dịch vụ 
          và không yêu cầu nhãn thật liên tục (\textit{unsupervised adaptation}).
\end{enumerate}

Từ những thách thức trên, đề tài này đặt ra những mục tiêu nghiên cứu và phát triển một hệ thống tự động phát hiện và thích ứng với hiện tượng concept drift trong môi trường dữ liệu streaming real-time.
\section{Mục tiêu đề tài}

\subsection{Khảo sát các phương pháp phát hiện drift}
Khảo sát toàn diện các phương pháp phát hiện \textit{concept drift}, 
bao gồm các phương pháp truyền thống và phương pháp ShapeDD. 
Nghiên cứu sẽ so sánh hiệu năng, phân tích ưu nhược điểm, 
từ đó xác định hướng cải tiến phù hợp.

\subsection{Đề xuất cải tiến cho ShapeDD}
Dựa trên cơ sở lý thuyết và kết quả đánh giá hiệu năng trên các loại 
\textit{drift} khác nhau, nghiên cứu sẽ đề xuất và áp dụng các cải tiến 
cho ShapeDD để thích ứng tốt hơn với các môi trường dữ liệu đa dạng.

\subsection{Tích hợp với mô hình phân loại drift CDT\_MSW}
Kết hợp ShapeDD (đã cải tiến) với mô hình phân loại \textit{drift} 
CDT\_MSW~\cite{guo2022cdtmsw} để xây dựng hệ thống hoàn chỉnh. 
Hệ thống này có khả năng: (1) phát hiện thời điểm xảy ra \textit{drift}, 
và (2) phân loại loại \textit{drift} để đưa ra chiến lược cập nhật mô hình phù hợp.

\subsection{Tối ưu hóa tham số}
Tiến hành thử nghiệm và tùy chỉnh các tham số của phương pháp 
để đảm bảo hiệu quả với các kiểu dữ liệu và loại \textit{drift} khác nhau.

\subsection{Triển khai và đánh giá}
Triển khai hệ thống trên môi trường dữ liệu \textit{streaming real-time} 
sử dụng Apache Kafka. Đánh giá hiệu năng trong các kịch bản thực tế 
để kiểm chứng tính khả thi của giải pháp đề xuất.

\section{Phạm vi đề tài và đối tượng nghiên cứu}
\subsection{Phạm vi đề tài}
Trong môi trường sản xuất công nghiệp, nhãn thật 
($y$) thường không có sẵn trong thời gian thực do chi phí và thời gian kiểm 
định chất lượng cao~\cite{jourdan2023process}. Do đó, nghiên cứu này tập trung vào 
phát hiện concept drift theo hướng \textbf{không giám sát} (unsupervised), 
cụ thể là phát hiện sự thay đổi trong phân phối dữ liệu đầu vào $P(X)$ --- 
còn gọi là \textit{virtual drift} hoặc \textit{data drift}. Việc phát hiện 
sớm sự thay đổi trong $P(X)$ đóng vai trò như tín hiệu cảnh báo, cho phép 
hệ thống kích hoạt các cơ chế kiểm tra hoặc cập nhật mô hình khi cần thiết.
Cụ thể, luận văn sẽ tập trung vào việc phát triển hệ thống kết hợp các phương pháp sau:
\begin{itemize}
    \item Nghiên cứu và cải thiện phương pháp ShapeDD để phát hiện concept drift không giám sát, so sánh với các phương pháp phát hiện drift truyền thống.
    \item Xây dựng các chiến lược thích ứng mô hình đa dạng tương ứng với từng loại drift được phát hiện.
    \item Thiết kế và triển khai kiến trúc streaming real-time sử dụng Apache Kafka cho việc xử lý event-driven có khả năng mở rộng.
    \item Tùy chỉnh và tối ưu hóa các tham số của phương pháp để đảm bảo hiệu quả với các kiểu dữ liệu và hiện tượng trôi dạt khác nhau.
    \item Đánh giá hiệu năng của hệ thống trên các tập dữ liệu thực tế và các kịch bản drift khác nhau.
\end{itemize}

\textbf{Trọng tâm đánh giá:} Mặc dù hệ thống được thiết kế để hỗ trợ 
nhiều loại \textit{drift} (sudden, gradual, incremental, recurrent, blip), 
luận văn này \textbf{tập trung đánh giá chuyên sâu vào sudden drift} 
(\textit{drift} đột ngột). Lý do bao gồm:

\begin{enumerate}
    \item \textbf{Tầm quan trọng thực tế:} Theo nghiên cứu về hệ thống 
          production~\cite{hinder2024survey_partA}, \textit{sudden drift} 
          chiếm đa số các trường hợp nghiêm trọng trong môi trường công nghiệp 
          (system failures, policy changes, equipment malfunctions).
          
    \item \textbf{Nền tảng lý thuyết:} ShapeDD dựa trên Triangle Shape 
          Property~\cite{hinder2021shapedd}, một kết quả toán học 
          \textbf{chỉ đúng cho sudden drift}. Các loại \textit{gradual} 
          và \textit{incremental drift} không tạo ra hình dạng tam giác 
          đặc trưng trong tín hiệu drift magnitude.
          
    \item \textbf{Đảm bảo chất lượng nghiên cứu:} Thay vì đánh giá nông 
          trên nhiều loại \textit{drift}, luận văn ưu tiên phân tích sâu 
          với statistical rigor (30 runs, Friedman test, confidence intervals).
\end{enumerate}

% Các loại drift khác đã được implement trong framework (xem Chương~\ref{chap:proposed-model}, Module \texttt{adaptation\_strategies.py}) và có thể được đánh giá trong nghiên cứu tương lai với minimal code changes.

\subsection{Đối tượng nghiên cứu}
Đối tượng nghiên cứu của luận văn bao gồm:
\begin{itemize}
    \item \textbf{Phương pháp phát hiện drift:} ShapeDD và các phương pháp 
          truyền thống (ADWIN, DDM, EDDM, Page-Hinkley).
    \item \textbf{Mô hình phân loại drift:} CDT\_MSW --- phương pháp xác định 
          loại \textit{drift} dựa trên multi-sliding windows.
    \item \textbf{Dữ liệu streaming:} Các tập dữ liệu benchmark 
          (SEA, Hyperplane, Electricity,...) và dữ liệu mô phỏng với 
          các loại \textit{drift} khác nhau.
    \item \textbf{Nền tảng xử lý streaming:} Apache Kafka cho việc triển khai 
          hệ thống \textit{real-time}.
\end{itemize}
