\chapter{Giới thiệu đề tài}

\section{Thực trạng chung}
Hệ thống phát hiện gian lận của một ngân hàng có thể đạt độ chính xác 99\% trong năm đầu vận hành, 
nhưng sau 18 tháng, tỷ lệ cảnh báo sai tăng đột biến lên gấp ba lần. 
Một mô hình dự báo bảo trì máy móc trong nhà máy thép hoạt động ổn định suốt mùa khô, 
nhưng khi mùa mưa đến, hiệu suất giảm mạnh do điều kiện vận hành thay đổi. 
Đây không phải là lỗi của thuật toán hay của kỹ sư xây dựng mô hình --- mà là hiện tượng 
\textbf{trôi dạt khái niệm} (concept drift): khi thực tế thay đổi nhưng mô hình vẫn ``đứng yên''.

Các mô hình học máy được huấn luyện với giả định rằng dữ liệu trong tương lai sẽ tuân theo cùng phân phối 
với dữ liệu huấn luyện. Tuy nhiên, môi trường sản xuất công nghiệp hiếm khi tĩnh: 
thiết bị lão hóa, quy trình thay đổi, hành vi người dùng biến đổi. 
Kết quả là hiệu suất mô hình suy giảm dần theo thời gian, đôi khi đột ngột, 
gây ảnh hưởng đến độ tin cậy hệ thống và chi phí vận hành. 
Để giải quyết vấn đề này, cần có cơ chế (1) phát hiện drift càng sớm càng tốt, 
và (2) thích ứng mô hình một cách hiệu quả theo các điều kiện mới.

\section{Tổng quan về bài toán}
Việc xử lý và phân tích dữ liệu đang trở thành một thách thức lớn do tính phức tạp và tính động của dữ liệu trong bối cảnh hiện nay. 
Một thách thức quan trọng là xử lý hiện tượng concept drift - một hiện tượng xảy ra khi phân phối dữ liệu đầu vào $P(X)$ hoặc mối quan hệ giữa đầu vào và đầu ra $P(y|X)$ thay đổi theo thời gian - điều này ảnh 
hưởng trực tiếp đến hiệu suất của các mô hình học máy và các hệ thống ra quyết định tự động.

Đối với các ứng dụng học máy trong môi trường sản xuất công nghiệp như IoT, việc phát hiện drift trong thời gian thực (real-time) 
hoặc gần thời gian thực là yêu cầu bắt buộc để đảm bảo chất lượng và độ tin cậy của hệ thống. Tuy nhiên, việc xây dựng hệ thống 
như vậy đặt ra nhiều thách thức: (1) phát hiện chính xác thời điểm xảy ra trôi dạt, (2) từ thông tin thu thập được có thể đưa ra được chiến lược phù hợp đối với mô hình học máy, (3) thay đổi hoặc cập nhật mô hình nhanh chóng mà không làm gián đoạn dịch vụ.

Từ những thách thức trên, đề tài này đặt ra những mục tiêu nghiên cứu và phát triển một hệ thống tự động phát hiện và thích ứng với hiện tượng concept drift trong môi trường dữ liệu streaming real-time.

\section{Mục tiêu đề tài}
\subsection{Nghiên cứu ShapeDD và các phương pháp phát hiện drift truyền thống}
Khảo sát toàn diện các phương pháp phát hiện concept drift cũng như phương pháp ShapeDD, so sánh hiệu năng và cách thức hoạt động của chúng. Thông qua việc phân tích này, nghiên cứu sẽ làm rõ mục tiêu và cách thức hoạt động của các phương pháp, từ đó đề xuất các cải tiến cho ShapeDD.
\subsection{Ứng dụng những đề xuất cải tiến}
Sau khi nắm bắt được cơ sở lý thuyết và cách hoạt động của ShapeDD, nghiên cứu sẽ tiến hành đánh giá hiệu năng trên các trường hợp trôi dạt khác nhau. Dựa trên những gì quan sát được, nghiên cứu sẽ đề xuất và áp dụng những cải thiện cho phương pháp hiện có để có thể thích nghi tốt hơn với các môi trường khác nhau.
\subsection{Kết hợp với mô hình phân loại hiện tượng trôi dạt CDT\_MSW}
Từ những gì đã nghiên cứu và cải thiện, nghiên cứu sẽ kết hợp với mô hình phân loại hiện tượng trôi dạt CDT\_MSW để xây dựng một hệ thống phát hiện và thích ứng trôi dạt hoàn chỉnh. Hệ thống này vừa xác định và phân loại drift để có thể đưa ra quyết định chính xác cho việc cập nhật mô hình.
\subsection{Tùy chỉnh và tối ưu hóa phương pháp}
Nghiên cứu sẽ tiến hành các thử nghiệm và tùy chỉnh các tham số của các phương pháp để đảm bảo rằng mô hình phù hợp và hiệu quả với bài toán phát hiện trôi dạt với các kiểu dữ liệu cũng như hiện tượng trôi dạt khác nhau.
\subsection{Triển khai và đánh giá}
Cuối cùng, nghiên cứu sẽ triển khai hệ thống phát hiện và thích ứng trôi dạt trên môi trường dữ liệu streaming real-time sử dụng Apache Kafka. Hiệu năng của hệ thống sẽ được đánh giá trong các kịch bản thực tế để đảm bảo tính khả thi và hiệu quả của giải pháp đề xuất.
\section{Giới hạn đề tài và đối tượng nghiên cứu}
\subsection{Phạm vi đề tài}
Phạm vi của đề tài này tập trung vào việc nghiên cứu và phát triển hệ thống phát hiện và thích ứng concept drift trong môi trường động. 
Cụ thể, luận văn sẽ tập trung vào việc phát triển hệ thống kết hợp các phương pháp sau:
\begin{itemize}
    \item Nghiên cứu và cải thiện phương pháp ShapeDD để phát hiện concept drift không giám sát, so sánh với các phương pháp phát hiện drift truyền thống.
    \item Xây dựng các chiến lược thích ứng mô hình đa dạng tương ứng với từng loại drift được phát hiện.
    \item Thiết kế và triển khai kiến trúc streaming real-time sử dụng Apache Kafka cho việc xử lý event-driven có khả năng mở rộng.
    \item Tùy chỉnh và tối ưu hóa các tham số của phương pháp để đảm bảo hiệu quả với các kiểu dữ liệu và hiện tượng trôi dạt khác nhau.
    \item Đánh giá hiệu năng của hệ thống trên các tập dữ liệu thực tế và các kịch bản drift khác nhau.
\end{itemize}

\textbf{Trọng tâm đánh giá:} Mặc dù hệ thống được thiết kế để hỗ trợ nhiều loại drift (sudden, gradual, incremental, recurrent, blip), luận văn này \textbf{tập trung đánh giá chuyên sâu vào sudden drift (trôi dạt đột ngột)}. Lý do cho sự lựa chọn này bao gồm: (1) sudden drift có tầm quan trọng thực tế cao nhất trong các hệ thống production, (2) ShapeDD có nền tảng lý thuyết vững chắc nhất cho hiện tượng trôi dạt đột ngột, và (3) để đảm bảo chất lượng nghiên cứu với phân tích sâu thay vì coverage nông. Các loại drift khác đã được implement trong framework và có thể được đánh giá trong nghiên cứu tương lai.

\subsection{Đối tượng nghiên cứu}
Đối tượng nghiên cứu của đề tài này là các mô hình và phương pháp phát hiện, phân loại và thích ứng concept drift, bao gồm:
\begin{itemize}
    \item Phương pháp ShapeDD: luận văn sẽ nghiên cứu chi tiết cơ chế hoạt động của ShapeDD - một phương pháp phát hiện concept drift không giám sát, so sánh với các phương pháp truyền thống để hiểu rõ ưu nhược điểm và đề xuất các cải thiện phù hợp.
    \item Mô hình CDT\_MSW: luận văn sẽ nghiên cứu và tích hợp mô hình phân loại loại drift tự động CDT\_MSW để xác định chính xác loại drift xảy ra, từ đó hỗ trợ việc lựa chọn chiến lược thích ứng mô hình phù hợp.
    \item Các chiến lược thích ứng mô hình: luận văn sẽ nghiên cứu và phát triển các chiến lược thích ứng đa dạng tương ứng với từng loại drift (sudden, gradual, incremental, recurring) để duy trì hiệu suất của mô hình.
    \item Kiến trúc streaming real-time: luận văn sẽ nghiên cứu và triển khai kiến trúc xử lý dữ liệu streaming sử dụng Apache Kafka để đảm bảo khả năng phát hiện và thích ứng drift trong thời gian thực hoặc gần thời gian thực.
    \item Phương pháp tùy chỉnh và tối ưu hóa: luận văn sẽ tiến hành các thử nghiệm để tùy chỉnh và tối ưu các tham số của hệ thống nhằm đảm bảo tính hiệu quả và khả thi trong các môi trường dữ liệu streaming khác nhau.
\end{itemize}
