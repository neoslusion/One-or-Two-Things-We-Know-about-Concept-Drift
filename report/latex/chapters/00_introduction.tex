\chapter{Giới thiệu đề tài}

\section{Thực trạng chung}
Hệ thống phát hiện gian lận của một ngân hàng có thể đạt độ chính xác cao
trong năm đầu vận hành, nhưng sau một thời gian dài vận hành, tỷ lệ cảnh báo sai tăng đột biến
lên gấp ba lần. Một mô hình dự báo bảo trì máy móc trong nhà máy thép hoạt
động ổn định suốt mùa khô, nhưng khi mùa mưa đến, hiệu suất giảm mạnh do
điều kiện vận hành thay đổi. Đây không phải là lỗi của thuật toán hay của
kỹ sư xây dựng mô hình --- mà là hiện tượng \textit{trôi dạt khái niệm}
(concept drift): khi phân phối dữ liệu thực tế thay đổi theo thời gian trong
khi mô hình vẫn ``đứng yên''~\cite{jourdan2023process}.

Về mặt toán học, concept drift xảy ra khi $P_{train}(X, Y) \neq P_{online,t}(X, Y)$,
trong đó $P_{train}$ và $P_{online,t}$ lần lượt là phân phối đồng thời của dữ liệu
đầu vào và nhãn trong giai đoạn huấn luyện và vận hành tại thời điểm $t$~\cite{jourdan2023process}.
Concept drift có thể xuất hiện dưới nhiều dạng: \textit{sudden drift},
\textit{gradual drift}, \textit{incremental drift}, và \textit{recurrent drift}~\cite{yuan2022advances}.

Các mô hình học máy được huấn luyện với giả định rằng dữ liệu trong tương lai
sẽ tuân theo cùng phân phối với dữ liệu huấn luyện (giả định i.i.d.). Tuy nhiên,
môi trường sản xuất công nghiệp hiếm khi tĩnh: thiết bị lão hóa, quy trình thay đổi,
cảm biến xuống cấp, và hành vi người dùng biến đổi~\cite{jourdan2023process}. Kết quả là
hiệu suất mô hình suy giảm dần theo thời gian --- đôi khi đột ngột --- làm giảm
độ tin cậy hệ thống và tăng chi phí vận hành. Để giải quyết vấn đề này, cần có cơ chế
(1) phát hiện drift càng sớm càng tốt, và (2) thích ứng mô hình một cách hiệu quả
theo các điều kiện mới.

\section{Tổng quan về bài toán}
Việc xử lý và phân tích dữ liệu đang trở thành một thách thức lớn do tính phức tạp và sự biến động của dữ liệu trong bối cảnh hiện nay.
Một trong những thách thức quan trọng là xử lý hiện tượng trôi dạt khái niệm (concept drift). Hiện tượng này xảy ra khi phân phối dữ liệu đầu vào $P(X)$ hoặc mối quan hệ $P(y|X)$ thay đổi theo thời gian, làm suy giảm hiệu suất của các mô hình học máy và hệ thống ra quyết định tự động.

Đối với các ứng dụng học máy trong môi trường sản xuất công nghiệp như IoT,
việc phát hiện drift trong thời gian thực hoặc gần thời gian thực là yêu cầu
bắt buộc để đảm bảo chất lượng và độ tin cậy của hệ thống~\cite{jourdan2023process}.
Tuy nhiên, việc hiện thực hóa một hệ thống như vậy đối mặt với ba thách thức kỹ thuật trọng yếu. Thứ nhất là yêu cầu về độ trễ thấp trong phát hiện điểm trôi dạt (\textit{short detection delay}). Thứ hai là khả năng phân loại chính xác các loại trôi dạt (sudden, gradual, incremental, recurrent) để xác định chiến lược thích ứng phù hợp. Cuối cùng là cơ chế cập nhật mô hình nhanh chóng mà không làm gián đoạn dịch vụ và không yêu cầu nhãn thật liên tục (\textit{unsupervised adaptation}).

Từ những thách thức trên, đề tài này đặt ra những mục tiêu nghiên cứu và phát triển một hệ thống tự động phát hiện và thích ứng với hiện tượng trôi dạt khái niệm trong môi trường luồng dữ liệu thời gian thực.
\section{Mục tiêu đề tài}

\subsection{Khảo sát các phương pháp phát hiện drift}
Khảo sát toàn diện các phương pháp phát hiện \textit{concept drift},
bao gồm các phương pháp truyền thống và phương pháp ShapeDD.
Nghiên cứu sẽ so sánh hiệu năng, phân tích ưu nhược điểm,
từ đó xác định hướng cải tiến phù hợp.

\subsection{Tích hợp các cải tiến cho ShapeDD}
Dựa trên cơ sở lý thuyết và kết quả đánh giá hiệu năng, nghiên cứu tích hợp kỹ thuật \textbf{IDW-MMD (Inverse Density-Weighted MMD)} vào phương pháp ShapeDD. Mục tiêu là giải quyết các hạn chế về chi phí tính toán và độ nhạy của phương pháp gốc. IDW-MMD sử dụng cơ chế trọng số nghịch biến với mật độ để giảm độ dao động kết quả, cho phép đạt độ chính xác cao với số lượng mẫu ít hơn.

\subsection{Đề xuất phương pháp phân loại drift SE-CDT}
Lấy cảm hứng từ ý tưởng phân loại loại drift của CDT-MSW (Concept Drift Type identification based on Multi-Sliding Windows~\cite{guo2022cdtmsw}, sẽ được trình bày chi tiết ở Mục~\ref{sec:cdt-msw-theory}), luận văn đề xuất phương pháp mới \textbf{SE-CDT} (ShapeDD-Enhanced Concept Drift Type Identification). Khác với CDT-MSW gốc yêu cầu nhãn (labels) để tính accuracy ratio, SE-CDT thay thế hoàn toàn cơ chế này bằng các đặc trưng hình học từ tín hiệu drift magnitude $\sigma(t)$ của ShapeDD.
Phương pháp này có khả năng: (1) phát hiện chính xác thời điểm xảy ra \textit{drift} nhờ ShapeDD, và (2) sử dụng thông tin đó để trích xuất cửa sổ dữ liệu chính xác cho việc phân loại loại \textit{drift}, từ đó đưa ra chiến lược thích ứng tối ưu.

\subsection{Tối ưu hóa tham số}
Tiến hành thử nghiệm và tùy chỉnh các tham số của phương pháp
để đảm bảo hiệu quả với các kiểu dữ liệu và loại \textit{drift} khác nhau.

\subsection{Triển khai và đánh giá}
Thiết kế kiến trúc hệ thống trên môi trường dữ liệu \textit{streaming real-time} sử dụng Apache Kafka. Thực hiện mô phỏng và đánh giá hiệu năng trong các kịch bản thử nghiệm để kiểm chứng tính khả thi của giải pháp đề xuất. (Lưu ý: Việc triển khai production thực tế và đánh giá hiệu năng chịu tải quy mô lớn là hướng phát triển tương lai).

\section{Phạm vi đề tài và đối tượng nghiên cứu}
\subsection{Phạm vi đề tài}
Trong môi trường sản xuất công nghiệp, nhãn thật
($y$) thường không có sẵn trong thời gian thực do chi phí và thời gian kiểm
định chất lượng cao~\cite{jourdan2023process}. Do đó, nghiên cứu này tập trung vào
phát hiện concept drift theo hướng \textbf{không giám sát} (unsupervised),
cụ thể là phát hiện sự thay đổi trong phân phối dữ liệu đầu vào $P(X)$ ---
còn gọi là \textit{virtual drift} hoặc \textit{data drift}. Việc phát hiện
sớm sự thay đổi trong $P(X)$ đóng vai trò như tín hiệu cảnh báo, cho phép
hệ thống kích hoạt các cơ chế kiểm tra hoặc cập nhật mô hình khi cần thiết.
Cụ thể, luận văn sẽ tập trung vào việc phát triển hệ thống kết hợp các phương pháp sau:
\begin{itemize}
	\item Nghiên cứu và cải thiện phương pháp ShapeDD để phát hiện concept drift không giám sát, so sánh với các phương pháp phát hiện drift truyền thống.
	\item Xây dựng các chiến lược thích ứng mô hình đa dạng tương ứng với từng loại drift được phát hiện.
	\item Thiết kế và triển khai kiến trúc streaming real-time sử dụng Apache Kafka cho việc xử lý event-driven có khả năng mở rộng.
	\item Tùy chỉnh và tối ưu hóa các tham số của phương pháp để đảm bảo hiệu quả với các kiểu dữ liệu và hiện tượng trôi dạt khác nhau.
	\item Đánh giá hiệu năng của hệ thống trên các tập dữ liệu thực tế và các kịch bản drift khác nhau.
\end{itemize}

\textbf{Phạm vi đánh giá:} Hệ thống được thiết kế và đánh giá trên
\textbf{đa dạng các loại drift} bao gồm sudden, gradual, incremental, recurrent, và blip nhưng vẫn tập trung vào sudden drift.
Cách tiếp cận đánh giá bao gồm:

\begin{enumerate}
	\item \textbf{Đánh giá toàn diện:} Thực nghiệm trên nhiều kịch bản drift khác nhau
	      với nhiều lần chạy độc lập để đảm bảo độ tin cậy thống kê.

	\item \textbf{Phân tích đa tầng:} Đánh giá cả khả năng \textit{phát hiện}
	      lẫn khả năng \textit{phân loại} của hệ thống, bao gồm phân loại
	      theo nhóm (TCD vs PCD) và theo loại cụ thể.

	\item \textbf{So sánh công bằng:} Đánh giá phương pháp đề xuất (SE-CDT không giám sát)
	      so với baseline CDT-MSW có giám sát trong cùng điều kiện thử nghiệm.
\end{enumerate}


\subsection{Đối tượng nghiên cứu}
Đối tượng nghiên cứu của luận văn bao gồm:
\begin{itemize}
	\item \textbf{Phương pháp phát hiện drift:} ShapeDD và các phương pháp
	      truyền thống (ADWIN, DDM, EDDM, Page-Hinkley).
	\item \textbf{Mô hình phân loại drift:} SE-CDT --- phương pháp xác định
	      loại \textit{drift} dựa trên multi-sliding windows lấy ý tưởng từ CDT-MSW.
	\item \textbf{Dữ liệu streaming:} Các tập dữ liệu benchmark
	      (SEA, Hyperplane, Electricity,...) và dữ liệu mô phỏng với
	      các loại \textit{drift} khác nhau.
	\item \textbf{Nền tảng xử lý streaming:} Đề xuất sử dụng Apache Kafka cho việc triển khai
	      hệ thống \textit{real-time}.
\end{itemize}
