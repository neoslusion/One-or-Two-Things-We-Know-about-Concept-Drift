\chapter{Giới thiệu đề tài}

\section{Thực trạng chung}

Trong những năm gần đây, lĩnh vực trí tuệ nhân tạo ngày càng phát triển nhanh chóng. Việc ứng dụng thành quả của trí tuệ nhân tạo ngày càng được phổ biến rộng rãi, không chỉ trong đời sống hằng ngày mà cả trong công việc. Các ứng dụng học máy không còn bị giới hạn trong phòng thí nghiệm mà đã được triển khai vào đời sống thực tế trong các lĩnh vực sản xuất như bảo trì thông minh và kiểm soát chất lượng. Khi đó, các câu hỏi liên quan đến độ tin cậy và độ bền liên tục của chúng nảy sinh.

Các tập dữ liệu tĩnh được sử dụng để huấn luyện các mô hình học máy chỉ có thể nắm bắt được một phần nhỏ các điều kiện có thể xảy ra trong thế giới thực. Các trường hợp trôi dạt, chẳng hạn như thay đổi điều kiện môi trường, thiết bị và vận hành có thể, theo thời gian, làm giảm đáng kể hiệu suất của các mô hình học máy, gây ảnh hưởng đến sự an toàn, độ tin cậy của mô hình và kinh tế nếu không được giải quyết đúng cách. Do đó, cần phải (1) phát hiện sự trôi dạt sớm nhất có thể và (2) điều chỉnh hiệu quả mô hình theo các điều kiện thay đổi động.

\section{Tổng quan về bài toán}

Trong lĩnh vực xử lý dữ liệu streaming, việc xử lý và phân tích dữ liệu đang trở thành một thách thức lớn do tính phức tạp và tính động của dữ liệu này. Một thách thức quan trọng là xử lý hiện tượng \textbf{concept drift} - một hiện tượng xảy ra khi phân phối dữ liệu đầu vào $P(X)$ hoặc mối quan hệ giữa đầu vào và đầu ra $P(y|X)$ thay đổi theo thời gian - điều này ảnh hưởng trực tiếp đến hiệu suất của các mô hình học máy và các hệ thống ra quyết định tự động.

Đối với các ứng dụng học máy trong môi trường production, việc phát hiện drift trong thời gian thực (real-time) hoặc gần thời gian thực là yêu cầu bắt buộc để đảm bảo chất lượng và độ tin cậy của hệ thống. Tuy nhiên, việc xây dựng hệ thống như vậy đặt ra nhiều thách thức: (1) phát hiện drift chính xác với false positive thấp, (2) xác định loại drift để chọn chiến lược thích ứng phù hợp, (3) cập nhật mô hình nhanh chóng mà không làm gián đoạn dịch vụ.

Luận văn này giải quyết các thách thức trên bằng cách xây dựng một hệ thống end-to-end kết hợp: (1) phát hiện drift không giám sát sử dụng ShapeDD với kernel-based statistical testing, (2) phân loại loại drift tự động sử dụng CDT\_MSW, (3) adaptation strategies đa dạng cho từng loại drift, và (4) kiến trúc streaming real-time sử dụng Apache Kafka cho việc xử lý event-driven có khả năng mở rộng.

\section{Mục tiêu đề tài}

Luận văn này hướng đến việc xây dựng một hệ thống phát hiện và thích ứng concept drift tự động cho môi trường dữ liệu streaming thời gian thực. Các mục tiêu cụ thể bao gồm:

\subsection{Nghiên cứu và lựa chọn phương pháp phát hiện drift phù hợp}

\textbf{Mục tiêu:} Khảo sát toàn diện các phương pháp phát hiện concept drift (window-based và streaming-based), so sánh ưu nhược điểm, và lựa chọn phương pháp phù hợp nhất cho hệ thống.

\textbf{Nhiệm vụ cụ thể:}
\begin{itemize}
    \item Nghiên cứu lịch sử phát triển của drift detection (từ FLORA 1996 đến deep learning 2024)
    \item So sánh hiệu suất của 18 methods trên 8 datasets (F1-score, MTTD, Precision-Recall)
    \item Chứng minh tại sao ShapeDD là lựa chọn tối ưu (balance Precision-Recall, label-free, có khả năng mở rộng với SNR-Adaptive)
\end{itemize}

\textbf{Output (Chapter 1):} Justification cho việc chọn ShapeDD + SNR-Adaptive strategy.

\subsection{Phát triển cơ sở lý thuyết và cải tiến phương pháp}

\textbf{Mục tiêu:} Xây dựng nền tảng toán học cho ShapeDD và đề xuất phương pháp SNR-Adaptive mới để tự động điều chỉnh độ nhạy dựa trên môi trường nhiễu.

\textbf{Nhiệm vụ cụ thể:}
\begin{itemize}
    \item Trình bày lý thuyết MMD, RKHS, và triangle pattern matching
    \item Phân tích buffer dilution effect (SNR theoretical vs observed)
    \item Đề xuất SNR-Adaptive với threshold calibration (0.010) dựa trên Neyman-Pearson criterion
    \item Chứng minh strategy balance (~50% aggressive / ~50% conservative) tối ưu cho Precision-Recall
\end{itemize}

\textbf{Output (Chapter 2):} Theoretical foundation + SNR-Adaptive algorithm (contribution chính của luận văn).

\subsection{Triển khai hệ thống phát hiện và thích ứng real-time}

\textbf{Mục tiêu:} Xây dựng hệ thống end-to-end trên Apache Kafka, tích hợp drift detection, drift type identification (CDT\_MSW), và model adaptation strategies.

\textbf{Nhiệm vụ cụ thể:}
\begin{itemize}
    \item Thiết kế kiến trúc Producer-Consumer-Adaptor với Kafka messaging
    \item Triển khai ShapeDD SNR-Adaptive trong buffer-based detection (L1=50, L2=150, buffer=750)
    \item Implement 5 adaptation strategies (sudden$\rightarrow$full reset, incremental$\rightarrow$online update, recurrent$\rightarrow$model cache, etc.)
    \item Xây dựng error handling mechanisms (retry, circuit breaker, graceful degradation) để đảm bảo fault tolerance
\end{itemize}

\textbf{Output (Chapter 3):} Production-ready system với high availability (uptime > 99.5\%).

\subsection{Đánh giá hiệu suất và so sánh với baseline methods}

\textbf{Mục tiêu:} Thực nghiệm toàn diện để chứng minh SNR-Adaptive cải thiện hiệu suất so với ShapeDD gốc và các methods khác.

\textbf{Nhiệm vụ cụ thể:}
\begin{itemize}
    \item Benchmark trên 8 synthetic datasets với 10 controlled drift points mỗi dataset
    \item So sánh SNR-Adaptive với 17 baseline methods (ADWIN, KSWIN, DDM, HDDM, etc.)
    \item Đánh giá metrics: F1-score, Precision, Recall, MTTD, False Positive Rate
    \item Phân tích strategy selection behavior (aggressive vs conservative) theo SNR environment
    \item Validate parameter choices (threshold=0.010, sensitivity='medium') qua ablation study
\end{itemize}

\textbf{Output (Chapter 4):} Experimental results chứng minh hiệu quả của SNR-Adaptive (target F1 $\geq$ 0.690, ranking top 25\%).

\section{Giới hạn đề tài và đối tượng nghiên cứu}

\subsection{Phạm vi đề tài}

Nghiên cứu này tập trung vào các phạm vi cụ thể sau:

\textbf{Phạm vi về phương pháp:}
\begin{itemize}
\item Nghiên cứu sâu về phương pháp Shape Drift Detector (ShapeDD) cùng với phiên bản cải tiến và so sánh đánh giá với các phương pháp phát hiện drift khác (ADWIN, DDM, EDDM, etc.)
\item Phát hiện concept drift trong môi trường dữ liệu streaming real-time sử dụng Apache Kafka
\item Nghiên cứu phương pháp phân loại loại drift (CDT\_MSW) để nhận diện sudden, incremental, gradual, recurrent và blip drift
\item Phát triển chiến lược thích ứng mô hình (model adaptation strategies) phù hợp với từng loại drift
\item Xây dựng hệ thống end-to-end từ phát hiện đến thích ứng tự động
\end{itemize}

\textbf{Phạm vi về dữ liệu:}
\begin{itemize}
\item Chủ yếu sử dụng tập dữ liệu synthetic có kiểm soát để đánh giá hiệu suất
\item Tập trung vào dữ liệu số với loại concept drift đột ngột
\item Không bao gồm dữ liệu phi cấu trúc như text, image hoặc audio
\end{itemize}

\textbf{Phạm vi về đánh giá:}
\begin{itemize}
\item Đánh giá hiệu suất phát hiện drift về mặt độ chính xác, thời gian phát hiện và tỷ lệ false alarm
\item So sánh với các baseline methods cơ bản
\item Không bao gồm đánh giá về computational complexity chi tiết hoặc scalability trên big data
\end{itemize}

\subsection{Đối tượng nghiên cứu}

\textbf{Đối tượng chính:} Hệ thống phát hiện và thích ứng concept drift tự động sử dụng ShapeDD detector và chiến lược thích ứng theo loại drift

\textbf{Các đối tượng cụ thể bao gồm:}
\begin{itemize}
\item Cơ sở lý thuyết Maximum Mean Discrepancy (MMD) trong RKHS và kernel-based drift detection
\item Các thành phần của thuật toán ShapeDD: kernel selection, window management, statistical validation
\item Phương pháp CDT\_MSW (Concept Drift Type Identification based on Multi-Sliding Windows) để phân loại loại drift
\item Chiến lược thích ứng mô hình: full reset (sudden), gradual updates (incremental), weighted updates (gradual), model caching (recurrent), minimal update (blip)
\item Kiến trúc streaming real-time: Apache Kafka message broker, Producer-Consumer pattern, event-driven architecture
\item Tập dữ liệu synthetic mô phỏng các loại concept drift khác nhau (SEA dataset variants)
\item Metrics đánh giá: F1-score, detection delay, recovery rate, accuracy degradation
\end{itemize}
