\chapter{Giới thiệu đề tài}

\section{Thực trạng chung}
Trong những năm gần đây, lĩnh vực trí tuệ nhân tạo ngày càng phát triển nhanh chóng. 
Việc ứng dụng thành quả của trí tuệ nhân tạo ngày càng được phổ biến rộng rãi, không chỉ trong đời sống hằng ngày mà cả trong công việc. 
Các ứng dụng học máy không còn bị giới hạn trong phòng thí nghiệm mà đã được triển khai vào đời sống thực tế trong các lĩnh vực sản 
xuất như bảo trì thông minh và kiểm soát chất lượng. Khi đó, các câu hỏi liên quan đến độ tin cậy và độ bền liên tục của chúng nảy sinh.

Các tập dữ liệu tĩnh được sử dụng để huấn luyện các mô hình học máy chỉ có thể nắm bắt được một phần nhỏ các điều kiện có 
thể xảy ra trong thế giới thực. Các trường hợp trôi dạt, chẳng hạn như thay đổi điều kiện môi trường, thiết bị và vận hành có thể, 
theo thời gian, làm giảm đáng kể hiệu suất của các mô hình học máy, gây ảnh hưởng đến sự an toàn, độ tin cậy của 
mô hình và kinh tế nếu không được giải quyết đúng cách. Do đó, cần phải (1) phát hiện sự trôi dạt sớm nhất có thể và (2) điều chỉnh 
hiệu quả mô hình theo các điều kiện thay đổi động.

\section{Tổng quan về bài toán}
Đối với việc xử lý dữ liệu streaming, việc xử lý và phân tích dữ liệu đang trở thành một thách thức lớn do tính phức tạp 
và tính động của dữ liệu này. Một thách thức quan trọng là xử lý hiện tượng \textbf{concept drift} - một hiện tượng xảy ra 
khi phân phối dữ liệu đầu vào $P(X)$ hoặc mối quan hệ giữa đầu vào và đầu ra $P(y|X)$ thay đổi theo thời gian - điều này ảnh 
hưởng trực tiếp đến hiệu suất của các mô hình học máy và các hệ thống ra quyết định tự động.

Đối với các ứng dụng học máy trong môi trường sản xuất công nghiệp như IoT, việc phát hiện drift trong thời gian thực (real-time) 
hoặc gần thời gian thực là yêu cầu bắt buộc để đảm bảo chất lượng và độ tin cậy của hệ thống. Tuy nhiên, việc xây dựng hệ thống 
như vậy đặt ra nhiều thách thức: (1) phát hiện drift chính xác, (2) từ sự kiện drift đó xác định được loại drift để chọn chiến lược 
thích ứng phù hợp, (3) thay đổi hoặc cập nhật mô hình nhanh chóng mà không làm gián đoạn dịch vụ.

Trong đề tài này, luận văn cố gắng giải quyết các thách thức trên bằng cách xây dựng một hệ thống end-to-end kết hợp: (1) phát hiện drift 
không giám sát sử dụng phương pháp ShapeDD - một phương pháp phát hiện trôi dạt, (2) phân loại loại drift tự động sử dụng CDT\_MSW, (3) 
adaptation strategies đa dạng cho từng loại drift, và (4) kiến trúc streaming real-time sử dụng Apache Kafka cho việc xử lý event-driven 
có khả năng mở rộng. Điều này đòi hỏi sự kết hợp giữa việc nhanh chóng phát hiện và phân loại drift, cùng với việc áp dụng các chiến lược
thích ứng mô hình phù hợp để duy trì hiệu suất của hệ thống trong môi trường dữ liệu streaming động.

\section{Mục tiêu đề tài}
\subsection{Nghiên cứu shapeDD và các phương pháp phát hiện drift truyền thống}
Khảo sát toàn diện các phương pháp phát hiện concept drift cũng như là phương pháp ShapeDD, so sánh hiệu năng và cách thức hoạt động của 
chúng, bằng cách này luận văn sẽ hiểu rõ hơn về mục tiêu, cách thức hoạt động của các phương pháp, từ đó có thể ứng dụng cải thiện cho phương pháp ShapeDD.
\subsection{Ứng dụng những đề xuất cải tiến}
Sau khi nắm bắt được cơ sở lý thuyết, cách hoạt động của ShapeDD, luận văn sẽ ứng dụng tiến hành đánh giá hiệu năng trên các 
trường hợp trôi dạt khác nhau, dựa trên các những gì quan sát được, luận văn sẽ đề xuất áp dụng những cải thiện cho phương pháp 
hiện có để có thể thích nghi tốt hơn với các môi trường khác nhau.
\subsection{Kết hợp với mô hình phân loại hiện tượng trôi dạt cùng với CDT\_MSW}
Từ những gì đã nghiên cứu và cải thiện, luận văn sẽ kết hợp với mô hình phân loại hiện tượng trôi dạt CDT\_MSW 
để xây dựng một hệ thống phát hiện và thích ứng trôi dạt hoàn chỉnh, bằng cách vừa xác định và phân loại để có 
thể đưa ra quyết định chính xác cho việc cập nhật mô hình.
\subsection{Tùy chỉnh và tối ưu hóa phương pháp}
Luận văn sẽ tiến hành các thử nghiệm, tùy chỉnh các tham số của các phương pháp để đảm bảo rằng mô hình phù hợp và 
hiệu quả với bài toán phát hiện trôi dạt với các kiểu dữ liệu cũng như hiện tượng trôi dạt khác nhau.
\subsection{Triển khai và đánh giá}
Cuối cùng, luận văn sẽ triển khai hệ thống phát hiện và thích ứng trôi dạt trên môi trường dữ liệu streaming real-time 
sử dụng Apache Kafka, đánh giá hiệu năng của hệ thống trong các kịch bản thực tế để đảm bảo tính khả thi và 
hiệu quả của giải pháp đề xuất.
\section{Giới hạn đề tài và đối tượng nghiên cứu}
\subsection{Phạm vi đề tài}
Phạm vi của đề tài này tập trung vào việc nghiên cứu và phát triển hệ thống phát hiện và thích ứng concept drift trong môi trường động. 
Cụ thể, luận văn sẽ tập trung vào việc phát triển hệ thống kết hợp các phương pháp sau:
\begin{itemize}
    \item Nghiên cứu và cải thiện phương pháp ShapeDD để phát hiện concept drift không giám sát, so sánh với các phương pháp phát hiện drift truyền thống.
    \item Tích hợp mô hình CDT\_MSW để phân loại các loại drift tự động, từ đó hỗ trợ việc lựa chọn chiến lược thích ứng phù hợp.
    \item Xây dựng các chiến lược thích ứng mô hình đa dạng tương ứng với từng loại drift được phát hiện.
    \item Thiết kế và triển khai kiến trúc streaming real-time sử dụng Apache Kafka cho việc xử lý event-driven có khả năng mở rộng.
    \item Tùy chỉnh và tối ưu hóa các tham số của phương pháp để đảm bảo hiệu quả với các kiểu dữ liệu và hiện tượng trôi dạt khác nhau.
    \item Đánh giá hiệu năng của hệ thống trên các tập dữ liệu thực tế và các kịch bản drift khác nhau.
\end{itemize}

\subsection{Đối tượng nghiên cứu}
Đối tượng nghiên cứu của đề tài này là các mô hình và phương pháp phát hiện, phân loại và thích ứng concept drift, bao gồm:
\begin{itemize}
    \item Phương pháp ShapeDD: luận văn sẽ nghiên cứu chi tiết cơ chế hoạt động của ShapeDD - một phương pháp phát hiện concept drift không giám sát, so sánh với các phương pháp truyền thống để hiểu rõ ưu nhược điểm và đề xuất các cải thiện phù hợp.
    \item Mô hình CDT\_MSW: luận văn sẽ nghiên cứu và tích hợp mô hình phân loại loại drift tự động CDT\_MSW để xác định chính xác loại drift xảy ra, từ đó hỗ trợ việc lựa chọn chiến lược thích ứng mô hình phù hợp.
    \item Các chiến lược thích ứng mô hình: luận văn sẽ nghiên cứu và phát triển các chiến lược thích ứng đa dạng tương ứng với từng loại drift (sudden, gradual, incremental, recurring) để duy trì hiệu suất của mô hình.
    \item Kiến trúc streaming real-time: luận văn sẽ nghiên cứu và triển khai kiến trúc xử lý dữ liệu streaming sử dụng Apache Kafka để đảm bảo khả năng phát hiện và thích ứng drift trong thời gian thực hoặc gần thời gian thực.
    \item Phương pháp tùy chỉnh và tối ưu hóa: luận văn sẽ tiến hành các thử nghiệm để tùy chỉnh và tối ưu các tham số của hệ thống nhằm đảm bảo tính hiệu quả và khả thi trong các môi trường dữ liệu streaming khác nhau.
\end{itemize}
