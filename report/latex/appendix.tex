\chapter*{PHỤ LỤC}
\addcontentsline{toc}{chapter}{PHỤ LỤC}

\section*{Phụ lục A: Ghi chú về số liệu thực nghiệm}

Các bảng số liệu thống kê chi tiết từ benchmark (bao gồm độ chính xác, độ trễ, và hiệu suất) đã được trình bày đầy đủ và cập nhật mới nhất trong Chương 4 (Thực nghiệm và Đánh giá). Phụ lục này sẽ tập trung vào cung cấp mã nguồn minh họa và hướng dẫn triển khai hệ thống.

\section*{Phụ lục B: Mã nguồn và pseudocode}

\subsection*{B.1 Thuật toán ShapeDD (4 bước chính)}

\begin{algorithm}[H]
\caption{Shape-based Drift Detector (ShapeDD)}
\label{alg:shapedd}
\begin{algorithmic}[1]
\REQUIRE Data stream $S = \{x_1, x_2, \ldots, x_T\}$, window size $L_1$, kernel bandwidth $\sigma$, significance level $\alpha$
\ENSURE Set of drift points $D = \{t_1, t_2, \ldots\}$
\STATE \textbf{// Bước 1: Adaptive Window Strategy}
\STATE Initialize sliding windows: $W_{\text{ref}}$ (reference), $W_{\text{test}}$ (test)
\STATE Set $L_1 = 0.05 \times |S|$
\FOR{each time step $t$ in stream $S$}
    \STATE Update $W_{\text{ref}}$ and $W_{\text{test}}$ with new sample $x_t$
    \IF{$|W_{\text{ref}}| = L_1$ AND $|W_{\text{test}}| = L_1$}
        \STATE \textbf{// Bước 2: Compute Kernel Matrix}
        \STATE Combine windows: $W = W_{\text{ref}} \cup W_{\text{test}}$
        \STATE Estimate $\sigma$ using median heuristic: $\sigma = \text{median}(\{||x_i - x_j|| : i \neq j\})$
        \FOR{$i, j \in \{1, \ldots, 2L_1\}$}
            \STATE $K_{ij} = \exp\left(-\frac{||w_i - w_j||^2}{2\sigma^2}\right)$
        \ENDFOR
        \STATE \textbf{// Bước 3: Weighted MMD Computation}
        \STATE Create triangular weight function: $w_i = 1 - \frac{i}{L_1}$ for $i \leq L_1$, else $w_i = -1 + \frac{i - L_1}{L_1}$
        \STATE Compute weighted MMD: $\text{MMD}^2 = \sum_{i,j} w_i w_j K_{ij}$
        \STATE \textbf{// Bước 4: Shape Detection via Convolution}
        \STATE Apply convolution to detect zero-crossing
        \IF{zero-crossing detected at position $t_c$}
            \STATE \textbf{// Permutation Test}
            \STATE Compute $\text{MMD}^2_{\text{obs}}$ at $t_c$
            \FOR{$p = 1$ to $N_{\text{perm}}$}
                \STATE Randomly permute labels in $W$
                \STATE Compute $\text{MMD}^2_{\text{perm}}(p)$
            \ENDFOR
            \STATE Calculate p-value: $p = \frac{\#\{\text{MMD}^2_{\text{perm}} \geq \text{MMD}^2_{\text{obs}}\}}{N_{\text{perm}}}$
            \IF{$p < \alpha$}
                \STATE Add $t_c$ to drift set $D$
                \STATE \textbf{Signal drift detection}
            \ENDIF
        \ENDIF
    \ENDIF
\ENDFOR
\RETURN $D$
\end{algorithmic}
\end{algorithm}

\subsection*{B.2 Thuật toán CDT\_MSW (Phân loại loại drift)}

\begin{algorithm}[H]
\caption{Concept Drift Type Classification using Multi-Sliding Windows}
\label{alg:cdt_msw}
\begin{algorithmic}[1]
\REQUIRE Data stream $S$, detected drift point $t_d$, window sizes $\{w_1, w_2, \ldots, w_k\}$
\ENSURE Drift type classification $\in \{\text{Sudden, Gradual, Incremental, Recurrent, Blip}\}$
\STATE \textbf{// Extract data around drift point}
\STATE $W_{\text{before}} = S[t_d - w_{\max} : t_d]$
\STATE $W_{\text{after}} = S[t_d : t_d + w_{\max}]$
\STATE \textbf{// Compute MMD across multiple window sizes}
\FOR{each window size $w_i \in \{w_1, \ldots, w_k\}$}
    \STATE Compute $\text{MMD}^2_{w_i} = \text{MMD}^2(W_{\text{before}}[{-w_i:}], W_{\text{after}}[{:w_i}])$
\ENDFOR
\STATE \textbf{// Extract drift characteristics}
\STATE $\text{magnitude} = \max_i \text{MMD}^2_{w_i}$
\STATE $\text{slope} = \frac{d(\text{MMD}^2)}{dt}$ (gradient of MMD curve)
\STATE $\text{duration} = \arg\max_i \{\text{MMD}^2_{w_i} > \theta\}$ (transition period)
\STATE \textbf{// Classification based on shape features}
\IF{magnitude high AND slope steep}
    \RETURN \textbf{Sudden Drift}
\ELSIF{magnitude high AND slope moderate}
    \RETURN \textbf{Gradual Drift}
\ELSIF{magnitude increases monotonically}
    \RETURN \textbf{Incremental Drift}
\ELSIF{pattern repeats from history}
    \RETURN \textbf{Recurrent Drift}
\ELSIF{magnitude high but short duration}
    \RETURN \textbf{Blip Drift}
\ENDIF
\end{algorithmic}
\end{algorithm}

\subsection*{B.3 Code snippet: ShapeDD implementation trong Python}

\begin{verbatim}
import numpy as np
from scipy.spatial.distance import pdist, squareform
from scipy.stats import permutation_test

class ShapeDD:
    def __init__(self, window_size_ratio=0.05, alpha=0.05,
                 n_permutations=1000):
        self.window_size_ratio = window_size_ratio
        self.alpha = alpha
        self.n_permutations = n_permutations

    def _compute_kernel_matrix(self, X):
        """Compute Gaussian RBF kernel matrix with median heuristic"""
        pairwise_dists = pdist(X, metric='euclidean')
        sigma = np.median(pairwise_dists)
        K = squareform(np.exp(-squareform(pairwise_dists)**2 /
                               (2 * sigma**2)))
        return K

    def _weighted_mmd(self, K, L1):
        """Compute weighted MMD with triangular shape"""
        n = K.shape[0]
        weights = np.concatenate([
            1 - np.arange(L1) / L1,  # Decreasing triangle
            -1 + np.arange(L1) / L1  # Increasing triangle
        ])
        mmd_squared = np.sum(weights[:, None] * weights[None, :] * K)
        return mmd_squared

    def detect(self, X):
        """Detect drift points in data stream X"""
        n = len(X)
        L1 = int(self.window_size_ratio * n)
        drift_points = []

        for t in range(L1, n - L1):
            window = X[t - L1 : t + L1]
            K = self._compute_kernel_matrix(window)
            mmd_obs = self._weighted_mmd(K, L1)

            # Permutation test
            null_mmds = []
            for _ in range(self.n_permutations):
                perm_idx = np.random.permutation(2 * L1)
                K_perm = K[perm_idx][:, perm_idx]
                null_mmds.append(self._weighted_mmd(K_perm, L1))

            p_value = np.mean(np.array(null_mmds) >= mmd_obs)
            if p_value < self.alpha:
                drift_points.append(t)

        return drift_points
\end{verbatim}

\subsection*{B.4 Code snippet: Full Model Reset adaptation}

\begin{verbatim}
from sklearn.linear_model import LogisticRegression
from sklearn.preprocessing import StandardScaler
from collections import deque

class AdaptiveMLPipeline:
    def __init__(self, adaptation_window=200):
        self.scaler = StandardScaler()
        self.model = LogisticRegression(random_state=42)
        self.adaptation_window = adaptation_window
        self.buffer = deque(maxlen=adaptation_window)

    def adapt_on_drift(self, drift_detected=False):
        """Full model reset when sudden drift detected"""
        if drift_detected and len(self.buffer) >= self.adaptation_window:
            # Extract recent data from buffer
            X_recent = [item[0] for item in self.buffer]
            y_recent = [item[1] for item in self.buffer]

            # Reset and retrain scaler
            self.scaler = StandardScaler()
            X_scaled = self.scaler.fit_transform(X_recent)

            # Reset and retrain model
            self.model = LogisticRegression(random_state=42)
            self.model.fit(X_scaled, y_recent)

            print(f"[ADAPTATION] Model reset completed with "
                  f"{len(self.buffer)} recent samples")

    def predict_and_update(self, x, y_true, drift_signal):
        """Make prediction and update buffer"""
        # Add to buffer
        self.buffer.append((x, y_true))

        # Adapt if drift detected
        self.adapt_on_drift(drift_detected=drift_signal)

        # Make prediction
        x_scaled = self.scaler.transform([x])
        y_pred = self.model.predict(x_scaled)

        return y_pred
\end{verbatim}

\section*{Phụ lục C: Hướng dẫn cài đặt và chạy thử nghiệm}

\subsection*{C.1 Yêu cầu hệ thống}

\textbf{Phần cứng tối thiểu:}
\begin{itemize}
    \item CPU: 4 cores (khuyến nghị 8 cores)
    \item RAM: 8 GB (khuyến nghị 16 GB cho các dataset lớn)
    \item Disk: 10 GB trống
    \item Network: Ethernet/WiFi cho kết nối Kafka (nếu distributed)
\end{itemize}

\textbf{Phần mềm:}
\begin{itemize}
    \item Operating System: Linux (Ubuntu 20.04+), macOS 11+, hoặc Windows 10+ với WSL2
    \item Python: 3.8, 3.9, hoặc 3.10
    \item Java: JDK 11 hoặc 17 (cho Apache Kafka)
    \item Docker: phiên bản 20.10+ (tùy chọn, cho Kafka containerized)
\end{itemize}

\subsection*{C.2 Cài đặt Python dependencies}

\textbf{Bước 1: Tạo virtual environment}
\begin{verbatim}
# Tạo môi trường ảo
python3 -m venv venv_thesis

# Kích hoạt
# Linux/macOS:
source venv_thesis/bin/activate
# Windows:
venv_thesis\Scripts\activate
\end{verbatim}

\textbf{Bước 2: Cài đặt thư viện cần thiết}
\begin{verbatim}
pip install --upgrade pip

# Core libraries
pip install numpy==1.23.5
pip install scipy==1.10.1
pip install scikit-learn==1.2.2
pip install pandas==2.0.1

# Streaming and drift detection
pip install river==0.18.0
pip install kafka-python==2.0.2

# Visualization
pip install matplotlib==3.7.1
pip install seaborn==0.12.2

# Jupyter for notebooks
pip install jupyter==1.0.0
pip install ipywidgets==8.0.6
\end{verbatim}

\subsection*{C.3 Cài đặt Apache Kafka}

\textbf{Phương án 1: Sử dụng Docker (Khuyến nghị)}
\begin{verbatim}
# Tải docker-compose.yml cho Kafka
wget https://raw.githubusercontent.com/confluentinc/\
cp-all-in-one/7.4.0-post/cp-all-in-one/docker-compose.yml

# Khởi chạy Kafka cluster
docker-compose up -d

# Kiểm tra Kafka đang chạy
docker-compose ps
\end{verbatim}

\textbf{Phương án 2: Cài đặt thủ công}
\begin{verbatim}
# Tải Kafka
wget https://downloads.apache.org/kafka/3.4.0/\
kafka_2.13-3.4.0.tgz
tar -xzf kafka_2.13-3.4.0.tgz
cd kafka_2.13-3.4.0

# Khởi chạy Zookeeper
bin/zookeeper-server-start.sh \
config/zookeeper.properties &

# Khởi chạy Kafka broker
bin/kafka-server-start.sh config/server.properties &

# Tạo topic cho stream
bin/kafka-topics.sh --create --topic drift-stream \
--bootstrap-server localhost:9092 --partitions 3 \
--replication-factor 1
\end{verbatim}

\subsection*{C.4 Chạy thử nghiệm từ Jupyter Notebook}

\textbf{Bước 1: Clone repository và mở notebook}
\begin{verbatim}
# Clone thesis repository
git clone https://github.com/username/concept-drift-thesis
cd concept-drift-thesis

# Mở Jupyter
jupyter notebook

# Mở file:
# experiments/notebooks/MultiDetectors_Evaluation_WithAdaptation.ipynb
\end{verbatim}

\textbf{Bước 2: Cấu trúc notebook chính}

Notebook \texttt{MultiDetectors\_Evaluation\_WithAdaptation.ipynb} bao gồm các phần:
\begin{enumerate}
    \item \textbf{Cell 1-3:} Import libraries và setup parameters
    \item \textbf{Cell 4-6:} Load datasets (SEA, STAGGER, Hyperplane)
    \item \textbf{Cell 7-10:} Initialize ShapeDD detector
    \item \textbf{Cell 11-15:} Run detection on 7 sudden drift scenarios
    \item \textbf{Cell 16-18:} CDT\_MSW drift type classification
    \item \textbf{Cell 19-22:} Full Model Reset adaptation
    \item \textbf{Cell 23-27:} Evaluate metrics (Precision, Recall, F1, Delay, Recovery)
    \item \textbf{Cell 28-30:} Visualization (detection plots, confusion matrix)
\end{enumerate}

\textbf{Bước 3: Chạy toàn bộ thử nghiệm}
\begin{verbatim}
# Trong Jupyter notebook, chạy tất cả cells:
Cell > Run All

# Hoặc chạy từng phần:
Shift + Enter (để chạy cell hiện tại và chuyển sang cell tiếp theo)

# Kết quả xuất hiện ở cuối notebook:
# - Bảng kết quả detection (F1=1.0, delay=4 samples)
# - Bảng kết quả adaptation (Recovery=82.8%)
# - Các biểu đồ trực quan hóa
\end{verbatim}

\subsection*{C.5 Cấu trúc thư mục source code}

\begin{verbatim}
concept-drift-thesis/
|-- experiments/
|   |-- notebooks/
|   |   \-- MultiDetectors_Evaluation_WithAdaptation.ipynb
|   |-- data/
|   |   |-- sea_abrupt_1.csv
|   |   |-- stagger_abrupt_1.csv
|   |   \-- hyperplane_sudden_1.csv
|   \-- results/
|       \-- detection_results.csv
|-- src/
|   |-- detectors/
|   |   |-- shapedd.py
|   |   \-- cdt_msw.py
|   |-- adaptation/
|   |   |-- full_model_reset.py
|   |   \-- adaptive_pipeline.py
|   \-- utils/
|       |-- evaluation_metrics.py
|       \-- visualization.py
|-- kafka_integration/
|   |-- producer.py
|   |-- consumer.py
|   \-- drift_processor.py
|-- report/
|   \-- latex/
|       \-- (cac file .tex cua luan van)
|-- requirements.txt
\-- README.md
\end{verbatim}

\subsection*{C.6 Khắc phục sự cố thường gặp}

\textbf{Lỗi 1: "Kafka connection refused"}
\begin{verbatim}
# Kiểm tra Kafka đang chạy
netstat -an | grep 9092

# Nếu không thấy, restart Kafka
docker-compose restart  # Nếu dùng Docker
# hoặc
bin/kafka-server-start.sh config/server.properties  # Nếu manual
\end{verbatim}

\textbf{Lỗi 2: "ModuleNotFoundError: No module named 'river'"}
\begin{verbatim}
# Đảm bảo virtual environment đã được kích hoạt
source venv_thesis/bin/activate

# Cài lại river
pip install river==0.18.0
\end{verbatim}

\textbf{Lỗi 3: "MemoryError during kernel matrix computation"}
\begin{verbatim}
# Giảm window size trong ShapeDD initialization
detector = ShapeDD(window_size_ratio=0.03)  # Thay vì 0.05

# Hoặc xử lý stream theo batch nhỏ hơn
\end{verbatim}

\section*{Phụ lục D: Kiến trúc hệ thống chi tiết}

\subsection*{D.1 Sơ đồ luồng dữ liệu (Data Flow Diagram)}

Hệ thống tích hợp theo kiến trúc 3 tầng:

\textbf{Tầng 1: Data Ingestion Layer}
\begin{itemize}
    \item \textbf{Kafka Producer:} Đọc dữ liệu từ streaming sources (sensors, logs, databases)
    \item \textbf{Kafka Topics:} \texttt{raw-stream-topic} (dữ liệu thô), \texttt{processed-stream-topic} (đã xử lý)
    \item \textbf{Schema Registry:} Quản lý schema versioning cho data evolution
\end{itemize}

\textbf{Tầng 2: Drift Detection \& Classification Layer}
\begin{itemize}
    \item \textbf{ShapeDD Consumer:} Nhận stream từ Kafka, áp dụng sliding window
    \item \textbf{MMD Computation Engine:} Tính toán kernel matrix và weighted MMD
    \item \textbf{CDT\_MSW Classifier:} Phân loại loại drift khi phát hiện
    \item \textbf{Drift Event Publisher:} Gửi drift alerts tới \texttt{drift-events-topic}
\end{itemize}

\textbf{Tầng 3: Adaptation \& Monitoring Layer}
\begin{itemize}
    \item \textbf{Adaptation Manager:} Nhận drift events, chọn strategy (Full Model Reset)
    \item \textbf{Model Registry:} Lưu trữ model versions (pre-drift, post-adaptation)
    \item \textbf{Monitoring Dashboard:} Real-time visualization của detection và adaptation metrics
\end{itemize}

\subsection*{D.2 Cấu hình Kafka topics}

\begin{verbatim}
# Topic cho raw data stream
bin/kafka-topics.sh --create \
  --topic raw-stream-topic \
  --bootstrap-server localhost:9092 \
  --partitions 3 \
  --replication-factor 1 \
  --config retention.ms=86400000

# Topic cho drift events
bin/kafka-topics.sh --create \
  --topic drift-events-topic \
  --bootstrap-server localhost:9092 \
  --partitions 1 \
  --replication-factor 1 \
  --config retention.ms=604800000

# Topic cho adaptation results
bin/kafka-topics.sh --create \
  --topic adaptation-results-topic \
  --bootstrap-server localhost:9092 \
  --partitions 1 \
  --replication-factor 1
\end{verbatim}

\subsection*{D.3 Message format}

\textbf{Raw Stream Message (JSON):}
\begin{verbatim}
{
  "timestamp": "2024-01-15T10:30:45.123Z",
  "sample_id": 12345,
  "features": [0.234, -0.567, 1.234, 0.891],
  "label": 1,
  "metadata": {
    "source": "sensor_A",
    "batch_id": "batch_001"
  }
}
\end{verbatim}

\textbf{Drift Event Message (JSON):}
\begin{verbatim}
{
  "timestamp": "2024-01-15T10:35:12.456Z",
  "drift_detected": true,
  "drift_point": 5234,
  "drift_type": "sudden",
  "mmd_value": 0.0234,
  "p_value": 0.001,
  "confidence": 0.999,
  "affected_features": [0, 2, 3]
}
\end{verbatim}

\textbf{Adaptation Result Message (JSON):}
\begin{verbatim}
{
  "timestamp": "2024-01-15T10:36:00.789Z",
  "adaptation_strategy": "full_model_reset",
  "samples_used": 200,
  "pre_drift_accuracy": 0.92,
  "post_drift_accuracy": 0.54,
  "post_adaptation_accuracy": 0.89,
  "recovery_rate": 0.850,
  "model_version": "v2.3.1"
}
\end{verbatim}

\section*{Phụ lục E: Thuật ngữ tiếng Việt - tiếng Anh}

\begin{table}[h]
\centering
\caption{Bảng thuật ngữ chuyên ngành}
\label{tab:terminology}
\begin{tabular}{|l|l|}
\hline
\textbf{Tiếng Việt} & \textbf{Tiếng Anh} \\
\hline
Trôi dạt khái niệm & Concept Drift \\
Phát hiện drift & Drift Detection \\
Thích ứng & Adaptation \\
Luồng dữ liệu & Data Stream \\
Cửa sổ trượt & Sliding Window \\
Không gian tái tạo hạt nhân Hilbert & Reproducing Kernel Hilbert Space (RKHS) \\
Độ lệch trung bình cực đại & Maximum Mean Discrepancy (MMD) \\
Hàm kernel Gaussian RBF & Gaussian RBF Kernel \\
Kiểm định hoán vị & Permutation Test \\
Độ trễ phát hiện & Detection Delay \\
Báo động giả & False Alarm \\
Tỷ lệ phục hồi & Recovery Rate \\
Drift đột ngột & Sudden Drift \\
Drift dần dần & Gradual Drift \\
Drift tăng dần & Incremental Drift \\
Drift tuần hoàn & Recurrent Drift \\
Drift nhấp nháy & Blip Drift \\
Đặt lại mô hình hoàn toàn & Full Model Reset \\
Học trực tuyến & Online Learning \\
Học theo lô & Batch Learning \\
Cân bằng lại phân phối & Distribution Rebalancing \\
Mô hình đông lạnh & Frozen Model \\
Môi trường không dừng & Non-stationary Environment \\
\hline
\end{tabular}
\end{table}

\section*{Phụ lục F: Tài liệu tham khảo mở rộng}

\subsection*{F.1 Papers quan trọng về Concept Drift}

\begin{enumerate}
    \item \textbf{Survey chính (2024):}
    \begin{itemize}
        \item Hinder et al., "One or two things we know about concept drift—a survey on monitoring in evolving environments. Part A: detecting concept drift"
        \item Hinder et al., "Part B: locating and explaining concept drift"
    \end{itemize}

    \item \textbf{ShapeDD method (2024):}
    \begin{itemize}
        \item Hinder et al., "Shape-Based Method for Concept Drift Detection and Signal Denoising", Frontiers in AI
    \end{itemize}

    \item \textbf{Drift type classification:}
    \begin{itemize}
        \item Guo et al. (2022), "Concept drift type identification based on multi-sliding windows (CDT\_MSW)", Information Sciences
    \end{itemize}

    \item \textbf{Classical detectors:}
    \begin{itemize}
        \item Gama et al. (2004), "Learning with Drift Detection (DDM)"
        \item Bifet \& Gavaldà (2007), "ADWIN: Adaptive Windowing"
        \item Ross et al. (2012), "EDDM: Early Drift Detection Method"
    \end{itemize}
\end{enumerate}

\subsection*{F.2 Datasets và benchmarks}

\begin{itemize}
    \item \textbf{River library:} \url{https://riverml.xyz/} - Synthetic datasets (SEA, STAGGER, Hyperplane)
    \item \textbf{Real-world datasets:}
    \begin{itemize}
        \item Electricity pricing (Harries, 1999)
        \item Weather prediction
        \item Network intrusion detection (KDD Cup)
    \end{itemize}
\end{itemize}

\subsection*{F.3 Tools và frameworks}

\begin{itemize}
    \item \textbf{scikit-learn:} \url{https://scikit-learn.org/} - ML algorithms
    \item \textbf{River:} \url{https://riverml.xyz/} - Online learning library
    \item \textbf{Apache Kafka:} \url{https://kafka.apache.org/} - Distributed streaming platform
    \item \textbf{Jupyter:} \url{https://jupyter.org/} - Interactive notebooks
\end{itemize}

\vspace{1cm}

\noindent\textit{---Hết Phụ lục---}
