\documentclass[aspectratio=169,12pt]{beamer}

% Vietnamese language support
\usepackage[utf8]{inputenc}
\usepackage[T1]{fontenc}
\usepackage[vietnamese,english]{babel}
\usepackage[utf8]{vntex}

% Theme and color scheme
\usetheme{Madrid}
\usecolortheme{default}

% Additional packages
\usepackage{graphicx}
\usepackage{amsmath,amssymb}
\usepackage{algorithm}
\usepackage{algorithmic}
\usepackage{booktabs}
\usepackage{tikz}
\usepackage{pgfplots}
\pgfplotsset{compat=1.18}

% Custom colors
\definecolor{univblue}{RGB}{0,51,102}
\definecolor{univred}{RGB}{204,0,0}
\setbeamercolor{structure}{fg=univblue}
\setbeamercolor{title}{fg=white,bg=univblue}
\setbeamercolor{frametitle}{fg=white,bg=univblue}

% Remove navigation symbols
\setbeamertemplate{navigation symbols}{}

% Footer
\setbeamertemplate{footline}{
  \leavevmode%
  \hbox{%
  \begin{beamercolorbox}[wd=.333333\paperwidth,ht=2.25ex,dp=1ex,center]{author in head/foot}%
    \usebeamerfont{author in head/foot}\insertshortauthor
  \end{beamercolorbox}%
  \begin{beamercolorbox}[wd=.333333\paperwidth,ht=2.25ex,dp=1ex,center]{title in head/foot}%
    \usebeamerfont{title in head/foot}\insertshorttitle
  \end{beamercolorbox}%
  \begin{beamercolorbox}[wd=.333333\paperwidth,ht=2.25ex,dp=1ex,right]{date in head/foot}%
    \usebeamerfont{date in head/foot}\insertshortdate{}\hspace*{2em}
    \insertframenumber{} / \inserttotalframenumber\hspace*{2ex}
  \end{beamercolorbox}}%
  \vskip0pt%
}

% Title page information
\title[Phát Hiện Concept Drift]{Nghiên cứu và phát triển hệ thống tự động phát hiện hiện tượng trôi dạt và cập nhật mô hình học máy thích ứng}
\subtitle{Luận văn tốt nghiệp}
\author[Tên sinh viên]{%
  Sinh viên: [Tên sinh viên]\\[0.5em]
  \small Giảng viên hướng dẫn: [Tên GVHD]
}
\institute[Trường Đại học]{Trường Đại học Bách Khoa}
\date{\today}

\begin{document}

% ==============================================================================
% TITLE SLIDE
% ==============================================================================
\begin{frame}
  \titlepage
\end{frame}

% ==============================================================================
% TABLE OF CONTENTS
% ==============================================================================
\begin{frame}{Nội dung trình bày}
  \tableofcontents
\end{frame}

% ==============================================================================
% SECTION 1: INTRODUCTION
% ==============================================================================
\section{Giới thiệu}

\begin{frame}{Bối cảnh nghiên cứu}
  \begin{columns}[T]
    \begin{column}{0.5\textwidth}
      \textbf{Thách thức:}
      \begin{itemize}
        \item Các ứng dụng học máy triển khai trong đời sống thực tế
        \item Dữ liệu thay đổi theo thời gian (non-stationary)
        \item Mô hình bị suy giảm hiệu suất
      \end{itemize}
    \end{column}
    \begin{column}{0.5\textwidth}
      \textbf{Ví dụ thực tế:}
      \begin{itemize}
        \item Phát hiện thư rác: Spammer thay đổi chiến lược
        \item Dự báo tài chính: Thị trường biến động
        \item Chẩn đoán y tế: Bệnh mới xuất hiện
      \end{itemize}
    \end{column}
  \end{columns}

  \vspace{1em}
  \begin{alertblock}{Concept Drift}
    Hiện tượng phân phối dữ liệu thay đổi theo thời gian, làm cho mô hình đã học không còn phù hợp.
  \end{alertblock}
\end{frame}

\begin{frame}{Vấn đề nghiên cứu}
  \begin{block}{Câu hỏi nghiên cứu chính}
    \begin{enumerate}
      \item Làm thế nào để \textbf{phát hiện} khi nào drift xảy ra?
      \item Làm thế nào để \textbf{phân loại} loại drift (sudden, gradual, incremental)?
      \item Làm thế nào để \textbf{thích ứng} mô hình một cách tự động?
    \end{enumerate}
  \end{block}

  \vspace{1em}
  \begin{columns}[T]
    \begin{column}{0.48\textwidth}
      \textbf{Thách thức kỹ thuật:}
      \begin{itemize}
        \item Độ chính xác phát hiện
        \item Độ trễ phát hiện thấp
        \item Chi phí tính toán hợp lý
      \end{itemize}
    \end{column}
    \begin{column}{0.48\textwidth}
      \textbf{Yêu cầu hệ thống:}
      \begin{itemize}
        \item Hoạt động real-time
        \item Tự động hóa hoàn toàn
        \item Dễ triển khai thực tế
      \end{itemize}
    \end{column}
  \end{columns}
\end{frame}

\begin{frame}{Mục tiêu nghiên cứu}
  \begin{enumerate}
    \item \textbf{Nghiên cứu lý thuyết:}
    \begin{itemize}
      \item Phân tích nền tảng toán học của Shape Drift Detector (ShapeDD)
      \item Maximum Mean Discrepancy (MMD) trong RKHS
    \end{itemize}

    \vspace{0.5em}
    \item \textbf{Xây dựng hệ thống:}
    \begin{itemize}
      \item Tích hợp ShapeDD với phân loại drift (CDT\_MSW)
      \item Phát triển chiến lược thích ứng tự động
    \end{itemize}

    \vspace{0.5em}
    \item \textbf{Đánh giá thực nghiệm:}
    \begin{itemize}
      \item So sánh với các phương pháp state-of-the-art
      \item Đo lường hiệu suất trên nhiều kịch bản drift
    \end{itemize}

    \vspace{0.5em}
    \item \textbf{Triển khai thực tế:}
    \begin{itemize}
      \item Xây dựng hệ thống end-to-end với Apache Kafka
      \item Hướng dẫn triển khai production
    \end{itemize}
  \end{enumerate}
\end{frame}

% ==============================================================================
% SECTION 2: THEORETICAL FOUNDATION
% ==============================================================================
\section{Cơ sở lý thuyết}

\begin{frame}{Các loại Concept Drift}
  \begin{figure}
    \centering
    \begin{tikzpicture}[scale=0.9]
      % Sudden drift
      \draw[thick, blue] (0,2) -- (2,2);
      \draw[thick, red] (2,2) -- (2,1) -- (4,1);
      \node[above] at (2,2.3) {\small \textbf{Sudden}};
      \draw[dashed] (2,0) -- (2,2.5);

      % Gradual drift
      \draw[thick, blue] (5,2) -- (6,2);
      \draw[thick, red] (6,2) to[out=0,in=180] (8,1);
      \node[above] at (7,2.3) {\small \textbf{Gradual}};

      % Incremental drift
      \draw[thick, blue] (9,2) -- (10,2);
      \draw[thick, red] (10,2) -- (10.5,1.7) -- (11,1.4) -- (11.5,1.1) -- (12,1);
      \node[above] at (11,2.3) {\small \textbf{Incremental}};

      % Recurrent drift
      \draw[thick, blue] (0,-0.5) -- (1,-0.5);
      \draw[thick, red] (1,-0.5) -- (2,-1.5);
      \draw[thick, blue] (2,-1.5) -- (3,-1.5);
      \draw[thick, red] (3,-1.5) -- (4,-0.5);
      \node[above] at (2,0) {\small \textbf{Recurrent}};

      % Blip
      \draw[thick, blue] (5,-0.5) -- (6,-0.5);
      \draw[thick, red] (6,-0.5) -- (6.5,-1.5) -- (7,-0.5);
      \draw[thick, blue] (7,-0.5) -- (8,-0.5);
      \node[above] at (6.5,0) {\small \textbf{Blip}};
    \end{tikzpicture}
  \end{figure}

  \begin{itemize}
    \item \textcolor{blue}{Sudden}: Thay đổi đột ngột, ngay lập tức
    \item \textcolor{blue}{Gradual}: Thay đổi dần dần qua thời gian dài
    \item \textcolor{blue}{Incremental}: Thay đổi từng bước nhỏ
    \item \textcolor{blue}{Recurrent}: Các khái niệm cũ xuất hiện lại
    \item \textcolor{blue}{Blip}: Thay đổi tạm thời rồi quay lại
  \end{itemize}
\end{frame}

\begin{frame}{Shape Drift Detector (ShapeDD) - Ý tưởng cơ bản}
  \begin{block}{Ý tưởng chính}
    Khi có drift, sự khác biệt giữa dữ liệu trước và sau tạo ra một \textbf{"hình tam giác"} (đỉnh nhọn) trong đường cong MMD.
  \end{block}

  \vspace{0.5em}
  \textbf{Quy trình 3 bước đơn giản:}

  \begin{enumerate}
    \item \textbf{Tính shape statistic:}
    \begin{itemize}
      \item Chia stream thành cửa sổ trượt (kích thước 2×L1)
      \item So sánh hai nửa cửa sổ bằng kernel RBF
      \item Tạo đường cong "shape" biểu diễn sự khác biệt
    \end{itemize}

    \vspace{0.3em}
    \item \textbf{Tìm đỉnh (zero-crossing):}
    \begin{itemize}
      \item Tính đạo hàm: shape\_prime[i] = shape[i] × shape[i+1]
      \item Nếu shape\_prime < 0 → Đỉnh (đổi dấu từ + sang -)
    \end{itemize}

    \vspace{0.3em}
    \item \textbf{Kiểm định MMD:}
    \begin{itemize}
      \item Với mỗi đỉnh: lấy window xung quanh (L2=150 samples)
      \item Chạy MMD test (2500 permutations)
      \item p-value < 0.05 → Drift thật!
    \end{itemize}
  \end{enumerate}
\end{frame}

\begin{frame}{ShapeDD - Ví dụ cụ thể}
  \textbf{Scenario: Stream có drift tại sample 1000}

  \vspace{6pt}
  \textbf{Bước 1: Tính shape curve}
  \begin{itemize}
    \item Cửa sổ tại position 995: shape[995] = 0.2
    \item Cửa sổ tại position 997: shape[997] = 0.6 (tăng)
    \item Cửa sổ tại position 999: shape[999] = 0.4 (giảm)
  \end{itemize}

  \vspace{6pt}
  \textbf{Bước 2: Tìm đỉnh (zero-crossing)}
  \begin{itemize}
    \item shape\_prime[997] = shape[997] × shape[998] = 0.6 × (-0.2) = -0.12 < 0
    \item \alert{→ Phát hiện đỉnh tại position 997!}
  \end{itemize}

  \vspace{6pt}
  \textbf{Bước 3: Kiểm định MMD}
  \begin{itemize}
    \item Lấy window [922, 1072] (150 samples xung quanh 997)
    \item Chạy 2500 permutations → p-value = 0.003 < 0.05
    \item \alert{→ Xác nhận: DRIFT thật tại position 997!}
  \end{itemize}

  \vspace{6pt}
  \begin{alertblock}{Kết luận}
    Phát hiện drift với độ trễ = 997 - 1000 = -3 samples (phát hiện trước!)
  \end{alertblock}
\end{frame}

\begin{frame}{ShapeDD - Ưu và nhược điểm}
  \begin{columns}[T]
    \begin{column}{0.48\textwidth}
      \textbf{Ưu điểm:}
      \begin{itemize}
        \item \textbf{Độ chính xác cao} với sudden drift
        \item \textbf{Độ trễ thấp:} Phát hiện gần real-time
        \item \textbf{Không cần nhãn:} Unsupervised
        \item \textbf{Phát hiện vị trí chính xác:} Zero-crossing localization
      \end{itemize}
    \end{column}
    \begin{column}{0.48\textwidth}
      \textbf{Hạn chế:}
      \begin{itemize}
        \item \textbf{Threshold thô:} shape[pos] > 0 (quá nhiều candidates)
        \item \textbf{Bỏ lỡ gradual drift:} Chỉ tìm đỉnh nhọn
        \item \textbf{Fixed scale:} L1=50 không phù hợp mọi tốc độ drift
        \item \textbf{Chi phí tính toán cao:} 2500 permutations/candidate
      \end{itemize}
    \end{column}
  \end{columns}

  \vspace{8pt}
  \begin{block}{Động lực cải tiến}
    Các hạn chế này dẫn đến các cải tiến: Baseline\_Adaptive, SNR\_Adaptive, GradualAware, MultiScale
  \end{block}
\end{frame}

\begin{frame}{Maximum Mean Discrepancy (MMD)}
  \begin{definition}[MMD]
    MMD đo khoảng cách giữa hai phân phối $P$ và $Q$ trong không gian RKHS:
    \begin{equation*}
      \text{MMD}^2(P, Q) = \left\| \mathbb{E}_{x \sim P}[\phi(x)] - \mathbb{E}_{y \sim Q}[\phi(y)] \right\|^2_{\mathcal{H}}
    \end{equation*}
  \end{definition}

  \vspace{0.5em}
  \textbf{Ước lượng thực nghiệm:}
  \begin{equation*}
    \widehat{\text{MMD}}^2 = \frac{1}{n^2} \sum_{i,j} k(x_i, x_j) + \frac{1}{m^2} \sum_{i,j} k(y_i, y_j) - \frac{2}{nm} \sum_{i,j} k(x_i, y_j)
  \end{equation*}

  \vspace{0.5em}
  \begin{alertblock}{Tính chất quan trọng}
    MMD = 0 $\Leftrightarrow$ P = Q (với kernel universal)
  \end{alertblock}
\end{frame}

\begin{frame}{Kernel Functions - RBF Kernel trong MMD}
  \textbf{RBF (Gaussian) Kernel:} (Sử dụng trong ShapeDD)
  \begin{equation*}
    k(x, y) = \exp\left(-\gamma \|x - y\|^2\right), \quad \gamma = \frac{1}{2\sigma^2}
  \end{equation*}

  \vspace{8pt}
  \textbf{Vai trò trong MMD:}
  \begin{itemize}
    \item \textbf{Ánh xạ không gian:} Biến đổi dữ liệu vào Reproducing Kernel Hilbert Space (RKHS)
    \item \textbf{So sánh phân phối:} MMD đo khoảng cách giữa mean embeddings của 2 distributions
    \item \textbf{Nhạy cảm drift:} Kernel tốt giúp phát hiện sự khác biệt tinh vi giữa distributions
  \end{itemize}

  \vspace{8pt}
  \begin{columns}[T]
    \begin{column}{0.5\textwidth}
      \textbf{Lựa chọn $\gamma$:}
      \begin{itemize}
        \item \textbf{Scott's rule:}
        \[
          \sigma = n^{-1/(d+4)} \cdot \text{std}(X)
        \]
        \item Tự động điều chỉnh theo độ phân tán dữ liệu
        \item Phù hợp với số chiều $d$
      \end{itemize}
    \end{column}
    \begin{column}{0.5\textwidth}
      \textbf{Ảnh hưởng của $\gamma$:}
      \begin{itemize}
        \item $\gamma$ \textbf{nhỏ} ($\sigma$ lớn):
        \begin{itemize}
          \item Kernel mượt, tổng quát hóa
          \item Ít false alarms
          \item Có thể bỏ lỡ drift nhỏ
        \end{itemize}
        \item $\gamma$ \textbf{lớn} ($\sigma$ nhỏ):
        \begin{itemize}
          \item Kernel sắc, nhạy chi tiết
          \item Phát hiện drift nhỏ
          \item Nhiều false alarms hơn
        \end{itemize}
      \end{itemize}
    \end{column}
  \end{columns}

  \vspace{8pt}
  \begin{block}{Tại sao quan trọng?}
    $\gamma$ tối ưu cân bằng giữa \textbf{sensitivity} (phát hiện drift) và \textbf{specificity} (tránh false alarms)
  \end{block}
\end{frame}


% ==============================================================================
% CÁC CẢI TIẾN SHAPEDD (CHỈ CÁC PHƯƠNG PHÁP THỰC SỰ ĐƯỢC SỬ DỤNG)
% ==============================================================================

\begin{frame}{Cải tiến 1: Baseline Adaptive - Sửa lỗi kỹ thuật}
\textbf{Phương pháp:} ShapeDD\_Baseline\_Adaptive

\vspace{8pt}
\textbf{Ba cải tiến kỹ thuật:}

\begin{enumerate}
  \item \textbf{Adaptive Gamma (RBF kernel):}
  \begin{itemize}
    \item \textit{Vấn đề:} ShapeDD gốc dùng $\gamma=1.0$ cố định → không phù hợp mọi dữ liệu
    \item \textit{Giải pháp:} Dùng Scott's rule hoặc median heuristic tự động tính $\gamma$
    \item \textit{Ý nghĩa:} Dataset khác nhau → kernel tự thích nghi, không cần chỉnh tay
  \end{itemize}

  \vspace{4pt}
  \item \textbf{Minimal Smoothing (window=3):}
  \begin{itemize}
    \item \textit{Vấn đề:} Nhiễu nhỏ tạo false peaks
    \item \textit{Giải pháp:} Smoothing nhẹ (window=3) giữ nguyên drift signal
    \item \textit{Ý nghĩa:} Giảm false positive mà vẫn giữ độ chính xác vị trí drift
  \end{itemize}

  \vspace{4pt}
  \item \textbf{Sensitivity = 'none':}
  \begin{itemize}
    \item Tắt adaptive threshold để làm baseline
    \item Chứng minh: Chỉ sửa lỗi kỹ thuật KHÔNG ĐỦ để cải thiện performance
  \end{itemize}
\end{enumerate}
\end{frame}

\begin{frame}{Cải tiến 2: SNR-Adaptive - Tự động chọn chiến lược}
\textbf{Phương pháp:} ShapeDD\_SNR\_Adaptive (MAIN CONTRIBUTION)

\vspace{8pt}
\textbf{Ý tưởng chính:}
\begin{itemize}
  \item \alert{Insight:} Không có phương pháp tối ưu cho MỌI trường hợp!
  \item High SNR (drift rõ) → Dùng aggressive (high recall)
  \item Low SNR (drift yếu/nhiễu) → Dùng conservative (high precision)
\end{itemize}

\vspace{8pt}
\textbf{Quy trình hoạt động:}
\begin{enumerate}
  \item \textbf{Ước lượng SNR:} Tính tỷ số signal variance / noise variance
  \begin{itemize}
    \item Signal variance = biến thiên giữa các windows
    \item Noise variance = biến thiên trong window
  \end{itemize}

  \item \textbf{Chọn chiến lược:}
  \begin{itemize}
    \item SNR > 0.010 → shape\_adaptive\_v2 (aggressive, sensitivity='medium')
    \item SNR ≤ 0.010 → shape gốc (conservative)
  \end{itemize}

  \item \textbf{Kết quả:} Tự động hóa hoàn toàn, không cần chỉnh tay
\end{enumerate}
\end{frame}

\begin{frame}{Cải tiến 3: GradualAware - Bắt drift từ từ}
\textbf{Phương pháp:} ShapeDD\_GradualAware

\vspace{8pt}
\textbf{Vấn đề cần giải quyết:}
\begin{itemize}
  \item ShapeDD gốc chỉ tìm đỉnh nhọn (zero-crossing)
  \item Gradual drift tạo plateau (vùng phẳng), không phải đỉnh
  \item \alert{F1 gradual drift: 0.20 (rất tệ!)}
\end{itemize}

\vspace{8pt}
\textbf{Giải pháp: Dual Detection}
\begin{columns}[T]
  \begin{column}{0.48\textwidth}
    \textbf{Detection 1: Peak}
    \begin{itemize}
      \item Giữ nguyên zero-crossing
      \item Bắt abrupt drift
    \end{itemize}
  \end{column}
  \begin{column}{0.48\textwidth}
    \textbf{Detection 2: Plateau (MỚI!)}
    \begin{itemize}
      \item Curvature thấp (< 0.05)
      \item Slope nhỏ (< 0.1)
      \item Elevated (> baseline)
      \item Sustained (≥ 100 samples)
    \end{itemize}
  \end{column}
\end{columns}

\vspace{8pt}
\textbf{Kết quả:} F1 gradual drift: 0.20 → \alert{0.50-0.65} (2-3× cải thiện!)
\end{frame}

\begin{frame}{Cải tiến 4: MultiScale - Nhiều tốc độ drift}
\textbf{Phương pháp:} ShapeDD\_MultiScale

\vspace{8pt}
\textbf{Vấn đề:}
\begin{itemize}
  \item Fixed window (L1=50) không bắt được mọi tốc độ drift
  \item Fast drift cần window nhỏ, slow drift cần window lớn
\end{itemize}

\vspace{8pt}
\textbf{Giải pháp: 4-Scale Matched Filter}
\begin{itemize}
  \item \textbf{L1=25:} Bắt fast drift (25-50 samples)
  \item \textbf{L1=50:} Bắt medium-fast (50-100 samples)
  \item \textbf{L1=100:} Bắt medium-slow (100-200 samples)
  \item \textbf{L1=200:} Bắt slow drift (200-500 samples)
\end{itemize}

\vspace{8pt}
\textbf{Fusion: OR-rule}
\begin{itemize}
  \item Detect nếu BẤT KỲ scale nào signals drift
  \item $p_{final} = \min(p_{25}, p_{50}, p_{100}, p_{200})$
\end{itemize}

\vspace{8pt}
\textbf{Ý nghĩa:} Robust với mọi drift speed, không cần biết trước loại drift

\textbf{Trade-off:} Chậm 4× (chạy 4 scales)
\end{frame}

\begin{frame}{So sánh các phương pháp ShapeDD}
\begin{table}
  \centering
  \footnotesize
  \begin{tabular}{lcccc}
    \toprule
    \textbf{Phương pháp} & \textbf{F1 Score} & \textbf{Đặc điểm chính} & \textbf{Khi nào dùng} \\
    \midrule
    ShapeDD (gốc) & 0.592 & Baseline, threshold=0 & So sánh baseline \\
    \midrule
    Baseline\_Adaptive & 0.563 & Sửa lỗi kỹ thuật & Chứng minh cần adaptive \\
    \midrule
    \rowcolor{yellow!20}
    \textbf{SNR\_Adaptive} & \textbf{0.607} & \textbf{Auto-select strategy} & \textbf{MAIN - Universal} \\
    \midrule
    GradualAware & 0.60-0.62 & Dual peak+plateau & Có gradual drift \\
    \midrule
    MultiScale & 0.62-0.65 & 4-scale filter & Mixed drift speeds \\
    \bottomrule
  \end{tabular}
\end{table}

\vspace{8pt}
\textbf{Evolution Track:}
\begin{center}
  ShapeDD (baseline) → Baseline\_Adaptive (fixes) \\
  → \alert{SNR\_Adaptive (auto-selection)} ← \textbf{MAIN} \\
  ↙ \hspace{3cm} ↘ \\
  GradualAware \hspace{2cm} MultiScale
\end{center}

\vspace{8pt}
\textbf{Contribution chính:} SNR-Adaptive tự động hóa việc chọn chiến lược, không cần chỉnh tay, hoạt động tốt trên mọi loại dữ liệu
\end{frame}

\section{Hệ thống đề xuất}
\begin{frame}{Kiến trúc tổng thể}
  \begin{figure}
    \centering
    \begin{tikzpicture}[
      node distance=1.5cm,
      block/.style={rectangle, draw, fill=blue!20, text width=3cm, text centered, rounded corners, minimum height=1cm},
      arrow/.style={thick,->,>=stealth}
    ]
      \node[block] (input) {Data Stream};
      \node[block, right of=input, xshift=2cm] (detect) {ShapeDD\\Drift Detection};
      \node[block, right of=detect, xshift=2cm] (classify) {CDT\_MSW\\Drift Classification};
      \node[block, below of=classify, yshift=-0.5cm] (adapt) {Adaptive Strategy};
      \node[block, below of=input, yshift=-0.5cm] (model) {ML Model};

      \draw[arrow] (input) -- (detect);
      \draw[arrow] (detect) -- node[above] {\small Drift?} (classify);
      \draw[arrow] (classify) -- node[right] {\small Type} (adapt);
      \draw[arrow] (adapt) -| (model);
      \draw[arrow] (input) |- (model);
      \draw[arrow] (model.east) -- ++(1,0) |- (detect.south);
    \end{tikzpicture}
  \end{figure}

  \textbf{Luồng hoạt động:}
  \begin{enumerate}
    \item Luồng dữ liệu được giám sát liên tục bởi ShapeDD
    \item Khi phát hiện drift → Kích hoạt phân loại (CDT\_MSW)
    \item Dựa vào loại drift → Áp dụng chiến lược thích ứng phù hợp
    \item Cập nhật mô hình và tiếp tục giám sát
  \end{enumerate}
\end{frame}

\begin{frame}{Thuật toán ShapeDD}
  \begin{algorithmic}[1]
    \REQUIRE Data stream $X$, window sizes $l_1, l_2$, permutations $n_{perm}$
    \ENSURE Drift positions and p-values
    \STATE Compute kernel matrix $K$ using RBF kernel
    \STATE Compute sliding MMD statistics using convolution
    \STATE Apply smoothing to reduce noise
    \STATE Compute shape statistics (second derivative)
    \STATE Identify candidate peaks (zero-crossings)
    \FOR{each candidate peak $pos$}
      \STATE Extract window around $pos$ with size $l_2$
      \STATE Run permutation test to compute p-value
      \IF{p-value $< 0.05$}
        \STATE \textbf{Report drift at position} $pos$
      \ENDIF
    \ENDFOR
  \end{algorithmic}

  \vspace{0.5em}
  \begin{block}{Cải tiến trong nghiên cứu}
    \textbf{ShapeDD\_Adaptive\_v2:} Ngưỡng thích ứng + FDR correction có điều kiện
  \end{block}
\end{frame}

\begin{frame}{Drift Classification (CDT\_MSW)}
  \textbf{Phương pháp Multiple Sliding Windows:}
  \begin{itemize}
    \item Sử dụng 3 cửa sổ khác nhau: nhỏ, trung bình, lớn
    \item So sánh tỷ lệ phát hiện giữa các cửa sổ
    \item Phân loại dựa trên pattern của detections
  \end{itemize}

  \vspace{1em}
  \begin{table}
    \centering
    \small
    \begin{tabular}{lccc}
      \toprule
      \textbf{Loại Drift} & \textbf{Window S} & \textbf{Window M} & \textbf{Window L} \\
      \midrule
      Sudden      & High & High & High \\
      Gradual     & Low  & Med  & High \\
      Incremental & Med  & Med  & High \\
      Recurrent   & High & Med  & Low  \\
      Blip        & High & Low  & Low  \\
      \bottomrule
    \end{tabular}
  \end{table}
\end{frame}

\begin{frame}{Chiến lược thích ứng}
  \textbf{Dựa vào loại drift được phân loại:}

  \begin{itemize}
    \item \textbf{Sudden Drift:}
    \begin{itemize}
      \item \alert{Reset model} hoàn toàn
      \item Huấn luyện lại trên dữ liệu mới
      \item Kích thước cửa sổ: 500 samples
    \end{itemize}

    \vspace{0.3em}
    \item \textbf{Gradual/Incremental:}
    \begin{itemize}
      \item \alert{Incremental learning} (cập nhật dần dần)
      \item Learning rate thấp để tránh catastrophic forgetting
      \item Cửa sổ lớn hơn: 1000 samples
    \end{itemize}

    \vspace{0.3em}
    \item \textbf{Recurrent:}
    \begin{itemize}
      \item \alert{Ensemble models} cho các concept đã thấy
      \item Lưu trữ models cũ trong repository
      \item Reactive reactivation khi pattern quay lại
    \end{itemize}

    \vspace{0.3em}
    \item \textbf{Blip:}
    \begin{itemize}
      \item \alert{Ignore} (không cập nhật)
      \item Chờ xác nhận drift thực sự
    \end{itemize}
  \end{itemize}
\end{frame}

\begin{frame}{Triển khai với Apache Kafka}
  \begin{columns}[T]
    \begin{column}{0.5\textwidth}
      \textbf{Kiến trúc streaming:}
      \begin{itemize}
        \item \textbf{Producer:} Sinh dữ liệu liên tục
        \item \textbf{Kafka:} Message broker
        \item \textbf{Consumer:} Phát hiện drift
        \item \textbf{Trainer:} Cập nhật model
      \end{itemize}

      \vspace{1em}
      \textbf{Topics:}
      \begin{itemize}
        \item \texttt{data-stream}
        \item \texttt{drift-alerts}
        \item \texttt{model-updates}
      \end{itemize}
    \end{column}
    \begin{column}{0.5\textwidth}
      \begin{figure}
        \centering
        \begin{tikzpicture}[scale=0.7,
          block/.style={rectangle, draw, fill=blue!20, text width=2cm, text centered, rounded corners, minimum height=0.8cm, font=\tiny},
          arrow/.style={->,>=stealth}
        ]
          \node[block] (producer) {Producer};
          \node[block, below of=producer, yshift=-0.5cm] (kafka) {Kafka};
          \node[block, below of=kafka, yshift=-0.5cm] (consumer) {Consumer};
          \node[block, below of=consumer, yshift=-0.5cm] (detector) {Detector};
          \node[block, below of=detector, yshift=-0.5cm] (trainer) {Trainer};

          \draw[arrow] (producer) -- (kafka);
          \draw[arrow] (kafka) -- (consumer);
          \draw[arrow] (consumer) -- (detector);
          \draw[arrow] (detector) -- (trainer);
          \draw[arrow] (trainer.west) -- ++(-0.5,0) |- (consumer.west);
        \end{tikzpicture}
      \end{figure}
    \end{column}
  \end{columns}
\end{frame}

% ==============================================================================
% SECTION 4: EXPERIMENTS
% ==============================================================================
\section{Thực nghiệm và đánh giá}

\begin{frame}{Thiết kế thực nghiệm}
  \textbf{17 Datasets × 5 ShapeDD methods = 85 experiments}

  \vspace{8pt}
  \textbf{Datasets theo loại drift:}
  \begin{itemize}
    \item \textbf{Sudden drift (8):} standard\_sea, enhanced\_sea, stagger, hyperplane, gen\_random (4 levels)
    \item \textbf{Gradual drift (4):} sea\_gradual, stagger\_gradual, hyperplane\_gradual, gen\_gradual
    \item \textbf{Incremental drift (2):} incremental\_clusters, incremental\_noise
    \item \textbf{Real-world (1):} electricity
    \item \textbf{Stationary (2):} static\_uniform, static\_normal (false positive test)
  \end{itemize}

  \vspace{8pt}
  \textbf{Cấu hình chung:}
  \begin{itemize}
    \item Stream size: 10,000 samples
    \item Number of drifts: 10 per stream
    \item Consistent windows: $l_1 = 50$, $l_2 = 150$ (fair comparison)
    \item Permutations: 2,500
    \item Buffer size: 750 samples, check every 150 samples
  \end{itemize}
\end{frame}

\begin{frame}{Metrics đánh giá}
  \begin{columns}[T]
    \begin{column}{0.5\textwidth}
      \textbf{Detection metrics:}
      \begin{itemize}
        \item \textbf{F1-Score:} Cân bằng precision-recall
        \item \textbf{Detection Rate:} Tỷ lệ drift được phát hiện
        \item \textbf{MTTD:} Mean Time To Detect (độ trễ)
      \end{itemize}

      \vspace{1em}
      \textbf{Adaptation metrics:}
      \begin{itemize}
        \item \textbf{Accuracy:} Độ chính xác phân loại
        \item \textbf{Recovery rate:} Tốc độ phục hồi
        \item \textbf{Degradation:} Mức độ suy giảm
      \end{itemize}
    \end{column}
    \begin{column}{0.5\textwidth}
      \textbf{Công thức tính:}
      \begin{align*}
        \text{Precision} &= \frac{TP}{TP + FP} \\[0.5em]
        \text{Recall} &= \frac{TP}{TP + FN} \\[0.5em]
        \text{F1} &= \frac{2 \times P \times R}{P + R} \\[0.5em]
        \text{MTTD} &= \frac{1}{n} \sum_{i=1}^{n} |t_i^{detect} - t_i^{true}|
      \end{align*}
    \end{column}
  \end{columns}
\end{frame}

% ==============================================================================
% SECTION 5: SOTA LANDSCAPE 2024-2025
% ==============================================================================
\section{Bối cảnh SOTA 2024-2025}

\begin{frame}{SOTA Methods (2024-2025)}
  \begin{table}
    \centering
    \footnotesize
    \begin{tabular}{llccc}
      \toprule
      \textbf{Method} & \textbf{Type} & \textbf{F1} & \textbf{Labels} & \textbf{Year} \\
      \midrule
      \textbf{CDSeer} & Semi-supervised & \textbf{0.86} & \textbf{1\%} & 2024 \\
      \textbf{DriftLens} & Unsupervised & 15/17 wins & 0\% & 2024 \\
      \textbf{CV4CDD-4D} & Supervised & 0.81-0.83 & 100\% & 2025 \\
      \textbf{ADA-ADF} & Time series & \textbf{0.92} & 0\% & 2025 \\
      \textbf{ShapeDD} & Unsupervised & 0.758 & 0\% & 2024 \\
      ADWIN & Unsupervised & 0.507 & 0\% & 2007 \\
      HDDM-W & Model-dependent & 0.80 & Yes & 2016 \\
      \bottomrule
    \end{tabular}
  \end{table}
\end{frame}

\begin{frame}{CDSeer: Semi-Supervised SOTA (Oct 2024)}
  \begin{columns}[T]
    \begin{column}{0.48\textwidth}
      \textbf{Kết quả ấn tượng:}
      \begin{itemize}
        \item \textcolor{red}{\textbf{57.1\% precision improvement}}
        \item \textcolor{red}{\textbf{99\% fewer labels}} (1\% vs 100\%)
        \item F1-score: \textbf{0.86}
        \item Validated at \textbf{Ericsson}
      \end{itemize}

      \vspace{0.5em}
      \textbf{Technical approach:}
      \begin{itemize}
        \item Confidence-based active learning
        \item Model-agnostic
        \item Distribution-agnostic
      \end{itemize}
    \end{column}
  \end{columns}

  \vspace{0.3em}
  \textbf{Citation:} arXiv 2410.09190 (Oct 2024, updated Aug 2025)
\end{frame}

\begin{frame}{DriftLens: Real-Time Unsupervised (June 2024)}
  \begin{columns}[T]
    \begin{column}{0.48\textwidth}
      \textbf{Performance:}
      \begin{itemize}
        \item \textbf{15/17 wins} (88\% win rate)
        \item \textbf{5× faster} than competitors
        \item \textbf{Correlation ≥ 0.85} with true drift
        \item Fully unsupervised
      \end{itemize}

      \vspace{0.5em}
      \textbf{Key innovations:}
      \begin{itemize}
        \item Deep learning embeddings
        \item Per-label distribution tracking
        \item Built-in explainability
        \item Open-source (PyPI)
      \end{itemize}
    \end{column}
    \begin{column}{0.48\textwidth}
      \textbf{Architecture:}
      \begin{enumerate}
        \item Offline: Estimate reference distributions
        \item Online: Process windows
        \item Monitor: Distribution distances
        \item Explain: Representative samples
      \end{enumerate}

      \vspace{0.5em}
      \textbf{Ưu điểm cho production:}
      \begin{itemize}
        \item Real-time capability
        \item Low complexity
        \item Works on unstructured data
        \item Drift characterization
      \end{itemize}
    \end{column}
  \end{columns}

  \vspace{0.3em}
  \textbf{Citation:} arXiv 2406.17813
\end{frame}

\begin{frame}{Other Notable SOTA Methods}
  \begin{table}
    \centering
    \tiny
    \begin{tabular}{llp{3cm}c}
      \toprule
      \textbf{Method} & \textbf{Specialty} & \textbf{Key Innovation} & \textbf{Performance} \\
      \midrule
      \textbf{Meta-ADD} & Meta-learning & Automatic detector selection & Pre-trained classifier \\
      \textbf{CDDRM} & Explainability & Feature-level drift + causal & F1 = 0.895 (Stagger) \\
      \textbf{AEF-CDA} & Ensemble & Medical IoT streams & Acc = 99.64\% \\
      \textbf{Transformer} & Time series & Temporal attention & Few-shot learning \\
      \textbf{ARF} & Online ensemble & Per-tree ADWIN & Production standard \\
      \textbf{HAT} & Adaptive trees & Built-in drift handling & Multi-label capable \\
      \bottomrule
    \end{tabular}
  \end{table}

  \vspace{0.5em}
  \begin{block}{Emerging Paradigms (2024)}
    \begin{itemize}
      \item \textbf{Test-Time Adaptation (TTA):} Adapt without source data
      \item \textbf{Continual Learning:} Avoid catastrophic forgetting
      \item \textbf{OOD Detection:} Separate drift from outliers
      \item \textbf{Bayesian UQ:} Epistemic vs aleatoric uncertainty
    \end{itemize}
  \end{block}
\end{frame}

% ==============================================================================
% SECTION 6: SYSTEM ARCHITECTURES
% ==============================================================================
\section{Kiến trúc hệ thống}

\begin{frame}{4 Architecture Patterns Chính}
  \begin{enumerate}
    \item \textbf{Model-Agnostic (65\% research - Chúng ta đang dùng)}
    \begin{itemize}
      \item Detector monitors: Feature distributions (X)
      \item Model: Bất kỳ (LogReg, RF, NN)
      \item Example: ShapeDD + LogisticRegression
      \item Flexible, May miss performance-relevant drift
    \end{itemize}

    \vspace{0.3em}
    \item \textbf{Model-Dependent (30\% research)}
    \begin{itemize}
      \item Detector monitors: Model errors/confidence
      \item Model: Coupled with detector
      \item Example: DDM, EDDM, ADWIN, CDSeer
      \item Detects relevant drift, Requires labels
    \end{itemize}

    \vspace{0.3em}
    \item \textbf{Integrated Model-Detector (20\%, growing)}
    \begin{itemize}
      \item Model có built-in detection
      \item Example: Adaptive Random Forest, HAT
      \item Fully automated, Model-specific
    \end{itemize}

    \vspace{0.3em}
    \item \textbf{Production MLOps (Industry standard)}
    \begin{itemize}
      \item Complete pipeline: Kafka + Flink + Monitoring
      \item Scalable, enterprise-ready
    \end{itemize}
  \end{enumerate}
\end{frame}

\begin{frame}{Kiến trúc hiện tại của chúng ta}
  \begin{figure}
    \centering
    \begin{tikzpicture}[
      box/.style={rectangle, draw=univblue, thick, fill=blue!10, minimum width=3cm, minimum height=0.8cm, align=center},
      arrow/.style={->, thick, >=stealth}
    ]
      % Data stream
      \node[box] (data) at (0,3) {Data Stream X};

      % Component 1: Model
      \node[box, fill=green!10] (model) at (0,1.5) {Component 1:\\LogisticRegression\\(FROZEN)};

      % Predictions
      \node[box, fill=yellow!10] (pred) at (0,0) {Predictions ŷ};

      % Component 2: Detector (parallel path)
      \node[box, fill=orange!10] (detector) at (5,1.5) {Component 2:\\ShapeDD SNR-Adaptive\\(Monitors X)};

      % Component 3: Adaptation
      \node[box, fill=red!10] (adapt) at (5,0) {Component 3:\\Retrain Strategy};

      % Arrows
      \draw[arrow] (data) -- (model);
      \draw[arrow] (model) -- (pred);
      \draw[arrow, dashed] (data) -- (detector);
      \draw[arrow] (detector) -- node[right] {Drift?} (adapt);
      \draw[arrow, dashed] (adapt) to[bend right=45] node[below] {Retrain} (model);
    \end{tikzpicture}
  \end{figure}

  \begin{alertblock}{Pattern: Model-Agnostic ✅}
    Detector và Model \textbf{độc lập} - có thể swap model mà không thay detector
  \end{alertblock}
\end{frame}

\begin{frame}{Production MLOps Architecture (Industry 2024)}
  \begin{figure}
    \centering
    \tiny
    \begin{tikzpicture}[
      box/.style={rectangle, draw, thick, minimum width=2cm, minimum height=0.6cm, align=center, font=\tiny},
      arrow/.style={->, thick, >=stealth}
    ]
      % Layers
      \node[box, fill=blue!20] (kafka) at (0,3) {Kafka\\Ingestion};
      \node[box, fill=green!20] (flink) at (0,2) {Flink\\Processing};
      \node[box, fill=yellow!20] (model) at (0,1) {Model\\Inference};
      \node[box, fill=orange!20] (detector) at (3,2) {Drift\\Detection};
      \node[box, fill=red!20] (monitor) at (3,1) {Performance\\Monitor};
      \node[box, fill=purple!20] (retrain) at (6,1.5) {Auto\\Retrain};
      \node[box, fill=cyan!20] (dash) at (6,0.5) {Grafana\\Dashboard};

      % Arrows
      \draw[arrow] (kafka) -- (flink);
      \draw[arrow] (flink) -- (model);
      \draw[arrow] (flink) -- (detector);
      \draw[arrow] (model) -- (monitor);
      \draw[arrow] (detector) -- (retrain);
      \draw[arrow] (monitor) -- (retrain);
      \draw[arrow] (retrain) -- (dash);
      \draw[arrow, dashed] (retrain) to[bend left=30] node[right, font=\tiny] {Deploy} (model);
    \end{tikzpicture}
  \end{figure}

  \vspace{0.3em}
  \textbf{Components:}
  \begin{itemize}
    \item \textbf{Kafka:} Stream ingestion (millions events/sec)
    \item \textbf{Flink:} Real-time processing (5× faster than Spark)
    \item \textbf{3-layer detection:} Data drift + Concept drift + Performance
    \item \textbf{Automated workflow:} Detect → Alert → Retrain → A/B test → Deploy
  \end{itemize}
\end{frame}

\begin{frame}{Streaming Models: ARF vs Batch Models}
  \begin{columns}[T]
    \begin{column}{0.48\textwidth}
      \textbf{Batch Model (Current):}
      \begin{itemize}
        \item LogisticRegression
        \item Train once, deploy frozen
        \item Retrain khi drift detected
        \item Full model replacement
      \end{itemize}

      \vspace{0.5em}
      \textbf{Ưu điểm:}
      \begin{itemize}
        \item Controlled experiments
        \item Reproducible
        \item Simple
      \end{itemize}

      \textbf{Nhược điểm:}
      \begin{itemize}
        \item High retraining cost
        \item Adaptation delay
        \item No online learning
      \end{itemize}
    \end{column}
    \begin{column}{0.48\textwidth}
      \textbf{Online Model (Proposal):}
      \begin{itemize}
        \item Adaptive Random Forest
        \item Incremental learning (sample-by-sample)
        \item Built-in per-tree ADWIN
        \item Dynamic tree replacement
      \end{itemize}

      \vspace{0.5em}
      \textbf{Ưu điểm:}
      \begin{itemize}
        \item Real-time adaptation
        \item Automatic drift handling
        \item Production-ready
      \end{itemize}

      \textbf{Nhược điểm:}
      \begin{itemize}
        \item Complex evaluation
        \item Less control
      \end{itemize}
    \end{column}
  \end{columns}

  \vspace{0.5em}
  \begin{block}{Proposal: Hybrid Approach}
    Keep ShapeDD (global) + Add ARF (per-tree ADWIN) → Compare both patterns
  \end{block}
\end{frame}

% ==============================================================================
% BACKUP SLIDES
% ==============================================================================
\appendix

\begin{frame}[noframenumbering]{Backup: Confusion Matrix}
  \begin{table}
    \centering
    \begin{tabular}{cc|cc}
      \multicolumn{2}{c}{} & \multicolumn{2}{c}{\textbf{Predicted}} \\
      & & Drift & No Drift \\
      \hline
      \multirow{2}{*}{\textbf{Actual}} & Drift & TP = 8 & FN = 2 \\
      & No Drift & FP = 2 & TN = 48 \\
    \end{tabular}
  \end{table}

  \vspace{1em}
  \begin{align*}
    \text{Precision} &= \frac{8}{8+2} = 0.80 \\
    \text{Recall} &= \frac{8}{8+2} = 0.80 \\
    \text{F1-Score} &= 0.80
  \end{align*}
\end{frame}

\begin{frame}[noframenumbering]{Backup: Runtime Comparison}
  \begin{table}
    \centering
    \small
    \begin{tabular}{lcc}
      \toprule
      \textbf{Method} & \textbf{Runtime (s)} & \textbf{Throughput (samples/s)} \\
      \midrule
      ShapeDD & 2.05 & 4,878 \\
      ShapeDD\_v2 & 1.15 & 8,696 \\
      ADWIN & 1.18 & 8,475 \\
      DAWIDD & 2.41 & 4,149 \\
      MMD & 0.52 & 19,231 \\
      KS & 0.21 & 47,619 \\
      \bottomrule
    \end{tabular}
  \end{table}

  \vspace{0.5em}
  \textbf{Nhận xét:} ShapeDD có chi phí tính toán cao hơn streaming methods, nhưng đạt độ chính xác tốt hơn đáng kể.
\end{frame}

\end{document}
