\documentclass[aspectratio=169,12pt]{beamer}

% Vietnamese language support
\usepackage[utf8]{inputenc}
\usepackage[T1]{fontenc}
\usepackage[vietnamese,english]{babel}
\usepackage[utf8]{vntex}

% Theme and color scheme
\usetheme{Madrid}
\usecolortheme{default}

% Additional packages
\usepackage{graphicx}
\usepackage{amsmath,amssymb}
\usepackage{algorithm}
\usepackage{algorithmic}
\usepackage{booktabs}
\usepackage{tikz}
\usepackage{pgfplots}
\usepackage{ragged2e} % For better text alignment
\pgfplotsset{compat=1.18}

% Custom colors
\definecolor{univblue}{RGB}{0,51,102}
\definecolor{univred}{RGB}{204,0,0}
\setbeamercolor{structure}{fg=univblue}
\setbeamercolor{title}{fg=white,bg=univblue}
\setbeamercolor{frametitle}{fg=white,bg=univblue}

% Remove navigation symbols
\setbeamertemplate{navigation symbols}{}

% Footer
\setbeamertemplate{footline}{
  \leavevmode%
  \hbox{%
  \begin{beamercolorbox}[wd=.333333\paperwidth,ht=2.25ex,dp=1ex,center]{author in head/foot}%
    \usebeamerfont{author in head/foot}\insertshortauthor
  \end{beamercolorbox}%
  \begin{beamercolorbox}[wd=.333333\paperwidth,ht=2.25ex,dp=1ex,center]{title in head/foot}%
    \usebeamerfont{title in head/foot}\insertshorttitle
  \end{beamercolorbox}%
  \begin{beamercolorbox}[wd=.333333\paperwidth,ht=2.25ex,dp=1ex,right]{date in head/foot}%
    \usebeamerfont{date in head/foot}\insertshortdate{}\hspace*{2em}
    \insertframenumber{} / \inserttotalframenumber\hspace*{2ex}
  \end{beamercolorbox}}%
  \vskip0pt%
}

% Title page information
\title[Phát Hiện Concept Drift]{Nghiên cứu và phát triển hệ thống tự động phát hiện hiện tượng trôi dạt và cập nhật mô hình học máy thích ứng}
\subtitle{Luận văn tốt nghiệp}
\author[Le Phuc Duc]{%
  Sinh viên thực hiện: Le Phuc Duc\\[0.5em]
  \small Giảng viên hướng dẫn: PGS.TS. Thoại Nam
}
\institute[HCMUT]{Trường Đại học Bách Khoa - ĐHQG-HCM}
\date{\today}

\begin{document}

% ==============================================================================
% TITLE SLIDE
% ==============================================================================
\begin{frame}
	\titlepage
\end{frame}

% ==============================================================================
% TABLE OF CONTENTS
% ==============================================================================
\begin{frame}{Nội dung trình bày}
	\tableofcontents
\end{frame}

% ==============================================================================
% SECTION 1: INTRODUCTION
% ==============================================================================
\section{Giới thiệu}

\begin{frame}{Bối cảnh nghiên cứu}
	\begin{columns}[T]
		\begin{column}{0.5\textwidth}
			\textbf{Vấn đề thực tiễn:}
			\begin{itemize}
				\item Môi trường dữ liệu luồng (Data Stream) luôn biến đổi.
				\item Giả định I.I.D bị phá vỡ trong thực tế.
				\item \alert{Concept Drift:} Sự thay đổi trong phân phối dữ liệu làm suy giảm hiệu suất mô hình.
			\end{itemize}
		\end{column}
		\begin{column}{0.5\textwidth}
			\textbf{Thách thức chính:}
			\begin{itemize}
				\item \textbf{High Throughput:} Xử lý hàng trăm ngàn mẫu/giây.
				\item \textbf{Label Latency:} Thiếu nhãn thật trong thời gian thực (Unsupervised).
				\item \textbf{Trade-off:} Cân bằng giữa tốc độ phát hiện, độ chính xác và chi phí tính toán.
			\end{itemize}
		\end{column}
	\end{columns}

	\vspace{1em}
	\begin{block}{Định nghĩa Concept Drift}
		$\exists t: P_t(X, Y) \neq P_{t+1}(X, Y)$
	\end{block}
\end{frame}

\begin{frame}{Mục tiêu nghiên cứu}
	\begin{enumerate}
		\item \textbf{Cải thiện hiệu năng phát hiện Drift:}
		      \begin{itemize}
			      \item Tích hợp ShapeDD với \textbf{IDW-MMD} và kiểm định tiệm cận (Asymptotic Test).
			      \item Mục tiêu: Tăng tốc độ xử lý và giảm False Positives.
		      \end{itemize}

		      \vspace{0.5em}
		\item \textbf{Phân loại Drift không giám sát:}
		      \begin{itemize}
			      \item Đề xuất phương pháp \textbf{SE-CDT} (ShapeDD-Enhanced Concept Drift Type).
			      \item Phân loại drift mà không cần nhãn (labels).
		      \end{itemize}

		      \vspace{0.5em}
		\item \textbf{Khung thích ứng tự động:}
		      \begin{itemize}
			      \item Xây dựng chiến lược cập nhật mô hình tương ứng với từng loại drift.
			      \item Triển khai hệ thống trên nền tảng \textbf{Apache Kafka}.
		      \end{itemize}
	\end{enumerate}
\end{frame}

% ==============================================================================
% SECTION 2: THEORETICAL FOUNDATION
% ==============================================================================
\section{Cơ sở lý thuyết & Phương pháp đề xuất}

\begin{frame}{Nền tảng: Shape Drift Detector (ShapeDD)}
	\begin{columns}[T]
		\begin{column}{0.6\textwidth}
			\textbf{Nguyên lý hoạt động:}
			\begin{itemize}
				\item Concept drift tạo ra sự khác biệt phân phối (Discrepancy) giữa dữ liệu quá khứ và hiện tại.
				\item Sự khác biệt này tạo thành một \textbf{"hình dạng" (shape)} đặc trưng trong tín hiệu giám sát.
				\item \textbf{Zero-crossing:} Thời điểm drift tương ứng với đỉnh của shape curve.
			\end{itemize}
		\end{column}
		\begin{column}{0.4\textwidth}
			\begin{alertblock}{Hạn chế của ShapeDD gốc}
				\begin{itemize}
					\item Phụ thuộc Permutation Test (chậm).
					\item Nhạy cảm với nhiễu nếu không chọn đúng Kernel.
				\end{itemize}
			\end{alertblock}
		\end{column}
	\end{columns}
\end{frame}

\begin{frame}{Cải tiến 1: IDW-MMD \& Asymptotic Test}
    \begin{columns}[T]
        \begin{column}{0.5\textwidth}
            \textbf{Vấn đề:}
            \begin{itemize}
                \item Standard MMD dùng trọng số đều.
                \item Permutation Test tốn $O(N_{perm} \cdot n^2)$.
            \end{itemize}
            
            \vspace{0.5em}
            \textbf{Giải pháp IDW-MMD:}
            \begin{itemize}
                \item Trọng số nghịch biến mật độ: $w_i \propto \frac{1}{\sqrt{\sum k(x_i, x_j)}}$.
                \item Tăng cường tín hiệu tại vùng biên (outliers).
            \end{itemize}
        \end{column}
        \begin{column}{0.5\textwidth}
            \textbf{Đột phá về tốc độ:}
            \begin{itemize}
                \item Thay thế Permutation Test bằng \textbf{Asymptotic Distribution} (xấp xỉ Gamma).
                \item \textbf{Kết quả:} Giảm độ phức tạp từ $O(B \cdot n^2)$ xuống $O(n^2)$.
                \item \textbf{Hiệu quả:} Tăng throughput từ $\sim$8,000 lên \textbf{$\sim$130,000 mẫu/giây}.
            \end{itemize}
        \end{column}
    \end{columns}
\end{frame}

\begin{frame}{Cải tiến 2: Phân loại SE-CDT (Unsupervised)}
    \textbf{Ý tưởng:} Sử dụng đặc trưng của tín hiệu drift magnitude $\sigma(t)$ để phân loại.
    
    \vspace{0.5em}
    \begin{table}
        \centering
        \small
        \begin{tabular}{l|l|l}
            \toprule
            \textbf{Feature} & \textbf{Công thức/Ý nghĩa} & \textbf{Phân loại} \\
            \midrule
            $N_{peaks}$ & Số lượng đỉnh drift & Sudden ($\le 4$), Recurrent (Nhiều) \\
            $WR$ & Width Ratio (Độ rộng đỉnh) & TCD (Hẹp, $<0.12$), PCD (Rộng) \\
            $CV$ & Coefficient of Variation & Recurrent (Đều đặn, $CV < 0.3$) \\
            $SNR$ & Tỷ lệ tín hiệu trên nhiễu & Sudden (Cao), Gradual (Thấp) \\
            \bottomrule
        \end{tabular}
    \end{table}
    
    \vspace{0.5em}
    \textbf{Ưu điểm:} Không cần nhãn thật để tính Accuracy Ratio như phương pháp cũ (CDT\_MSW).
\end{frame}

\section{Kết quả thực nghiệm}

\begin{frame}{Thiết lập thực nghiệm Benchmark}
    \textbf{Dữ liệu:} 11 Synthetic Datasets mô phỏng Sudden, Gradual, Incremental, Blip, và Virtual Drifts.
    \textbf{So sánh:} 8 phương pháp (ShapeDD gốc, D3, DAWIDD, MMD, KS-Test, v.v.).
    \textbf{Chỉ số:} F1-Score, Detection Delay, Throughput, False Positives.
    
    \vspace{1em}
    \begin{table}
        \centering
        \footnotesize
        \caption{Kết quả Benchmark Phát hiện Drift (Trung bình 30 lần chạy)}
        \begin{tabular}{lcccc}
            \toprule
            \textbf{Method} & \textbf{F1-Score} & \textbf{False Pos} & \textbf{Delay (samples)} & \textbf{Throughput} \\
            \midrule
            \rowcolor{green!10}
            \textbf{ShapeDD\_WMMD} & \textbf{0.481} & \textbf{0} & \textbf{18} & \textbf{131,280} \\
            ShapeDD (Original) & 0.440 & 4 & 25 & 6,399 \\
            D3 (Discriminative) & 0.517 & 121 & 35 & 27,693 \\
            KS-Test & 0.299 & 289 & 10 & 28,161 \\
            \bottomrule
        \end{tabular}
    \end{table}
    
    \footnotesize{\textit{*ShapeDD\_WMMD đạt FP=0 do chỉ phát hiện thay đổi P(X), không báo động sai trên virtual drift (P(Y|X) đổi).}}
\end{frame}

\begin{frame}{Chi tiết về 11 Datasets Thực nghiệm}
    \textbf{Phân loại theo bản chất thay đổi phân phối:}
    
    \vspace{0.5em}
    \textbf{Nhóm 1: Sudden Drift (P(X) thay đổi đột ngột):}
    \begin{itemize}
        \item \textit{Gaussian Shift:} Dịch chuyển trung bình ($\delta=1.5$) trong không gian 10 chiều
        \item \textit{STAGGER:} Thay đổi quy luật với tín hiệu P(X) rõ ràng
        \item \textit{Random RBF (4 mức):} Mild, Moderate, Severe, Ultra-severe ($\delta$ từ 0.125 đến 2.0)
    \end{itemize}
    
    \vspace{0.5em}
    \textbf{Nhóm 2: Blip Drift (thay đổi ngắn hạn):}
    \begin{itemize}
        \item \textit{RBF Blips:} Drift tạm thời, kiểm tra khả năng bắt tín hiệu nhanh
    \end{itemize}
    
    \vspace{0.5em}
    \textbf{Nhóm 3: Virtual Drift - Control Group (chỉ P(Y|X) đổi):}
    \begin{itemize}
        \item \textit{SEA, Hyperplane, LED Abrupt:} P(X) không đổi
        \item \textbf{Mục đích:} Kiểm chứng phương pháp unsupervised \textbf{không} phát hiện sai (FP = 0)
    \end{itemize}
\end{frame}

\begin{frame}{Đánh giá Phân loại Drift (SE-CDT)}
    \textbf{Bài toán:} Phân loại drift thành nhóm TCD (Transient) và PCD (Progressive).
    \textbf{Baseline:} CDT\_MSW (Supervised - Cần nhãn thật).
    
    \vspace{0.5em}
    \begin{columns}[T]
        \begin{column}{0.5\textwidth}
            \begin{table}
                \centering
                \small
                \begin{tabular}{lcc}
                    \toprule
                    \textbf{Method} & \textbf{CAT} & \textbf{SUB} \\
                    \midrule
                    \textbf{SE-CDT} & \textbf{85.8\%} & \textbf{46.6\%} \\
                    CDT\_MSW & 38.7\% & 18.7\% \\
                    \bottomrule
                \end{tabular}
                \caption*{\footnotesize CAT: Category (TCD vs PCD)\\SUB: Subcategory (5 loại chi tiết)}
            \end{table}
        \end{column}
        \begin{column}{0.5\textwidth}
            \textbf{Phân tích:}
            \begin{itemize}
                \item SE-CDT: Category 85.8\%, Subcategory 46.6\%
                \item \textbf{Thành công:} Phân biệt TCD/PCD rất tốt
                \item \textbf{Thách thức:} Gradual $\leftrightarrow$ Incremental dễ nhầm lẫn (tín hiệu MMD tương tự)
                \item \textbf{Trade-off:} EDR 94.4\% nhưng FP cao (1394)
            \end{itemize}
        \end{column}
    \end{columns}
    
    \vspace{0.5em}
    \footnotesize{\textit{*Kết quả trên 17 configurations × 10 runs = 170 test cases}}
\end{frame}

\section{Hệ thống đề xuất}
\begin{frame}{Kiến trúc tổng thể}
	\begin{figure}
		\centering
		\begin{tikzpicture}[
				node distance=1.5cm,
				block/.style={rectangle, draw, fill=blue!20, text width=3cm, text centered, rounded corners, minimum height=1cm},
				arrow/.style={thick,->,>=stealth}
			]
			\node[block] (input) {Data Stream};
			\node[block, right of=input, xshift=2cm] (detect) {ShapeDD\\Drift Detection};
			\node[block, right of=detect, xshift=2cm] (classify) {CDT\_MSW\\Drift Classification};
			\node[block, below of=classify, yshift=-0.5cm] (adapt) {Adaptive Strategy};
			\node[block, below of=input, yshift=-0.5cm] (model) {ML Model};

			\draw[arrow] (input) -- (detect);
			\draw[arrow] (detect) -- node[above] {\small Drift?} (classify);
			\draw[arrow] (classify) -- node[right] {\small Type} (adapt);
			\draw[arrow] (adapt) -| (model);
			\draw[arrow] (input) |- (model);
			\draw[arrow] (model.east) -- ++(1,0) |- (detect.south);
		\end{tikzpicture}
	\end{figure}

	\textbf{Luồng hoạt động:}
	\begin{enumerate}
		\item Luồng dữ liệu được giám sát liên tục bởi ShapeDD
		\item Khi phát hiện drift → Kích hoạt phân loại (CDT\_MSW)
		\item Dựa vào loại drift → Áp dụng chiến lược thích ứng phù hợp
		\item Cập nhật mô hình và tiếp tục giám sát
	\end{enumerate}
\end{frame}

\begin{frame}{Thuật toán ShapeDD}
	\begin{algorithmic}[1]
		\REQUIRE Data stream $X$, window sizes $l_1, l_2$, permutations $n_{perm}$
		\ENSURE Drift positions and p-values
		\STATE Compute kernel matrix $K$ using RBF kernel
		\STATE Compute sliding MMD statistics using convolution
		\STATE Apply smoothing to reduce noise
		\STATE Compute shape statistics (second derivative)
		\STATE Identify candidate peaks (zero-crossings)
		\FOR{each candidate peak $pos$}
		\STATE Extract window around $pos$ with size $l_2$
		\STATE Run permutation test to compute p-value
		\IF{p-value $< 0.05$}
		\STATE \textbf{Report drift at position} $pos$
		\ENDIF
		\ENDFOR
	\end{algorithmic}

	\vspace{0.5em}
	\begin{block}{Cải tiến trong nghiên cứu}
		\textbf{ShapeDD\_OW\_MMD:} Tích hợp trọng số tối ưu giúp giảm phương sai và tăng tốc độ.
	\end{block}
\end{frame}

\begin{frame}{Drift Classification (CDT\_MSW)}
	\textbf{Phương pháp Multiple Sliding Windows:}
	\begin{itemize}
		\item Sử dụng 3 cửa sổ khác nhau: nhỏ, trung bình, lớn
		\item So sánh tỷ lệ phát hiện giữa các cửa sổ
		\item Phân loại dựa trên pattern của detections
	\end{itemize}

	\vspace{1em}
	\begin{table}
		\centering
		\small
		\begin{tabular}{lccc}
			\toprule
			\textbf{Loại Drift} & \textbf{Window S} & \textbf{Window M} & \textbf{Window L} \\
			\midrule
			Sudden              & High              & High              & High              \\
			Gradual             & Low               & Med               & High              \\
			Incremental         & Med               & Med               & High              \\
			Recurrent           & High              & Med               & Low               \\
			Blip                & High              & Low               & Low               \\
			\bottomrule
		\end{tabular}
	\end{table}
\end{frame}

\begin{frame}{Chiến lược thích ứng}
	\textbf{Dựa vào loại drift được phân loại:}

	\begin{itemize}
		\item \textbf{Sudden Drift:}
		      \begin{itemize}
			      \item \alert{Reset model} hoàn toàn
			      \item Huấn luyện lại trên dữ liệu mới
			      \item Kích thước cửa sổ: 500 samples
		      \end{itemize}

		      \vspace{0.3em}
		\item \textbf{Gradual/Incremental:}
		      \begin{itemize}
			      \item \alert{Incremental learning} (cập nhật dần dần)
			      \item Learning rate thấp để tránh mô hình "quên" kiến thức cũ
			      \item Cửa sổ lớn hơn: 1000 samples
		      \end{itemize}

		      \vspace{0.3em}
		\item \textbf{Recurrent:}
		      \begin{itemize}
			      \item \alert{Ensemble models} cho các concept đã thấy
			      \item Lưu trữ models cũ trong repository
			      \item Reactive reactivation khi pattern quay lại
		      \end{itemize}

		      \vspace{0.3em}
		\item \textbf{Blip:}
		      \begin{itemize}
			      \item \alert{Ignore} (không cập nhật)
			      \item Chờ xác nhận drift thực sự
		      \end{itemize}
	\end{itemize}
\end{frame}

\begin{frame}{Triển khai với Apache Kafka}
	\begin{columns}[T]
		\begin{column}{0.5\textwidth}
			\textbf{Kiến trúc streaming:}
			\begin{itemize}
				\item \textbf{Producer:} Sinh dữ liệu liên tục
				\item \textbf{Kafka:} Message broker
				\item \textbf{Consumer:} Phát hiện drift
				\item \textbf{Trainer:} Cập nhật model
			\end{itemize}

			\vspace{1em}
			\textbf{Topics:}
			\begin{itemize}
				\item \texttt{data-stream}
				\item \texttt{drift-alerts}
				\item \texttt{model-updates}
			\end{itemize}
		\end{column}
		\begin{column}{0.5\textwidth}
			\begin{figure}
				\centering
				\begin{tikzpicture}[scale=0.7,
						block/.style={rectangle, draw, fill=blue!20, text width=2cm, text centered, rounded corners, minimum height=0.8cm, font=\tiny},
						arrow/.style={->,>=stealth}
					]
					\node[block] (producer) {Producer};
					\node[block, below of=producer, yshift=-0.5cm] (kafka) {Kafka};
					\node[block, below of=kafka, yshift=-0.5cm] (consumer) {Consumer};
					\node[block, below of=consumer, yshift=-0.5cm] (detector) {Detector};
					\node[block, below of=detector, yshift=-0.5cm] (trainer) {Trainer};

					\draw[arrow] (producer) -- (kafka);
					\draw[arrow] (kafka) -- (consumer);
					\draw[arrow] (consumer) -- (detector);
					\draw[arrow] (detector) -- (trainer);
					\draw[arrow] (trainer.west) -- ++(-0.5,0) |- (consumer.west);
				\end{tikzpicture}
			\end{figure}
		\end{column}
	\end{columns}
\end{frame}

% ==============================================================================
% SECTION 4: EXPERIMENTS
% ==============================================================================
\section{Thực nghiệm và đánh giá}

\begin{frame}{Thiết kế thực nghiệm (Cập nhật)}
	\textbf{11 Datasets × 8 Methods × 30 Runs = 2,640 Experiments}

	\vspace{8pt}
	\textbf{Phân loại Dataset theo đặc tính Drift:}
	\begin{itemize}
		\item \textbf{Sudden Drift (P(X) thay đổi):}
		      \begin{itemize}
			      \item \textit{Gaussian Shift:} Dịch chuyển trung bình ($\delta=1.5$)
			      \item \textit{STAGGER:} Thay đổi quy luật (có tín hiệu P(X))
			      \item \textit{Random RBF:} 4 mức độ (Mild $\to$ Ultra-severe)
		      \end{itemize}
		\item \textbf{Blip Drift:} \textit{RBF Blips} (thay đổi ngắn hạn)
		\item \textbf{Control Group (Chỉ P(Y|X) thay đổi):}
		      \begin{itemize}
			      \item \textit{Standard SEA, Hyperplane, LED Abrupt}
			      \item Mục tiêu: Kiểm chứng tính "bền vững" (không báo động sai) của detector không giám sát
		      \end{itemize}
	\end{itemize}

	\vspace{8pt}
	\textbf{Cấu hình chung:}
	\begin{itemize}
		\item Stream size: 10,000 samples (10 điểm drift)
		\item Windows: $L_1 = 50$, $L_2 = 150$
		\item Permutations: 2,500 (cho ShapeDD gốc) vs Optimized Weights (cho ADW-MMD)
	\end{itemize}
\end{frame}

\begin{frame}{Metrics đánh giá}
	\begin{columns}[T]
		\begin{column}{0.5\textwidth}
			\textbf{Detection metrics:}
			\begin{itemize}
				\item \textbf{F1-Score:} Cân bằng precision-recall
				\item \textbf{Detection Rate:} Tỷ lệ drift được phát hiện
				\item \textbf{MTTD:} Mean Time To Detect (độ trễ)
			\end{itemize}

			\vspace{1em}
			\textbf{Adaptation metrics:}
			\begin{itemize}
				\item \textbf{Accuracy:} Độ chính xác phân loại
				\item \textbf{Recovery rate:} Tốc độ phục hồi
				\item \textbf{Degradation:} Mức độ suy giảm
			\end{itemize}
		\end{column}
		\begin{column}{0.5\textwidth}
			\textbf{Công thức tính:}
			\begin{align*}
				\text{Precision} & = \frac{TP}{TP + FP}                                     \\[0.5em]
				\text{Recall}    & = \frac{TP}{TP + FN}                                     \\[0.5em]
				\text{F1}        & = \frac{2 \times P \times R}{P + R}                      \\[0.5em]
				\text{MTTD}      & = \frac{1}{n} \sum_{i=1}^{n} |t_i^{detect} - t_i^{true}|
			\end{align*}
		\end{column}
	\end{columns}
\end{frame}

% ==============================================================================
% SECTION 4: SYSTEM ARCHITECTURE
% ==============================================================================
\section{Kiến trúc hệ thống đề xuất}

\begin{frame}{Tổng quan kiến trúc Event-Driven}
	\begin{figure}
		\centering
		\includegraphics[width=0.8\textwidth]{image/system_architecture.png}
		\caption{Sơ đồ kiến trúc xử lý luồng dữ liệu thời gian thực với Apache Kafka}
	\end{figure}

	\begin{itemize}
		\item \textbf{Loose Coupling:} Các thành phần (Detector, Classifier, Adapter) hoạt động độc lập.
		\item \textbf{Scalability:} Dễ dàng mở rộng consumer group để xử lý throughput lớn.
		\item \textbf{Resilience:} Cơ chế Replication và Offset Management của Kafka đảm bảo an toàn dữ liệu.
	\end{itemize}
\end{frame}

\begin{frame}{Chiến lược thích ứng (Adaptation Framework)}
	\textbf{Ma trận quyết định dựa trên loại Drift:}

	\vspace{1em}
	\begin{table}
		\centering
		\small
		\begin{tabular}{|l|l|}
			\hline
			\textbf{Loại Drift} & \textbf{Chiến lược} \\
			\hline
			\textbf{Sudden} & \textbf{Full Reset:} Huấn luyện lại hoàn toàn trên dữ liệu mới. Tránh nhiễu từ quá khứ. \\
			\hline
			\textbf{Incremental} & \textbf{Continuous Update:} Cập nhật trọng số mô hình (Online Learning). \\
			\hline
			\textbf{Gradual} & \textbf{Weighted Window:} Cửa sổ trượt có trọng số, pha trộn dữ liệu cũ và mới. \\
			\hline
			\textbf{Recurrent} & \textbf{Model Reuse:} Kích hoạt lại mô hình cũ từ kho lưu trữ (Model Repository). \\
			\hline
			\textbf{Blip} & \textbf{Ignore:} Coi là nhiễu, không cập nhật để giữ ổn định. \\
			\hline
		\end{tabular}
	\end{table}
\end{frame}

% ==============================================================================
% SECTION 5: CONCLUSION
% ==============================================================================
\section{Kết luận & Hướng phát triển}

\begin{frame}{Tổng kết đóng góp}
	\begin{enumerate}
		\item \textbf{Về Phương pháp (Methodological):}
			\begin{itemize}
				\item Tích hợp IDW-MMD + Asymptotic Test vào ShapeDD.
				\item Đề xuất SE-CDT: Phân loại drift không giám sát dựa trên Shape features.
			\end{itemize}
		
		\item \textbf{Về Thực nghiệm (Empirical):}
			\begin{itemize}
				\item \textbf{Hiệu suất:} Thông lượng tăng \textbf{17--20 lần} ($\approx$ 131,000 mẫu/giây).
				\item \textbf{Độ tin cậy:} F1-Score = 0.481 với \textbf{0 False Positives} trên detection benchmark.
				\item \textbf{Phân loại:} Độ chính xác SE-CDT (85.8\%) cao hơn baseline CDT\_MSW (38.7\%).
			\end{itemize}

		\item \textbf{Về Hệ thống (System):}
			\begin{itemize}
				\item Thiết kế kiến trúc Streaming hoàn chỉnh với Apache Kafka.
			\end{itemize}
	\end{enumerate}
\end{frame}

\begin{frame}{Hạn chế \& Thách thức}
	\textbf{1. Thiết kế Unsupervised - Không phát hiện Virtual Drift:}
	\begin{itemize}
		\item ShapeDD chỉ phát hiện thay đổi $P(X)$, bỏ qua thay đổi $P(Y|X)$
		\item \textbf{Kết quả:} F1=0.0 trên SEA, Hyperplane, LED (mong đợi)
		\item \textit{Ý nghĩa tích cực:} FP=0, không báo động sai khi $P(X)$ ổn định
	\end{itemize}

	\vspace{0.5em}
	\textbf{2. Mild Drift - Thất bại hoàn toàn (F1=0.006):}
	\begin{itemize}
		\item IDW-MMD lọc quá mạnh, coi drift nhẹ là noise
		\item Cần điều chỉnh trọng số cho vùng mật độ cao
	\end{itemize}

	\vspace{0.5em}
	\textbf{3. Subcategory Classification hạn chế (46.6\%):}
	\begin{itemize}
		\item Gradual $\leftrightarrow$ Incremental: Tín hiệu MMD tương tự
		\item Recurrent: Xử lý như chuỗi Sudden riêng lẻ
	\end{itemize}

	\vspace{0.5em}
	\textbf{4. Đánh giá trên Synthetic Data:}
	\begin{itemize}
		\item Cần validation trên real-world traces (production logs)
	\end{itemize}
\end{frame}

\begin{frame}{Hướng phát triển tương lai}
	\textbf{Ưu tiên ngắn hạn (3-6 tháng):}
	\begin{enumerate}
		\item \textbf{Khắc phục Mild Drift:}
		\begin{itemize}
			\item Điều chỉnh IDW-MMD: Giảm ảnh hưởng trọng số trong vùng mật độ cao
			\item Adaptive threshold dựa trên noise level của dataset
		\end{itemize}
		
		\item \textbf{Cải thiện Subcategory Classification (46.6\% $\rightarrow$ 70\%+):}
		\begin{itemize}
			\item Thêm features: Slope analysis, Multi-scale analysis
			\item Phân biệt Gradual vs Incremental bằng tốc độ thay đổi
		\end{itemize}
	\end{enumerate}
	
	\vspace{0.5em}
	\textbf{Ưu tiên dài hạn (6-12 tháng):}
	\begin{enumerate}
		\setcounter{enumi}{2}
		\item \textbf{Real-world Validation:} Đánh giá trên production logs (Network traffic, IoT sensors)
		\item \textbf{Meta-Learning:} Tự động chọn detector + strategy dựa trên đặc tính stream
		\item \textbf{Benchmark chuẩn:} Đóng góp dataset có ground truth cho cộng đồng
	\end{enumerate}
\end{frame}

\begin{frame}[plain]
	\centering
	\Huge \textcolor{univblue}{\textbf{CẢM ƠN THẦY CÔ VÀ CÁC BẠN ĐÃ LẮNG NGHE!}}
	
	\vspace{2cm}
	\normalsize
	\textbf{Q \& A}
\end{frame}

\appendix
\begin{frame}[noframenumbering]{Backup: Runtime Details}
	\begin{table}
		\centering
		\small
		\begin{tabular}{lcc}
			\toprule
			\textbf{Method} & \textbf{Runtime (s)} & \textbf{Throughput (samples/s)} \\
			\midrule
			\textbf{ShapeDD\_OW\_MMD} & \textbf{0.19} & \textbf{131,280} \\
			MMD\_OW & 0.20 & 130,500 \\
			KS Test & 0.36 & 28,161 \\
			D3 & 0.36 & 27,693 \\
			ShapeDD (Original) & 1.56 & 6,399 \\
			DAWIDD & 3.07 & 3,253 \\
			\bottomrule
		\end{tabular}
	\end{table}
	\textit{Note: Thời gian đo trên cửa sổ 200 samples, trung bình 30 lần chạy.}
\end{frame}

\end{document}
