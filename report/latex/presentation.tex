\documentclass[aspectratio=169,12pt]{beamer}

% Vietnamese language support
\usepackage[utf8]{inputenc}
\usepackage[T1]{fontenc}
\usepackage[vietnamese,english]{babel}
\usepackage[utf8]{vntex}

% Theme and color scheme
\usetheme{Madrid}
\usecolortheme{default}

% Additional packages
\usepackage{graphicx}
\usepackage{booktabs}
\usepackage{tikz}
\usepackage{ragged2e}

% Define paths for images
\graphicspath{{image/}{../../results/plots/}}

% Custom colors
\definecolor{univblue}{RGB}{0,51,102}
\setbeamercolor{structure}{fg=univblue}
\setbeamercolor{title}{fg=white,bg=univblue}
\setbeamercolor{frametitle}{fg=white,bg=univblue}

% Remove navigation symbols
\setbeamertemplate{navigation symbols}{}

% Footer setup
\setbeamertemplate{footline}{
  \leavevmode%
  \hbox{%
  \begin{beamercolorbox}[wd=.333333\paperwidth,ht=2.25ex,dp=1ex,center]{author in head/foot}%
    \usebeamerfont{author in head/foot}\insertshortauthor
  \end{beamercolorbox}%
  \begin{beamercolorbox}[wd=.333333\paperwidth,ht=2.25ex,dp=1ex,center]{title in head/foot}%
    \usebeamerfont{title in head/foot}\insertshorttitle
  \end{beamercolorbox}%
  \begin{beamercolorbox}[wd=.333333\paperwidth,ht=2.25ex,dp=1ex,right]{date in head/foot}%
    \usebeamerfont{date in head/foot}\insertshortdate{}\hspace*{2em}
    \insertframenumber{} / \inserttotalframenumber\hspace*{2ex}
  \end{beamercolorbox}}%
  \vskip0pt%
}

% Title page info
\title[Phát Hiện Concept Drift]{Nghiên cứu và phát triển hệ thống tự động phát hiện hiện tượng trôi dạt và cập nhật mô hình học máy thích ứng}
\subtitle{Luận văn tốt nghiệp}
\author[Le Phuc Duc]{Sinh viên thực hiện: Le Phuc Duc\\ GVHD: PGS.TS. Thoại Nam}
\institute[HCMUT]{Trường Đại học Bách Khoa - ĐHQG-HCM}
\date{\today}

\begin{document}

% -----------------------------------------------------------------------------
% SLIDE 1: Title
% -----------------------------------------------------------------------------
\begin{frame}
	\titlepage
\end{frame}

% -----------------------------------------------------------------------------
% SLIDE 2: TOC
% -----------------------------------------------------------------------------
\begin{frame}{Nội dung trình bày}
	\tableofcontents
\end{frame}

% =============================================================================
% SECTION 1: Introduction
% =============================================================================
\section{Giới thiệu (Chương 0)}

\begin{frame}{Vấn đề Concept Drift \& Mục tiêu}
	\begin{columns}[T]
		\begin{column}{0.6\textwidth}
			\textbf{Thách thức thực tiễn:}
			\begin{itemize}
				\item Dữ liệu luồng (Streaming Data) thay đổi liên tục theo thời gian.
				\item \alert{Concept Drift:} $P_t(X,y) \neq P_{t+1}(X,y)$ làm mô hình lỗi thời nhanh chóng.
				\item \textbf{Yêu cầu:} Phát hiện nhanh, độ trễ thấp, không phụ thuộc nhãn (Unsupervised).
			\end{itemize}
			
			\vspace{1em}
			\textbf{Mục tiêu nghiên cứu:}
			\begin{enumerate}
				\item \textbf{Detection:} Nhanh \& Chính xác (Giảm False Positive).
				\item \textbf{Classification:} Phân loại loại drift (Sudden/Gradual...) không cần nhãn.
				\item \textbf{Adaptation:} Hệ thống tự động thích ứng với Kafka.
			\end{enumerate}
		\end{column}
		\begin{column}{0.4\textwidth}
			\centering
			\includegraphics[width=\linewidth]{image/patern_based_concept_drift.png}
			\captionof{figure}{\scriptsize Các loại Drift phổ biến}
		\end{column}
	\end{columns}
\end{frame}

% =============================================================================
% SECTION 2: Theoretical Foundation
% =============================================================================
\section{Cơ sở lý thuyết (Chương 1 \& 2)}

\begin{frame}{Định nghĩa Concept Drift}
	\begin{columns}[T]
		\begin{column}{0.6\textwidth}
			\textbf{Khái niệm:}
			\begin{itemize}
				\item Thay đổi trong mối quan hệ giữa dữ liệu đầu vào và nhãn mục tiêu theo thời gian.
				\item $\exists t: P_t(X, Y) \neq P_{t+1}(X, Y)$
			\end{itemize}
			
			\vspace{1em}
			\textbf{Phân loại theo bản chất thay đổi $P(X)$ vs $P(Y|X)$:}
			\begin{enumerate}
				\item \textbf{Real Drift:} $P(Y|X)$ thay đổi (biên quyết định thay đổi). Ảnh hưởng trực tiếp độ chính xác mô hình.
				\item \textbf{Virtual Drift (Covariate Shift):} $P(X)$ thay đổi nhưng $P(Y|X)$ giữ nguyên.
			\end{enumerate}
		\end{column}
		\begin{column}{0.4\textwidth}
			\centering
			\includegraphics[width=0.9\linewidth]{image/distribution_based_concept_drift.png}
			\captionof{figure}{\scriptsize Phân loại Drift}
		\end{column}
	\end{columns}
\end{frame}

\begin{frame}{Phân loại Drift theo Thời gian}
	\textbf{Các mẫu hình thay đổi phổ biến:}
	
	\vspace{1em}
	\begin{columns}[T]
		\begin{column}{0.5\textwidth}
			\begin{itemize}
				\item \textbf{Sudden (Đột ngột):} Thay đổi tức thời (ví dụ: cảm biến hỏng).
				\item \textbf{Gradual (Dần dần):} Giai đoạn chuyển tiếp, xen kẽ cũ/mới.
				\item \textbf{Incremental (Tăng dần):} Thay đổi liên tục, tuyến tính.
				\item \textbf{Recurrent (Lặp lại):} Quay lại trạng thái cũ (ví dụ: theo mùa).
				\item \textbf{Blip (Nhiễu):} Thay đổi ngắn hạn rồi trở lại bình thường.
			\end{itemize}
		\end{column}
		\begin{column}{0.5\textwidth}
			\centering
			\includegraphics[width=\linewidth]{image/patern_based_concept_drift.png}
			\captionof{figure}{\scriptsize Mô hình thay đổi theo thời gian}
		\end{column}
	\end{columns}
\end{frame}

\begin{frame}{Tổng quan các phương pháp hiện có}
	\begin{table}[]
		\centering
		\small
		\begin{tabular}{@{}lp{5cm}l@{}}
			\toprule
			\textbf{Nhóm phương pháp} & \textbf{Cơ chế hoạt động} & \textbf{Đại diện} \\ \midrule
			\textbf{Error Rate-based} & Theo dõi tỷ lệ lỗi mô hình (Supervised). Cần nhãn ngay lập tức. & DDM, EDDM \\ \addlinespace
			\textbf{Window-based} & So sánh thống kê giữa 2 cửa sổ trượt (Reference vs Current). & ADWIN, KSWIN \\ \addlinespace
			\textbf{Data Distribution} & So sánh phân phối dữ liệu gốc (Unsupervised). & KS-Test, \textbf{MMD} \\ \addlinespace
			\textbf{Shape-based} & Phân tích hình thái tín hiệu thay đổi ($d(P, Q)$). & \textbf{ShapeDD} \\ \bottomrule
		\end{tabular}
	\end{table}
	
	\vspace{0.5em}
	\textbf{Hạn chế chung:} Phần lớn các phương pháp Unsupervised (như MMD gốc) chạy chậm ($O(n^2)$) hoặc báo động giả cao với nhiễu.
\end{frame}

% =============================================================================
% SECTION 3: Proposed Model
% =============================================================================
\section{Mô hình đề xuất (Chương 3)}

\begin{frame}{Kiến trúc Hệ thống SE-CDT-Stream}
	\centering
	\includegraphics[width=0.9\linewidth,height=0.65\textheight,keepaspectratio]{image/system_architecture.png}
	
	\vspace{0.5em}
	\small
	\textbf{Luồng xử lý:} Dữ liệu Streaming $\to$ ShapeDD (Phát hiện) $\to$ SE-CDT (Phân loại) $\to$ Adaptation (Cập nhật Model).
\end{frame}

\begin{frame}{Đóng góp 1: Từ ShapeDD đến ShapeDD-IDW}
	\begin{columns}
		\begin{column}{0.6\textwidth}
			\textbf{Vấn đề của ShapeDD gốc:}
			\begin{itemize}
				\item Dựa vào \textbf{Permutation Test} để xác thực thống kê $\to$ Rất chậm ($O(N \cdot W^2)$).
				\item Không phù hợp cho luồng dữ liệu tốc độ cao (High-throughput Streaming).
			\end{itemize}
			
			\vspace{0.5em}
			\textbf{Giải pháp đề xuất: ShapeDD-IDW}
			\begin{itemize}
				\item \textbf{Cơ chế IDW (Inverse Density Weighting):} Trọng số nghịch đảo mật độ giúp tăng độ nhạy tại các vùng biên phân phối (nơi drift thường bắt đầu).
				\item \textbf{Asymptotic Test:} Thay thế Permutation Test bằng kiểm định tiệm cận (dựa trên Gamma distribution) $\to$ Giảm độ phức tạp tính toán.
			\end{itemize}
		\end{column}
		\begin{column}{0.4\textwidth}
			\centering
			\includegraphics[width=\linewidth]{../../results/plots/throughput_comparison.png}
			\captionof{figure}{\scriptsize Tăng tốc độ xử lý 17 lần}
		\end{column}
	\end{columns}
\end{frame}

\begin{frame}{Cải tiến 2: Phân loại Drift (SE-CDT)}
	\begin{columns}
		\begin{column}{0.6\textwidth}
			\textbf{Ý tưởng chủ đạo:}
			\begin{itemize}
				\item Phân tích hình thái của tín hiệu Drift Magnitude ($\sigma(t)$).
				\item Không cần nhãn thật (Unsupervised).
			\end{itemize}
			
			\vspace{1em}
			\begin{table}[]
				\centering \small
				\begin{tabular}{@{}ll@{}}
					\toprule
					\textbf{Đặc trưng tín hiệu} & \textbf{Loại Drift} \\ \midrule
					Đỉnh nhọn, SNR cao & \textbf{Sudden} \\
					Biên độ tăng dần, R\textsuperscript{2} cao & \textbf{Incremental} \\
					Dao động, WR rộng & \textbf{Gradual} \\
					Nhiều đỉnh đều đặn & \textbf{Recurrent} \\ 
					Đỉnh đôi sát nhau & \textbf{Blip} \\ \bottomrule
				\end{tabular}
			\end{table}
		\end{column}
		\begin{column}{0.4\textwidth}
			\centering
			% Placeholder for visually showing classifications if available
			\includegraphics[width=\linewidth]{image/vis_mixed_a_SE.png} 
			\captionof{figure}{\scriptsize Minh họa tín hiệu Drift}
		\end{column}
	\end{columns}
\end{frame}

% =============================================================================
% SECTION 4: Experiments
% =============================================================================
\section{Thực nghiệm và Đánh giá (Chương 4)}

\begin{frame}{Kết quả: Khả năng phát hiện (Detection)}
	\begin{columns}
		\begin{column}{0.5\textwidth}
			\centering
			\includegraphics[width=\linewidth]{../../results/plots/critical_difference_f1.png}
			\captionof{figure}{\scriptsize So sánh thống kê (Nemenyi Test)}
		\end{column}
		\begin{column}{0.5\textwidth}
			\textbf{Hiệu suất vượt trội:}
			\begin{itemize}
				\item \textbf{F1-Score:} 0.481 (Cao nhất nhóm Unsupervised).
				\item \textbf{False Positives:} \textbf{0} (Trên các tập Virtual Drift).
				\item ShapeDD-IDW (SE-CDT) nằm trong nhóm dẫn đầu về độ tin cậy.
			\end{itemize}
		\end{column}
	\end{columns}
\end{frame}

\begin{frame}{Kết quả: Khả năng phân loại (Classification)}
	\begin{columns}
		\begin{column}{0.5\textwidth}
			\textbf{SE-CDT vs Baseline (CDT-MSW):}
			\begin{itemize}
				\item \textbf{Accuracy:} \textbf{85.8\%} (so với 38.7\%).
				\item Phân biệt rất tốt giữa drifted (Sudden/Incremental) và non-drifted.
				\item Nhận diện chính xác 98\% các trường hợp Blip (Nhiễu ngắn hạn).
			\end{itemize}
		\end{column}
		\begin{column}{0.5\textwidth}
			\centering
			\includegraphics[width=0.9\linewidth]{../../results/plots/confusion_matrix_se_cdt.png}
			\captionof{figure}{\scriptsize Confusion Matrix (SE-CDT)}
		\end{column}
	\end{columns}
\end{frame}

\begin{frame}{Kết quả: Khả năng thích ứng (Adaptation)}
	\begin{columns}
		\begin{column}{0.65\textwidth}
			\centering
			\includegraphics[width=\linewidth]{../../results/plots/fig_prequential_stepping.png}
			\vspace{-0.5em}
			\captionof{figure}{\scriptsize Prequential Accuracy trên Stepping Drift}
		\end{column}
		\begin{column}{0.35\textwidth}
			\small
			\textbf{Phân tích:}
			\begin{itemize}
				\item Đường Accuracy hồi phục nhanh sau mỗi lần drift (nét đứt đỏ).
				\item Chiến lược \textit{Reset/Retrain} hoạt động hiệu quả khi phát hiện đúng loại drift.
			\end{itemize}
		\end{column}
	\end{columns}
\end{frame}

\begin{frame}{Triển khai thực tế trên Kafka (Deploy)}
	\centering
	\includegraphics[width=0.85\linewidth]{image/kafka_results_real.png}
	
	\vspace{0.5em}
	\small
	\textbf{Dashboard giám sát:} Hệ thống hoạt động ổn định với throughput thực tế cao, trực quan hóa trạng thái drift và accuracy theo thời gian thực.
\end{frame}

% =============================================================================
% SECTION 5: Conclusion
% =============================================================================
\section{Kết luận (Chương 5)}

\begin{frame}{Kết luận}
	\begin{block}{Đóng góp chính}
		\begin{enumerate}
			\item \textbf{Hiệu năng:} Tối ưu hóa thuật toán ShapeDD (IDW-MMD) tăng tốc 17 lần.
			\item \textbf{Phương pháp mới:} Đề xuất SE-CDT phân loại drift không giám sát (Acc 85.8\%).
			\item \textbf{Hệ thống:} Xây dựng pipeline xử lý trọn vẹn trên Apache Kafka.
		\end{enumerate}
	\end{block}
	
	\vspace{1em}
	\textbf{Hướng phát triển:}
	\begin{itemize}
		\item Cải thiện khả năng phát hiện Drift nhẹ (Mild Drift).
		\item Tích hợp cơ chế Meta-Learning để tự động chọn tham số.
	\end{itemize}
\end{frame}

\begin{frame}[plain]
	\centering
	\Huge \textcolor{univblue}{\textbf{CẢM ƠN THẦY CÔ VÀ CÁC BẠN ĐÃ LẮNG NGHE!}}
	
	\vspace{2cm}
	\normalsize
	\textbf{Q \& A}
\end{frame}

\end{document}
