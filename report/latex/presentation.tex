\documentclass[aspectratio=169,12pt]{beamer}

% Vietnamese language support
\usepackage[utf8]{inputenc}
\usepackage[T1]{fontenc}
\usepackage[vietnamese,english]{babel}
\usepackage[utf8]{vntex}

% Theme and color scheme
\usetheme{Madrid}
\usecolortheme{default}

% Additional packages
\usepackage{graphicx}
\usepackage{amsmath,amssymb}
\usepackage{algorithm}
\usepackage{algorithmic}
\usepackage{booktabs}
\usepackage{tikz}
\usepackage{pgfplots}
\pgfplotsset{compat=1.18}

% Custom colors
\definecolor{univblue}{RGB}{0,51,102}
\definecolor{univred}{RGB}{204,0,0}
\setbeamercolor{structure}{fg=univblue}
\setbeamercolor{title}{fg=white,bg=univblue}
\setbeamercolor{frametitle}{fg=white,bg=univblue}

% Remove navigation symbols
\setbeamertemplate{navigation symbols}{}

% Footer
\setbeamertemplate{footline}{
  \leavevmode%
  \hbox{%
  \begin{beamercolorbox}[wd=.333333\paperwidth,ht=2.25ex,dp=1ex,center]{author in head/foot}%
    \usebeamerfont{author in head/foot}\insertshortauthor
  \end{beamercolorbox}%
  \begin{beamercolorbox}[wd=.333333\paperwidth,ht=2.25ex,dp=1ex,center]{title in head/foot}%
    \usebeamerfont{title in head/foot}\insertshorttitle
  \end{beamercolorbox}%
  \begin{beamercolorbox}[wd=.333333\paperwidth,ht=2.25ex,dp=1ex,right]{date in head/foot}%
    \usebeamerfont{date in head/foot}\insertshortdate{}\hspace*{2em}
    \insertframenumber{} / \inserttotalframenumber\hspace*{2ex}
  \end{beamercolorbox}}%
  \vskip0pt%
}

% Title page information
\title[Phát Hiện Concept Drift]{Nghiên cứu và phát triển hệ thống tự động phát hiện hiện tượng trôi dạt và cập nhật mô hình học máy thích ứng}
\subtitle{Luận văn tốt nghiệp}
\author[Lê Phúc Đức]{%
  Sinh viên: Lê Phúc Đức\\[0.5em]
  \small Giảng viên hướng dẫn: [Tên GVHD]
}
\institute[Trường Đại học]{Trường Đại học [Tên Trường]}
\date{\today}

\begin{document}

% ==============================================================================
% TITLE SLIDE
% ==============================================================================
\begin{frame}
  \titlepage
\end{frame}

% ==============================================================================
% TABLE OF CONTENTS
% ==============================================================================
\begin{frame}{Nội dung trình bày}
  \tableofcontents
\end{frame}

% ==============================================================================
% SECTION 1: INTRODUCTION
% ==============================================================================
\section{Giới thiệu}

\begin{frame}{Bối cảnh nghiên cứu}
  \begin{columns}[T]
    \begin{column}{0.5\textwidth}
      \textbf{Thách thức:}
      \begin{itemize}
        \item Các ứng dụng học máy triển khai trong đời sống thực tế
        \item Dữ liệu thay đổi theo thời gian (non-stationary)
        \item Mô hình bị suy giảm hiệu suất
      \end{itemize}
    \end{column}
    \begin{column}{0.5\textwidth}
      \textbf{Ví dụ thực tế:}
      \begin{itemize}
        \item Phát hiện thư rác: Spammer thay đổi chiến lược
        \item Dự báo tài chính: Thị trường biến động
        \item Chẩn đoán y tế: Bệnh mới xuất hiện
      \end{itemize}
    \end{column}
  \end{columns}

  \vspace{1em}
  \begin{alertblock}{Concept Drift}
    Hiện tượng phân phối dữ liệu thay đổi theo thời gian, làm cho mô hình đã học không còn phù hợp.
  \end{alertblock}
\end{frame}

\begin{frame}{Vấn đề nghiên cứu}
  \begin{block}{Câu hỏi nghiên cứu chính}
    \begin{enumerate}
      \item Làm thế nào để \textbf{phát hiện} khi nào drift xảy ra?
      \item Làm thế nào để \textbf{phân loại} loại drift (sudden, gradual, incremental)?
      \item Làm thế nào để \textbf{thích ứng} mô hình một cách tự động?
    \end{enumerate}
  \end{block}

  \vspace{1em}
  \begin{columns}[T]
    \begin{column}{0.48\textwidth}
      \textbf{Thách thức kỹ thuật:}
      \begin{itemize}
        \item Độ chính xác phát hiện
        \item Độ trễ phát hiện thấp
        \item Chi phí tính toán hợp lý
      \end{itemize}
    \end{column}
    \begin{column}{0.48\textwidth}
      \textbf{Yêu cầu hệ thống:}
      \begin{itemize}
        \item Hoạt động real-time
        \item Tự động hóa hoàn toàn
        \item Dễ triển khai thực tế
      \end{itemize}
    \end{column}
  \end{columns}
\end{frame}

\begin{frame}{Mục tiêu nghiên cứu}
  \begin{enumerate}
    \item \textbf{Nghiên cứu lý thuyết:}
    \begin{itemize}
      \item Phân tích nền tảng toán học của Shape Drift Detector (ShapeDD)
      \item Maximum Mean Discrepancy (MMD) trong RKHS
    \end{itemize}

    \vspace{0.5em}
    \item \textbf{Xây dựng hệ thống:}
    \begin{itemize}
      \item Tích hợp ShapeDD với phân loại drift (CDT\_MSW)
      \item Phát triển chiến lược thích ứng tự động
    \end{itemize}

    \vspace{0.5em}
    \item \textbf{Đánh giá thực nghiệm:}
    \begin{itemize}
      \item So sánh với các phương pháp state-of-the-art
      \item Đo lường hiệu suất trên nhiều kịch bản drift
    \end{itemize}

    \vspace{0.5em}
    \item \textbf{Triển khai thực tế:}
    \begin{itemize}
      \item Xây dựng hệ thống end-to-end với Apache Kafka
      \item Hướng dẫn triển khai production
    \end{itemize}
  \end{enumerate}
\end{frame}

% ==============================================================================
% SECTION 2: THEORETICAL FOUNDATION
% ==============================================================================
\section{Cơ sở lý thuyết}

\begin{frame}{Các loại Concept Drift}
  \begin{figure}
    \centering
    \begin{tikzpicture}[scale=0.9]
      % Sudden drift
      \draw[thick, blue] (0,2) -- (2,2);
      \draw[thick, red] (2,2) -- (2,1) -- (4,1);
      \node[above] at (2,2.3) {\small \textbf{Sudden}};
      \draw[dashed] (2,0) -- (2,2.5);

      % Gradual drift
      \draw[thick, blue] (5,2) -- (6,2);
      \draw[thick, red] (6,2) to[out=0,in=180] (8,1);
      \node[above] at (7,2.3) {\small \textbf{Gradual}};

      % Incremental drift
      \draw[thick, blue] (9,2) -- (10,2);
      \draw[thick, red] (10,2) -- (10.5,1.7) -- (11,1.4) -- (11.5,1.1) -- (12,1);
      \node[above] at (11,2.3) {\small \textbf{Incremental}};

      % Recurrent drift
      \draw[thick, blue] (0,-0.5) -- (1,-0.5);
      \draw[thick, red] (1,-0.5) -- (2,-1.5);
      \draw[thick, blue] (2,-1.5) -- (3,-1.5);
      \draw[thick, red] (3,-1.5) -- (4,-0.5);
      \node[above] at (2,0) {\small \textbf{Recurrent}};

      % Blip
      \draw[thick, blue] (5,-0.5) -- (6,-0.5);
      \draw[thick, red] (6,-0.5) -- (6.5,-1.5) -- (7,-0.5);
      \draw[thick, blue] (7,-0.5) -- (8,-0.5);
      \node[above] at (6.5,0) {\small \textbf{Blip}};
    \end{tikzpicture}
  \end{figure}

  \begin{itemize}
    \item \textcolor{blue}{Sudden}: Thay đổi đột ngột, ngay lập tức
    \item \textcolor{blue}{Gradual}: Thay đổi dần dần qua thời gian dài
    \item \textcolor{blue}{Incremental}: Thay đổi từng bước nhỏ
    \item \textcolor{blue}{Recurrent}: Các khái niệm cũ xuất hiện lại
    \item \textcolor{blue}{Blip}: Thay đổi tạm thời rồi quay lại
  \end{itemize}
\end{frame}

\begin{frame}{Shape Drift Detector (ShapeDD)}
  \begin{block}{Ý tưởng chính}
    Sử dụng hình dạng (shape) của đường cong Maximum Mean Discrepancy (MMD) để phát hiện điểm drift.
  \end{block}

  \vspace{0.5em}
  \textbf{Quy trình phát hiện 3 giai đoạn:}
  \begin{enumerate}
    \item \textbf{Tính MMD trượt:} Tính MMD giữa hai cửa sổ liên tiếp
    \item \textbf{Phân tích shape:} Tìm các đỉnh (peaks) trong đường cong MMD
    \item \textbf{Kiểm định thống kê:} Xác nhận drift bằng permutation test
  \end{enumerate}

  \vspace{0.5em}
  \begin{columns}[T]
    \begin{column}{0.5\textwidth}
      \textbf{Ưu điểm:}
      \begin{itemize}
        \item Độ chính xác cao
        \item Độ trễ thấp
        \item Không cần nhãn
      \end{itemize}
    \end{column}
    \begin{column}{0.5\textwidth}
      \textbf{Hạn chế:}
      \begin{itemize}
        \item Chi phí tính toán cao
        \item Tham số cần tinh chỉnh
      \end{itemize}
    \end{column}
  \end{columns}
\end{frame}

\begin{frame}{Maximum Mean Discrepancy (MMD)}
  \begin{definition}[MMD]
    MMD đo khoảng cách giữa hai phân phối $P$ và $Q$ trong không gian RKHS:
    \begin{equation*}
      \text{MMD}^2(P, Q) = \left\| \mathbb{E}_{x \sim P}[\phi(x)] - \mathbb{E}_{y \sim Q}[\phi(y)] \right\|^2_{\mathcal{H}}
    \end{equation*}
  \end{definition}

  \vspace{0.5em}
  \textbf{Ước lượng thực nghiệm:}
  \begin{equation*}
    \widehat{\text{MMD}}^2 = \frac{1}{n^2} \sum_{i,j} k(x_i, x_j) + \frac{1}{m^2} \sum_{i,j} k(y_i, y_j) - \frac{2}{nm} \sum_{i,j} k(x_i, y_j)
  \end{equation*}

  \vspace{0.5em}
  \begin{alertblock}{Tính chất quan trọng}
    MMD = 0 $\Leftrightarrow$ P = Q (với kernel universal)
  \end{alertblock}
\end{frame}

\begin{frame}{Kernel Functions}
  \textbf{RBF (Gaussian) Kernel:} (Sử dụng trong nghiên cứu)
  \begin{equation*}
    k(x, y) = \exp\left(-\frac{\|x - y\|^2}{2\sigma^2}\right)
  \end{equation*}

  \vspace{1em}
  \begin{columns}[T]
    \begin{column}{0.5\textwidth}
      \textbf{Lựa chọn tham số $\gamma = 1/(2\sigma^2)$:}
      \begin{itemize}
        \item \textbf{Scott's rule:} Thích nghi với số chiều
        \item \textbf{Median heuristic:} Bền vững với outliers
      \end{itemize}
    \end{column}
    \begin{column}{0.5\textwidth}
      \textbf{Ảnh hưởng của $\gamma$:}
      \begin{itemize}
        \item $\gamma$ nhỏ: Kernel mượt, tổng quát hóa tốt
        \item $\gamma$ lớn: Kernel sắc, nhạy với chi tiết
      \end{itemize}
    \end{column}
  \end{columns}
\end{frame}


% Adaptive gamma
\begin{frame}{1. Adaptive Kernel Bandwidth (Gamma)}
\textbf{Áp dụng:} \texttt{shape\_adaptive}, \texttt{shape\_adaptive\_v2}, \texttt{shape\_sensitive}, \texttt{shape\_gradual\_aware}

\vspace{6pt}
\textbf{Giải thích lý thuyết mở rộng:}
\begin{itemize}
  \item \textbf{Vai trò của kernel bandwidth:} Trong MMD với RBF kernel $k(x,y)=\exp(-\gamma\|x-y\|^2)$, tham số $\gamma$ điều khiển độ nhạy theo khoảng cách. $\gamma$ quá nhỏ (kernel rộng) làm mờ sự khác biệt giữa hai phân phối; $\gamma$ quá lớn (kernel hẹp) gây nhiễu do chỉ so sánh điểm gần.
  \item \textbf{Heuristics toán học:} median heuristic (dựa trên khoảng cách trung vị) và Scott's rule (dựa trên độ lệch chuẩn và chiều dữ liệu) xuất phát từ lý thuyết ước lượng mật độ kernel — giúp cân bằng bias/variance trong ước lượng MMD.
  \item \textbf{Hệ quả cho kiểm định:} Power của test MMD phụ thuộc mạnh vào lựa chọn $\gamma$. Việc chọn $\gamma$ thích nghi làm tăng khả năng phát hiện drift ở nhiều tình huống (thay vì một $\gamma$ cố định chỉ tốt cho một loại drift).
\end{itemize}
\end{frame}

% Smoothing
\begin{frame}{2. Smoothing nhẹ để giảm nhiễu}
\textbf{Áp dụng:} smoothing window nhỏ trong các phiên bản adaptive

\vspace{6pt}
\textbf{Giải thích lý thuyết mở rộng:}
\begin{itemize}
  \item \textbf{Mục tiêu:} loại bỏ nhiễu tần số cao (high-frequency noise) mà không làm mờ cạnh (edges) tương ứng với drift abrupt.
  \item \textbf{Tại sao window nhỏ?} Smoothing linear (ví dụ uniform filter) có đặc tính low-pass. Cửa sổ lớn cắt bỏ tín hiệu có băng thông tương ứng (các chuyển đổi ngắn), gây mất detection cho abrupt drift. Cửa sổ nhỏ (size=3) giảm variance nhưng giữ lại đặc trưng thời gian.
  \item \textbf{Lý thuyết liên quan:} trade-off bias–variance trong xử lý tín hiệu: smoothing giảm variance (ít false alarms), nhưng tăng bias (có thể giảm recall). Việc chọn cửa sổ nhỏ là một compromise hữu ích khi cần giữ độ phân giải thời gian.
\end{itemize}
\end{frame}

% Percentile thresholds
\begin{frame}{3. Ngưỡng dựa trên Percentile thay vì mean $\pm$ kσ}
\textbf{Áp dụng:} percentile-based threshold trong \texttt{shape\_adaptive\_v2}, \texttt{shape\_gradual\_aware}

\vspace{6pt}
\textbf{Giải thích lý thuyết mở rộng:}
\begin{itemize}
  \item \textbf{Vấn đề với mean $\pm$ kσ:} Khi phân phối của thống kê (shape) lệch ( skewed ) hoặc nặng đuôi (heavy-tailed), mean và std không đại diện cho baseline — dẫn đến threshold không phù hợp.
  \item \textbf{Percentile-based:} chọn percentile (ví dụ 10th) của các giá trị shape dương làm baseline — tiếp cận nonparametric, robust với ngoại lệ.
  \item \textbf{Hệ quả thống kê:} percentile phản ánh vị trí tương đối trong phân phối thống kê, nên thích hợp cho môi trường phi tĩnh hoặc khi magnitude drift thay đổi theo thời gian.
\end{itemize}
\end{frame}

% Sensitivity modes
\begin{frame}{4. Chế độ Sensitivity (low, medium, high, ultrahigh)}
\textbf{Áp dụng:} các hàm adaptive

\vspace{6pt}
\textbf{Giải thích lý thuyết mở rộng:}
\begin{itemize}
  \item \textbf{Cơ sở:} trong Signal Detection Theory, lựa chọn ngưỡng trực tiếp điều khiển trade-off giữa \emph{False Positive Rate} (α) và \emph{Detection Power} (1−β). Sensitivity mode là một cách thực tế để bật/tắt mức độ bảo thủ.
  \item \textbf{Ý nghĩa toán học:} thay đổi sensitivity tương đương thay đổi $\eta$ trong quy tắc likelihood ratio: $\Lambda(x)\gtrless \eta$ → thay đổi $\eta$ thay đổi ROC point (operating point).
  \item \textbf{Ứng dụng:} chọn 'high' để tối đa recall khi drift nhỏ; chọn 'low' để tối ưu precision khi dữ liệu nhiều nhiễu.
\end{itemize}
\end{frame}

% MMD + FDR
\begin{frame}{5. MMD kết hợp FDR (Benjamini–Hochberg)}
\textbf{Áp dụng:} kiểm định MMD + điều chỉnh nhiều kiểm định

\vspace{6pt}
\textbf{Giải thích lý thuyết mở rộng:}
\begin{itemize}
  \item \textbf{MMD permutation test:} trả về p-value cho mỗi candidate — liệu hai cửa sổ dữ liệu trước/sau có khác nhau về phân phối không.
  \item \textbf{Vấn đề multiple-testing:} trong chuỗi lớn, ta thực hiện nhiều phép kiểm định → tỉ lệ false positive tích lũy. BH-FDR điều chỉnh để kiểm soát kỳ vọng tỉ lệ false discoveries (FDR).
  \item \textbf{Mối quan hệ với Neyman–Pearson:} MMD cung cấp kiểm định cho mỗi vị trí (kiểm định đơn lẻ). BH-FDR là bước bổ sung để duy trì kiểm soát lỗi khi kết hợp nhiều quyết định.
\end{itemize}
\end{frame}

% Skip FDR when dense
\begin{frame}{6. Bỏ FDR khi mật độ drift lớn (dense drift)}
\textbf{Áp dụng:} logic skip-FDR trong \texttt{shape\_adaptive\_v2}

\vspace{6pt}
\textbf{Giải thích lý thuyết mở rộng:}
\begin{itemize}
  \item \textbf{Giả định FDR:} BH-FDR giả định phần lớn giả thuyết H0 là đúng (sparsity). Nếu giả định này bị vi phạm (nhiều H1), FDR sẽ trở nên quá bảo thủ và làm mất power.
  \item \textbf{Hệ quả thực nghiệm:} trong multi-drift streams, FDR có thể loại bỏ phần lớn các phát hiện thật → giảm recall mạnh.
  \item \textbf{Chiến lược của bạn:} kiểm tra \emph{detection density} và chỉ áp dụng FDR khi density thấp (sparse). Đây là một áp dụng thực tế của lý thuyết multiple testing khi giả định sparsity bị nghi ngờ.
\end{itemize}
\end{frame}

% Detect plateau
\begin{frame}{7. Nhận diện Plateau cho Gradual Drift}
\textbf{Áp dụng:} \texttt{shape\_gradual\_aware} — phát hiện plateau bằng đạo hàm và curvature

\vspace{6pt}
\textbf{Giải thích lý thuyết mở rộng:}
\begin{itemize}
  \item \textbf{Đặc trưng của gradual drift:} khi drift xảy ra chậm, các thống kê hai-mẫu (MMD) sẽ tăng và giữ ở mức cao trong một khoảng rộng — tạo plateau hơn là đỉnh nhọn.
  \item \textbf{Tiêu chí hình học:}
    \begin{itemize}
      \item Slope gần 0: $|f'(t)| < \varepsilon$ (không tăng/giảm rõ rệt).
      \item Curvature nhỏ: $|f''(t)| < \delta$ (không có turn sắc).
      \item Elevation: $f(t)$ vượt baseline → không phải plateau do noise.
      \item Width threshold: plateau phải đủ rộng để loại bỏ nhiễu ngẫu nhiên.
    \end{itemize}
  \item \textbf{Cơ sở toán học:} curvature $\kappa = \frac{|f''|}{(1 + f'^2)^{3/2}}$ — dùng để phân biệt peak (cao $\kappa$) và plateau (thấp $\kappa$).
  \item \textbf{Ý nghĩa thực tế:} bổ sung này khắc phục hạn chế cơ bản của zero-crossing peak detector, làm tăng khả năng phát hiện drift từ từ.
\end{itemize}
\end{frame}

% Multiscale
\begin{frame}{8. Multi-scale Detection (Wavelet-like)}
\textbf{Áp dụng:} \texttt{shape\_multiscale}

\vspace{6pt}
\textbf{Giải thích lý thuyết mở rộng:}
\begin{itemize}
  \item \textbf{Vấn đề:} một kích thước cửa sổ cố định không bắt được mọi tốc độ drift — drift nhanh cần window nhỏ, drift chậm cần window lớn.
  \item \textbf{Multi-scale idea:} phân tích ở nhiều scale (ví dụ $l_1\in\{25,50,100,200\}$) tương tự phân tích đa phân giải (wavelet), cho phép phát hiện các đặc trưng thời gian khác nhau.
  \item \textbf{Lý thuyết liên quan:} transform đa độ phân giải giúp cô lập tín hiệu ở các băng tần thời gian khác nhau; hợp nhất kết quả theo một quy tắc (voting, OR-rule) cho detector robust.
\end{itemize}
\end{frame}

% SNR adaptive
\begin{frame}{9. SNR-based Strategy Selection}
\textbf{Áp dụng:} \texttt{shape\_snr\_adaptive} và \texttt{shape\_snr\_adaptive\_v2}

\vspace{6pt}
\textbf{Giải thích lý thuyết mở rộng:}
\begin{itemize}
  \item \textbf{Định nghĩa SNR ở đây:} tỷ số giữa biến thiên \emph{signal} (magnitude của phân phối trái–phải) và biến thiên \emph{noise} (biến đổi ngẫu nhiên trong vùng không có drift).
  \item \textbf{Lý thuyết:} theo Neyman–Pearson, ngưỡng tối ưu tỉ lệ nghịch với SNR: SNR cao cho phép hạ ngưỡng để tăng power mà vẫn giữ false alarm thấp; SNR thấp đòi hỏi ngưỡng cao để tránh false alarms.
  \item \textbf{Chiến lược thực thi:} ước lượng SNR từ dữ liệu (estimate\_snr / robust bootstrap) → chọn giữa conservative (shape gốc) và aggressive (adaptive\_v2).
  \item \textbf{Ưu điểm:} giảm lỗi mô hình cố định; detector tự thích nghi với môi trường dữ liệu.
\end{itemize}
\end{frame}

% Bootstrap + hybrid ensemble
\begin{frame}{10. Bootstrap SNR + Confidence-weighted Hybrid}
\textbf{Áp dụng:} \texttt{shape\_snr\_adaptive\_v2}

\vspace{6pt}
\textbf{Giải thích lý thuyết mở rộng:}
\begin{itemize}
  \item \textbf{Bootstrap:} dùng bootstrap để thu được phân bố ước lượng SNR, xử lý uncertainty trong ước lượng (Efron \& Tibshirani).
  \item \textbf{Confidence metric:} xác suất $P(\text{SNR} > \text{threshold} \mid \text{bootstrap})$ dùng để quyết định chiến lược:
    \begin{itemize}
      \item High confidence $\Rightarrow$ aggressive;
      \item Low confidence $\Rightarrow$ conservative;
      \item Trung gian $\Rightarrow$ hybrid ensemble.
    \end{itemize}
  \item \textbf{Hybrid ensemble:} kết hợp hai detector với trọng số theo confidence. P-value kết hợp theo OR-rule (min p-value) để ưu tiên sensitivity.
  \item \textbf{Lý do hiệu quả:} ensemble giảm rủi ro chọn sai mô hình do variance cao trong ước lượng SNR; confidence-weighting cho phép chuyển tiếp mượt giữa chiến lược, tránh chuyển đổi "đột ngột".
\end{itemize}
\end{frame}

\section{Hệ thống đề xuất}
\begin{frame}{Kiến trúc tổng thể}
  \begin{figure}
    \centering
    \begin{tikzpicture}[
      node distance=1.5cm,
      block/.style={rectangle, draw, fill=blue!20, text width=3cm, text centered, rounded corners, minimum height=1cm},
      arrow/.style={thick,->,>=stealth}
    ]
      \node[block] (input) {Data Stream};
      \node[block, right of=input, xshift=2cm] (detect) {ShapeDD\\Drift Detection};
      \node[block, right of=detect, xshift=2cm] (classify) {CDT\_MSW\\Drift Classification};
      \node[block, below of=classify, yshift=-0.5cm] (adapt) {Adaptive Strategy};
      \node[block, below of=input, yshift=-0.5cm] (model) {ML Model};

      \draw[arrow] (input) -- (detect);
      \draw[arrow] (detect) -- node[above] {\small Drift?} (classify);
      \draw[arrow] (classify) -- node[right] {\small Type} (adapt);
      \draw[arrow] (adapt) -| (model);
      \draw[arrow] (input) |- (model);
      \draw[arrow] (model.east) -- ++(1,0) |- (detect.south);
    \end{tikzpicture}
  \end{figure}

  \textbf{Luồng hoạt động:}
  \begin{enumerate}
    \item Luồng dữ liệu được giám sát liên tục bởi ShapeDD
    \item Khi phát hiện drift → Kích hoạt phân loại (CDT\_MSW)
    \item Dựa vào loại drift → Áp dụng chiến lược thích ứng phù hợp
    \item Cập nhật mô hình và tiếp tục giám sát
  \end{enumerate}
\end{frame}

\begin{frame}{Thuật toán ShapeDD}
  \begin{algorithmic}[1]
    \REQUIRE Data stream $X$, window sizes $l_1, l_2$, permutations $n_{perm}$
    \ENSURE Drift positions and p-values
    \STATE Compute kernel matrix $K$ using RBF kernel
    \STATE Compute sliding MMD statistics using convolution
    \STATE Apply smoothing to reduce noise
    \STATE Compute shape statistics (second derivative)
    \STATE Identify candidate peaks (zero-crossings)
    \FOR{each candidate peak $pos$}
      \STATE Extract window around $pos$ with size $l_2$
      \STATE Run permutation test to compute p-value
      \IF{p-value $< 0.05$}
        \STATE \textbf{Report drift at position} $pos$
      \ENDIF
    \ENDFOR
  \end{algorithmic}

  \vspace{0.5em}
  \begin{block}{Cải tiến trong nghiên cứu}
    \textbf{ShapeDD\_Adaptive\_v2:} Ngưỡng thích ứng + FDR correction có điều kiện
  \end{block}
\end{frame}

\begin{frame}{Drift Classification (CDT\_MSW)}
  \textbf{Phương pháp Multiple Sliding Windows:}
  \begin{itemize}
    \item Sử dụng 3 cửa sổ khác nhau: nhỏ, trung bình, lớn
    \item So sánh tỷ lệ phát hiện giữa các cửa sổ
    \item Phân loại dựa trên pattern của detections
  \end{itemize}

  \vspace{1em}
  \begin{table}
    \centering
    \small
    \begin{tabular}{lccc}
      \toprule
      \textbf{Loại Drift} & \textbf{Window S} & \textbf{Window M} & \textbf{Window L} \\
      \midrule
      Sudden      & High & High & High \\
      Gradual     & Low  & Med  & High \\
      Incremental & Med  & Med  & High \\
      Recurrent   & High & Med  & Low  \\
      Blip        & High & Low  & Low  \\
      \bottomrule
    \end{tabular}
  \end{table}
\end{frame}

\begin{frame}{Chiến lược thích ứng}
  \textbf{Dựa vào loại drift được phân loại:}

  \begin{itemize}
    \item \textbf{Sudden Drift:}
    \begin{itemize}
      \item \alert{Reset model} hoàn toàn
      \item Huấn luyện lại trên dữ liệu mới
      \item Kích thước cửa sổ: 500 samples
    \end{itemize}

    \vspace{0.3em}
    \item \textbf{Gradual/Incremental:}
    \begin{itemize}
      \item \alert{Incremental learning} (cập nhật dần dần)
      \item Learning rate thấp để tránh catastrophic forgetting
      \item Cửa sổ lớn hơn: 1000 samples
    \end{itemize}

    \vspace{0.3em}
    \item \textbf{Recurrent:}
    \begin{itemize}
      \item \alert{Ensemble models} cho các concept đã thấy
      \item Lưu trữ models cũ trong repository
      \item Reactive reactivation khi pattern quay lại
    \end{itemize}

    \vspace{0.3em}
    \item \textbf{Blip:}
    \begin{itemize}
      \item \alert{Ignore} (không cập nhật)
      \item Chờ xác nhận drift thực sự
    \end{itemize}
  \end{itemize}
\end{frame}

\begin{frame}{Triển khai với Apache Kafka}
  \begin{columns}[T]
    \begin{column}{0.5\textwidth}
      \textbf{Kiến trúc streaming:}
      \begin{itemize}
        \item \textbf{Producer:} Sinh dữ liệu liên tục
        \item \textbf{Kafka:} Message broker
        \item \textbf{Consumer:} Phát hiện drift
        \item \textbf{Trainer:} Cập nhật model
      \end{itemize}

      \vspace{1em}
      \textbf{Topics:}
      \begin{itemize}
        \item \texttt{data-stream}
        \item \texttt{drift-alerts}
        \item \texttt{model-updates}
      \end{itemize}
    \end{column}
    \begin{column}{0.5\textwidth}
      \begin{figure}
        \centering
        \begin{tikzpicture}[scale=0.7,
          block/.style={rectangle, draw, fill=blue!20, text width=2cm, text centered, rounded corners, minimum height=0.8cm, font=\tiny},
          arrow/.style={->,>=stealth}
        ]
          \node[block] (producer) {Producer};
          \node[block, below of=producer, yshift=-0.5cm] (kafka) {Kafka};
          \node[block, below of=kafka, yshift=-0.5cm] (consumer) {Consumer};
          \node[block, below of=consumer, yshift=-0.5cm] (detector) {Detector};
          \node[block, below of=detector, yshift=-0.5cm] (trainer) {Trainer};

          \draw[arrow] (producer) -- (kafka);
          \draw[arrow] (kafka) -- (consumer);
          \draw[arrow] (consumer) -- (detector);
          \draw[arrow] (detector) -- (trainer);
          \draw[arrow] (trainer.west) -- ++(-0.5,0) |- (consumer.west);
        \end{tikzpicture}
      \end{figure}
    \end{column}
  \end{columns}
\end{frame}

% ==============================================================================
% SECTION 4: EXPERIMENTS
% ==============================================================================
\section{Thực nghiệm và đánh giá}

\begin{frame}{Thiết kế thực nghiệm}
  \textbf{Datasets:} 8 tập dữ liệu tổng hợp
  \begin{itemize}
    \item SEA (standard, enhanced)
    \item STAGGER
    \item Hyperplane (rotating)
    \item gen\_random (4 mức độ: mild, moderate, severe, ultra-severe)
  \end{itemize}

  \vspace{0.5em}
  \textbf{Cấu hình:}
  \begin{itemize}
    \item Stream size: 10,000 samples
    \item Number of drifts: 10 per stream
    \item Window size: $l_1 = 50$, $l_2 = 150$
    \item Permutations: 2,500
  \end{itemize}

  \vspace{0.5em}
  \textbf{Phương pháp so sánh:}
  \begin{itemize}
    \item Window-based: D3, DAWIDD, MMD, KS
    \item Streaming: ADWIN, DDM, EDDM, HDDM\_A, HDDM\_W, FHDDM
  \end{itemize}
\end{frame}

\begin{frame}{Metrics đánh giá}
  \begin{columns}[T]
    \begin{column}{0.5\textwidth}
      \textbf{Detection metrics:}
      \begin{itemize}
        \item \textbf{F1-Score:} Cân bằng precision-recall
        \item \textbf{Detection Rate:} Tỷ lệ drift được phát hiện
        \item \textbf{MTTD:} Mean Time To Detect (độ trễ)
      \end{itemize}

      \vspace{1em}
      \textbf{Adaptation metrics:}
      \begin{itemize}
        \item \textbf{Accuracy:} Độ chính xác phân loại
        \item \textbf{Recovery rate:} Tốc độ phục hồi
        \item \textbf{Degradation:} Mức độ suy giảm
      \end{itemize}
    \end{column}
    \begin{column}{0.5\textwidth}
      \textbf{Công thức tính:}
      \begin{align*}
        \text{Precision} &= \frac{TP}{TP + FP} \\[0.5em]
        \text{Recall} &= \frac{TP}{TP + FN} \\[0.5em]
        \text{F1} &= \frac{2 \times P \times R}{P + R} \\[0.5em]
        \text{MTTD} &= \frac{1}{n} \sum_{i=1}^{n} |t_i^{detect} - t_i^{true}|
      \end{align*}
    \end{column}
  \end{columns}
\end{frame}

\begin{frame}{Kết quả: Detection Performance}
  \begin{table}
    \centering
    \footnotesize
    \begin{tabular}{lccc}
      \toprule
      \textbf{Method} & \textbf{F1-Score} & \textbf{Detection Rate} & \textbf{MTTD} \\
      \midrule
      \textbf{ShapeDD (Original)} & \textbf{0.758} & 78.8\% & \textbf{71.9} \\
      ShapeDD\_Adaptive\_v2\_High & 0.730 & 72.5\% & 100.0 \\
      ShapeDD\_Adaptive\_None & 0.665 & 75.0\% & 118.0 \\
      DAWIDD & 0.657 & 87.5\% & 80.0 \\
      ADWIN & 0.632 & 66.7\% & 129.4 \\
      MMD & 0.598 & 82.5\% & 88.1 \\
      EDDM & 0.465 & 65.0\% & 182.2 \\
      \bottomrule
    \end{tabular}
  \end{table}

  \vspace{0.5em}
  \begin{block}{Kết luận}
    \begin{itemize}
      \item ShapeDD đạt F1 cao nhất (0.758) và MTTD thấp nhất (71.9 samples)
      \item ShapeDD\_v2 cải thiện recall nhưng trade-off với precision
      \item Window-based methods tốt hơn streaming methods về accuracy
    \end{itemize}
  \end{block}
\end{frame}

\begin{frame}{Kết quả: Adaptation Performance}
  \textbf{Thực nghiệm trên SEA dataset:}

  \begin{table}
    \centering
    \small
    \begin{tabular}{lc}
      \toprule
      \textbf{Metric} & \textbf{Giá trị} \\
      \midrule
      Pre-drift accuracy & 99.6\% \\
      Accuracy degradation & 55.0\% (suy giảm 44.6\%) \\
      Post-adaptation accuracy & 91.9\% \\
      Recovery rate & \textbf{82.8\%} \\
      Detection delay (ShapeDD) & 4 samples \\
      Adaptation time & 96 samples \\
      \bottomrule
    \end{tabular}
  \end{table}

  \vspace{0.5em}
  \begin{alertblock}{Nhận xét}
    Hệ thống phục hồi được 82.8\% hiệu suất ban đầu sau khi xảy ra drift, chứng tỏ chiến lược thích ứng hiệu quả.
  \end{alertblock}
\end{frame}

\begin{frame}{So sánh trên các dataset}
  \begin{figure}
    \centering
    \begin{tikzpicture}
      \begin{axis}[
        ybar,
        width=11cm,
        height=6cm,
        ylabel={F1-Score},
        symbolic x coords={SEA, STAGGER, Hyperplane, gen\_mild, gen\_moderate, gen\_severe},
        xtick=data,
        x tick label style={font=\tiny, rotate=45, anchor=east},
        legend style={at={(0.5,-0.25)}, anchor=north, legend columns=3, font=\tiny},
        ymin=0, ymax=1,
        bar width=3pt
      ]
        \addplot coordinates {(SEA,0.83) (STAGGER,0.91) (Hyperplane,0.60) (gen\_mild,0.65) (gen\_moderate,0.76) (gen\_severe,0.86)};
        \addplot coordinates {(SEA,0.72) (STAGGER,0.70) (Hyperplane,0.55) (gen\_mild,0.60) (gen\_moderate,0.68) (gen\_severe,0.75)};
        \addplot coordinates {(SEA,0.78) (STAGGER,0.65) (Hyperplane,0.82) (gen\_mild,0.55) (gen\_moderate,0.60) (gen\_severe,0.70)};

        \legend{ShapeDD, ShapeDD\_v2, ADWIN}
      \end{axis}
    \end{tikzpicture}
  \end{figure}

  \begin{itemize}
    \item ShapeDD tốt nhất trên: SEA (0.83), STAGGER (0.91), gen\_severe (0.86)
    \item ADWIN tốt hơn trên Hyperplane (0.82) - gradual drift
  \end{itemize}
\end{frame}

\begin{frame}{Phân tích cải tiến ShapeDD\_v2}
  \textbf{Các thay đổi chính trong v2:}

  \begin{enumerate}
    \item \textbf{Percentile thấp hơn:} 10th thay vì 20th → Tăng recall
    \item \textbf{Multiplier tích cực hơn:} 0.5 thay vì 0.8 → Ngưỡng thấp hơn
    \item \textbf{Absolute minimum floor:} Ngăn lọc drift yếu
    \item \textbf{FDR có điều kiện:} Chỉ áp dụng khi detection density < 3\%
  \end{enumerate}

  \vspace{0.5em}
  \begin{block}{Kết quả cải tiến}
    \begin{itemize}
      \item Recall tăng: 60\% → 75\% (+15\%)
      \item Precision giảm nhẹ: 80\% → 73\% (-7\%)
      \item F1 tăng: 0.665 → 0.730 (+10\%)
    \end{itemize}
  \end{block}

  \vspace{0.5em}
  \textbf{Trade-off:} Ưu tiên recall (không bỏ sót drift) hơn precision (ít false alarm hơn)
\end{frame}

% ==============================================================================
% SECTION 5: SOTA LANDSCAPE 2024-2025
% ==============================================================================
\section{Bối cảnh SOTA 2024-2025}

\begin{frame}{SOTA Methods (2024-2025)}
  \begin{table}
    \centering
    \footnotesize
    \begin{tabular}{llccc}
      \toprule
      \textbf{Method} & \textbf{Type} & \textbf{F1} & \textbf{Labels} & \textbf{Year} \\
      \midrule
      \textbf{CDSeer} & Semi-supervised & \textbf{0.86} & \textbf{1\%} & 2024 \\
      \textbf{DriftLens} & Unsupervised & 15/17 wins & 0\% & 2024 \\
      \textbf{CV4CDD-4D} & Supervised & 0.81-0.83 & 100\% & 2025 \\
      \textbf{ADA-ADF} & Time series & \textbf{0.92} & 0\% & 2025 \\
      \textbf{ShapeDD} & Unsupervised & 0.758 & 0\% & 2024 \\
      ADWIN & Unsupervised & 0.507 & 0\% & 2007 \\
      HDDM-W & Model-dependent & 0.80 & Yes & 2016 \\
      \bottomrule
    \end{tabular}
  \end{table}
\end{frame}

\begin{frame}{CDSeer: Semi-Supervised SOTA (Oct 2024)}
  \begin{columns}[T]
    \begin{column}{0.48\textwidth}
      \textbf{Kết quả ấn tượng:}
      \begin{itemize}
        \item \textcolor{red}{\textbf{57.1\% precision improvement}}
        \item \textcolor{red}{\textbf{99\% fewer labels}} (1\% vs 100\%)
        \item F1-score: \textbf{0.86}
        \item Validated at \textbf{Ericsson}
      \end{itemize}

      \vspace{0.5em}
      \textbf{Technical approach:}
      \begin{itemize}
        \item Confidence-based active learning
        \item Model-agnostic
        \item Distribution-agnostic
      \end{itemize}
    \end{column}
  \end{columns}

  \vspace{0.3em}
  \textbf{Citation:} arXiv 2410.09190 (Oct 2024, updated Aug 2025)
\end{frame}

\begin{frame}{DriftLens: Real-Time Unsupervised (June 2024)}
  \begin{columns}[T]
    \begin{column}{0.48\textwidth}
      \textbf{Performance:}
      \begin{itemize}
        \item \textbf{15/17 wins} (88\% win rate)
        \item \textbf{5× faster} than competitors
        \item \textbf{Correlation ≥ 0.85} with true drift
        \item Fully unsupervised
      \end{itemize}

      \vspace{0.5em}
      \textbf{Key innovations:}
      \begin{itemize}
        \item Deep learning embeddings
        \item Per-label distribution tracking
        \item Built-in explainability
        \item Open-source (PyPI)
      \end{itemize}
    \end{column}
    \begin{column}{0.48\textwidth}
      \textbf{Architecture:}
      \begin{enumerate}
        \item Offline: Estimate reference distributions
        \item Online: Process windows
        \item Monitor: Distribution distances
        \item Explain: Representative samples
      \end{enumerate}

      \vspace{0.5em}
      \textbf{Ưu điểm cho production:}
      \begin{itemize}
        \item Real-time capability
        \item Low complexity
        \item Works on unstructured data
        \item Drift characterization
      \end{itemize}
    \end{column}
  \end{columns}

  \vspace{0.3em}
  \textbf{Citation:} arXiv 2406.17813
\end{frame}

\begin{frame}{Other Notable SOTA Methods}
  \begin{table}
    \centering
    \tiny
    \begin{tabular}{llp{3cm}c}
      \toprule
      \textbf{Method} & \textbf{Specialty} & \textbf{Key Innovation} & \textbf{Performance} \\
      \midrule
      \textbf{Meta-ADD} & Meta-learning & Automatic detector selection & Pre-trained classifier \\
      \textbf{CDDRM} & Explainability & Feature-level drift + causal & F1 = 0.895 (Stagger) \\
      \textbf{AEF-CDA} & Ensemble & Medical IoT streams & Acc = 99.64\% \\
      \textbf{Transformer} & Time series & Temporal attention & Few-shot learning \\
      \textbf{ARF} & Online ensemble & Per-tree ADWIN & Production standard \\
      \textbf{HAT} & Adaptive trees & Built-in drift handling & Multi-label capable \\
      \bottomrule
    \end{tabular}
  \end{table}

  \vspace{0.5em}
  \begin{block}{Emerging Paradigms (2024)}
    \begin{itemize}
      \item \textbf{Test-Time Adaptation (TTA):} Adapt without source data
      \item \textbf{Continual Learning:} Avoid catastrophic forgetting
      \item \textbf{OOD Detection:} Separate drift from outliers
      \item \textbf{Bayesian UQ:} Epistemic vs aleatoric uncertainty
    \end{itemize}
  \end{block}
\end{frame}

% ==============================================================================
% SECTION 6: SYSTEM ARCHITECTURES
% ==============================================================================
\section{Kiến trúc hệ thống}

\begin{frame}{4 Architecture Patterns Chính}
  \begin{enumerate}
    \item \textbf{Model-Agnostic (65\% research - Chúng ta đang dùng)}
    \begin{itemize}
      \item Detector monitors: Feature distributions (X)
      \item Model: Bất kỳ (LogReg, RF, NN)
      \item Example: ShapeDD + LogisticRegression
      \item ✅ Flexible, ❌ May miss performance-relevant drift
    \end{itemize}

    \vspace{0.3em}
    \item \textbf{Model-Dependent (30\% research)}
    \begin{itemize}
      \item Detector monitors: Model errors/confidence
      \item Model: Coupled with detector
      \item Example: DDM, EDDM, ADWIN, CDSeer
      \item ✅ Detects relevant drift, ❌ Requires labels
    \end{itemize}

    \vspace{0.3em}
    \item \textbf{Integrated Model-Detector (20\%, growing)}
    \begin{itemize}
      \item Model có built-in detection
      \item Example: Adaptive Random Forest, HAT
      \item ✅ Fully automated, ❌ Model-specific
    \end{itemize}

    \vspace{0.3em}
    \item \textbf{Production MLOps (Industry standard)}
    \begin{itemize}
      \item Complete pipeline: Kafka + Flink + Monitoring
      \item ✅ Scalable, enterprise-ready
    \end{itemize}
  \end{enumerate}
\end{frame}

\begin{frame}{Kiến trúc hiện tại của chúng ta}
  \begin{figure}
    \centering
    \begin{tikzpicture}[
      box/.style={rectangle, draw=univblue, thick, fill=blue!10, minimum width=3cm, minimum height=0.8cm, align=center},
      arrow/.style={->, thick, >=stealth}
    ]
      % Data stream
      \node[box] (data) at (0,3) {Data Stream X};

      % Component 1: Model
      \node[box, fill=green!10] (model) at (0,1.5) {Component 1:\\LogisticRegression\\(FROZEN)};

      % Predictions
      \node[box, fill=yellow!10] (pred) at (0,0) {Predictions ŷ};

      % Component 2: Detector (parallel path)
      \node[box, fill=orange!10] (detector) at (5,1.5) {Component 2:\\ShapeDD SNR-Adaptive\\(Monitors X)};

      % Component 3: Adaptation
      \node[box, fill=red!10] (adapt) at (5,0) {Component 3:\\Retrain Strategy};

      % Arrows
      \draw[arrow] (data) -- (model);
      \draw[arrow] (model) -- (pred);
      \draw[arrow, dashed] (data) -- (detector);
      \draw[arrow] (detector) -- node[right] {Drift?} (adapt);
      \draw[arrow, dashed] (adapt) to[bend right=45] node[below] {Retrain} (model);
    \end{tikzpicture}
  \end{figure}

  \begin{alertblock}{Pattern: Model-Agnostic ✅}
    Detector và Model \textbf{độc lập} - có thể swap model mà không thay detector
  \end{alertblock}
\end{frame}

\begin{frame}{Production MLOps Architecture (Industry 2024)}
  \begin{figure}
    \centering
    \tiny
    \begin{tikzpicture}[
      box/.style={rectangle, draw, thick, minimum width=2cm, minimum height=0.6cm, align=center, font=\tiny},
      arrow/.style={->, thick, >=stealth}
    ]
      % Layers
      \node[box, fill=blue!20] (kafka) at (0,3) {Kafka\\Ingestion};
      \node[box, fill=green!20] (flink) at (0,2) {Flink\\Processing};
      \node[box, fill=yellow!20] (model) at (0,1) {Model\\Inference};
      \node[box, fill=orange!20] (detector) at (3,2) {Drift\\Detection};
      \node[box, fill=red!20] (monitor) at (3,1) {Performance\\Monitor};
      \node[box, fill=purple!20] (retrain) at (6,1.5) {Auto\\Retrain};
      \node[box, fill=cyan!20] (dash) at (6,0.5) {Grafana\\Dashboard};

      % Arrows
      \draw[arrow] (kafka) -- (flink);
      \draw[arrow] (flink) -- (model);
      \draw[arrow] (flink) -- (detector);
      \draw[arrow] (model) -- (monitor);
      \draw[arrow] (detector) -- (retrain);
      \draw[arrow] (monitor) -- (retrain);
      \draw[arrow] (retrain) -- (dash);
      \draw[arrow, dashed] (retrain) to[bend left=30] node[right, font=\tiny] {Deploy} (model);
    \end{tikzpicture}
  \end{figure}

  \vspace{0.3em}
  \textbf{Components:}
  \begin{itemize}
    \item \textbf{Kafka:} Stream ingestion (millions events/sec)
    \item \textbf{Flink:} Real-time processing (5× faster than Spark)
    \item \textbf{3-layer detection:} Data drift + Concept drift + Performance
    \item \textbf{Automated workflow:} Detect → Alert → Retrain → A/B test → Deploy
  \end{itemize}
\end{frame}

\begin{frame}{Streaming Models: ARF vs Batch Models}
  \begin{columns}[T]
    \begin{column}{0.48\textwidth}
      \textbf{Batch Model (Current):}
      \begin{itemize}
        \item LogisticRegression
        \item Train once, deploy frozen
        \item Retrain khi drift detected
        \item Full model replacement
      \end{itemize}

      \vspace{0.5em}
      \textbf{Ưu điểm:}
      \begin{itemize}
        \item ✅ Controlled experiments
        \item ✅ Reproducible
        \item ✅ Simple
      \end{itemize}

      \textbf{Nhược điểm:}
      \begin{itemize}
        \item ❌ High retraining cost
        \item ❌ Adaptation delay
        \item ❌ No online learning
      \end{itemize}
    \end{column}
    \begin{column}{0.48\textwidth}
      \textbf{Online Model (Proposal):}
      \begin{itemize}
        \item Adaptive Random Forest
        \item Incremental learning (sample-by-sample)
        \item Built-in per-tree ADWIN
        \item Dynamic tree replacement
      \end{itemize}

      \vspace{0.5em}
      \textbf{Ưu điểm:}
      \begin{itemize}
        \item ✅ Real-time adaptation
        \item ✅ Automatic drift handling
        \item ✅ Production-ready
      \end{itemize}

      \textbf{Nhược điểm:}
      \begin{itemize}
        \item ❌ Complex evaluation
        \item ❌ Less control
      \end{itemize}
    \end{column}
  \end{columns}

  \vspace{0.5em}
  \begin{block}{Proposal: Hybrid Approach}
    Keep ShapeDD (global) + Add ARF (per-tree ADWIN) → Compare both patterns
  \end{block}
\end{frame}

% ==============================================================================
% SECTION 7: INTEGRATION PROPOSALS
% ==============================================================================
\section{Đề xuất tích hợp}

\begin{frame}{Gaps trong phương pháp hiện tại}
  \begin{table}
    \centering
    \footnotesize
    \begin{tabular}{llcc}
      \toprule
      \textbf{Aspect} & \textbf{Current} & \textbf{SOTA} & \textbf{Gap} \\
      \midrule
      \rowcolor{red!10}
      Incremental drift & F1 = 0.143 & F1 = 0.73 & \textcolor{red}{\textbf{-0.587}} \\
      Labels required & 0\% (detection) & 1\% (CDSeer) & - \\
      Explainability & None & SHAP, Feature-level & ❌ \\
      Semi-supervised & No & Yes (CDSeer) & ❌ \\
      Ensemble & Single model & ARF, Meta-ADD & ❌ \\
      Method selection & Manual & Meta-learning & ❌ \\
      \bottomrule
    \end{tabular}
  \end{table}

  \vspace{0.5em}
  \begin{alertblock}{Biggest Gap}
    \textbf{Incremental drift:} F1 gap = -0.587 (largest weakness)
  \end{alertblock}

  \vspace{0.3em}
  \textbf{Opportunities:}
  \begin{itemize}
    \item ✅ Integrate CDSeer → Close incremental gap
    \item ✅ Add explainability → Industry requirement
    \item ✅ Build ensemble → No universal winner problem
  \end{itemize}
\end{frame}

\begin{frame}{Proposal 1: Semi-Supervised ShapeDD (CDSeer-inspired)}
  \begin{columns}[T]
    \begin{column}{0.48\textwidth}
      \textbf{Current approach:}
      \begin{enumerate}
        \item Detect drift (ShapeDD)
        \item Wait 50 samples
        \item Collect 800 samples
        \item Retrain with \textbf{all 800 labels}
      \end{enumerate}

      \vspace{0.5em}
      \textbf{Problem:}
      \begin{itemize}
        \item Requires 800 labels
        \item Expensive in production
        \item Slow adaptation
      \end{itemize}
    \end{column}
    \begin{column}{0.48\textwidth}
      \textbf{Proposed approach:}
      \begin{enumerate}
        \item Detect drift (ShapeDD)
        \item \textbf{Confidence-based sampling}
        \item Request label if conf < 0.6
        \item Retrain with \textbf{8 labels only} (1\%)
      \end{enumerate}

      \vspace{0.5em}
      \textbf{Benefits:}
      \begin{itemize}
        \item \textcolor{red}{\textbf{99\% label reduction}}
        \item F1: 0.562 → \textbf{0.70+} (estimated)
        \item \textbf{Faster} adaptation
        \item Handles incremental drift
      \end{itemize}
    \end{column}
  \end{columns}

  \vspace{0.5em}
  \begin{block}{Implementation}
    \textbf{Effort:} 2-3 weeks | \textbf{Priority:} Tier 1 (Highest) | \textbf{Impact:} High
  \end{block}
\end{frame}

\begin{frame}{Proposal 2: Ensemble Architecture (ShapeDD + ARF)}
  \begin{figure}
    \centering
    \begin{tikzpicture}[
      box/.style={rectangle, draw, thick, minimum width=2.5cm, minimum height=0.7cm, align=center, font=\small},
      arrow/.style={->, thick, >=stealth}
    ]
      % Data
      \node[box, fill=blue!20] (data) at (0,3) {Data Stream};

      % Two detection paths
      \node[box, fill=green!20] (shapedd) at (-2,1.5) {ShapeDD\\(Global)};
      \node[box, fill=yellow!20] (arf) at (2,1.5) {ARF\\(Per-tree ADWIN)};

      % Detection outputs
      \node[box, fill=orange!20] (global) at (-2,0) {Global Drift};
      \node[box, fill=orange!20] (local) at (2,0) {Local Drift};

      % Ensemble decision
      \node[box, fill=red!20] (decision) at (0,-1) {Ensemble Decision};

      % Arrows
      \draw[arrow] (data) -- (-2,2.2);
      \draw[arrow] (-2,2.2) -- (shapedd);
      \draw[arrow] (data) -- (2,2.2);
      \draw[arrow] (2,2.2) -- (arf);
      \draw[arrow] (shapedd) -- (global);
      \draw[arrow] (arf) -- (local);
      \draw[arrow] (global) -- (decision);
      \draw[arrow] (local) -- (decision);
    \end{tikzpicture}
  \end{figure}

  \begin{itemize}
    \item \textbf{ShapeDD:} Model-agnostic, unsupervised, global patterns
    \item \textbf{ARF:} Model-dependent, per-tree detection, local adaptation
    \item \textbf{Ensemble:} Redundancy, comparison study, best of both
  \end{itemize}

  \vspace{0.3em}
  \begin{block}{Research Contribution}
    First study comparing \textbf{SNR-based global} vs. \textbf{per-tree local} detection
  \end{block}
\end{frame}

\begin{frame}{Proposal 3: Explainability Module (SHAP-based)}
  \begin{columns}[T]
    \begin{column}{0.48\textwidth}
      \textbf{Current problem:}
      \begin{itemize}
        \item ShapeDD detects drift
        \item But \textbf{why} drift happened?
        \item Which features changed?
        \item Operators need insights
      \end{itemize}

      \vspace{0.5em}
      \textbf{Industry requirement:}
      \begin{itemize}
        \item 67\% enterprises miss drift
        \item Need actionable insights
        \item Root cause analysis
        \item Trust in ML systems
      \end{itemize}
    \end{column}
    \begin{column}{0.48\textwidth}
      \textbf{Proposed solution:}
      \begin{enumerate}
        \item Detect drift (ShapeDD)
        \item Compute \textbf{Shapley values}
        \item Rank features by contribution
        \item Visualize top-k drift causes
      \end{enumerate}

      \vspace{0.5em}
      \textbf{Output example:}
      \begin{itemize}
        \item Feature 3: SHAP = 0.45 ⬆️
        \item Feature 7: SHAP = 0.32 ⬆️
        \item Feature 1: SHAP = 0.15 ⬆️
        \item → Focus on features 3, 7, 1
      \end{itemize}
    \end{column}
  \end{columns}

  \vspace{0.5em}
  \begin{block}{Implementation}
    \textbf{Effort:} 1-2 weeks | \textbf{Priority:} Tier 1 | \textbf{Impact:} Differentiator
  \end{block}
\end{frame}

% ==============================================================================
% BACKUP SLIDES
% ==============================================================================
\appendix

\begin{frame}[noframenumbering]{Backup: Confusion Matrix}
  \begin{table}
    \centering
    \begin{tabular}{cc|cc}
      \multicolumn{2}{c}{} & \multicolumn{2}{c}{\textbf{Predicted}} \\
      & & Drift & No Drift \\
      \hline
      \multirow{2}{*}{\textbf{Actual}} & Drift & TP = 8 & FN = 2 \\
      & No Drift & FP = 2 & TN = 48 \\
    \end{tabular}
  \end{table}

  \vspace{1em}
  \begin{align*}
    \text{Precision} &= \frac{8}{8+2} = 0.80 \\
    \text{Recall} &= \frac{8}{8+2} = 0.80 \\
    \text{F1-Score} &= 0.80
  \end{align*}
\end{frame}

\begin{frame}[noframenumbering]{Backup: Runtime Comparison}
  \begin{table}
    \centering
    \small
    \begin{tabular}{lcc}
      \toprule
      \textbf{Method} & \textbf{Runtime (s)} & \textbf{Throughput (samples/s)} \\
      \midrule
      ShapeDD & 2.05 & 4,878 \\
      ShapeDD\_v2 & 1.15 & 8,696 \\
      ADWIN & 1.18 & 8,475 \\
      DAWIDD & 2.41 & 4,149 \\
      MMD & 0.52 & 19,231 \\
      KS & 0.21 & 47,619 \\
      \bottomrule
    \end{tabular}
  \end{table}

  \vspace{0.5em}
  \textbf{Nhận xét:} ShapeDD có chi phí tính toán cao hơn streaming methods, nhưng đạt độ chính xác tốt hơn đáng kể.
\end{frame}

\end{document}
