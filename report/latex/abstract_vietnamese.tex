\chapter*{Tóm tắt luận văn}
\addcontentsline{toc}{chapter}{Tóm tắt luận văn}

Hiện nay, hiện tượng trôi dạt khái niệm (concept drift) là một trong những thách thức lớn làm suy giảm hiệu suất của các hệ thống học máy, đặc biệt là trong môi trường dữ liệu streaming liên tục thay đổi. Vì vậy, trong luận văn này, học viên tiến hành nghiên cứu các giải pháp phát hiện sự thay đổi phân phối dữ liệu, tập trung vào các phương pháp không giám sát và đồng thời áp dụng các cải tiến thuật toán (như ShapeDD) để đạt được độ chính xác cao hơn trong quá trình phát hiện. Ngoài ra, luận văn cũng tiến hành đánh giá thực nghiệm toàn diện, so sánh hiệu năng của mô hình đề xuất với các phương pháp hiện có để chỉ ra điểm mạnh, yếu và các hướng phát triển. Luận văn còn đề xuất thiết kế và triển khai hệ thống giám sát thời gian thực (sử dụng Kafka) nhằm áp dụng mô hình phát hiện trôi dạt vào bài toán xử lý dữ liệu thực tế.

\textbf{Từ khóa:} trôi dạt khái niệm, học máy, hệ thống thích ứng, khai thác luồng dữ liệu, môi trường không dừng, phát hiện drift, chiến lược thích ứng
