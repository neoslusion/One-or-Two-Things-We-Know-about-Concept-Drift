\chapter*{Tóm tắt luận văn}
\addcontentsline{toc}{chapter}{Tóm tắt luận văn}

Trôi dạt khái niệm (concept drift) là một thách thức cơ bản trong lĩnh vực học máy, xảy ra khi phân phối dữ liệu cơ bản thay đổi theo thời gian, gây suy giảm hiệu suất của mô hình. Luận văn này nghiên cứu phát hiện trôi dạt khái niệm sử dụng phương pháp Shape Drift Detector (ShapeDD), với trọng tâm đặc biệt vào nền tảng lý thuyết, triển khai thực tế, và hiệu quả của phương pháp này trên các kịch bản drift khác nhau. Nghiên cứu trình bày một phân tích toàn diện về thuật toán ShapeDD, sử dụng Maximum Mean Discrepancy (MMD) trong không gian Reproducing Kernel Hilbert Space (RKHS) để phát hiện sự thay đổi phân phối trong luồng dữ liệu. Luận văn bao gồm phân tích chi tiết về nền tảng lý thuyết của MMD, quy trình phát hiện đa giai đoạn của ShapeDD, và đặc điểm hiệu suất của nó trên nhiều dạng drift khác nhau.

Thông qua đánh giá thực nghiệm toàn diện trên 10 tập dữ liệu tổng hợp (synthetic) được kiểm soát, luận văn chứng minh hiệu quả của phương pháp ShapeDD\_OW\_MMD -- tích hợp Optimally-Weighted MMD để đạt hiệu suất tính toán cao (tăng tốc 7 lần) trong khi duy trì độ chính xác phát hiện cao. Luận văn phân tích tác động của các tham số quan trọng như kích thước cửa sổ, lựa chọn kernel, và ngưỡng ý nghĩa thống kê đến hiệu suất phát hiện. Nghiên cứu tích hợp ShapeDD với phương pháp CDT\_MSW để tự động phân loại năm loại drift (sudden, gradual, incremental, recurrent, blip) và áp dụng chiến lược thích ứng phù hợp cho từng loại. Các đóng góp chính bao gồm: (1) phân tích lý thuyết chi tiết về Shape Drift Detector và nền tảng toán học, (2) tích hợp OW-MMD cho phát hiện real-time thông lượng cao, (3) phát triển các chiến lược thích ứng theo loại drift với cơ chế caching mô hình cho pattern lặp lại, và (4) phát triển hệ thống đầu-cuối dựa trên Kafka kết hợp phát hiện drift, phân loại loại drift, và thích ứng mô hình tự động.

\textbf{Từ khóa:} trôi dạt khái niệm, học máy, hệ thống thích ứng, khai thác luồng dữ liệu, môi trường không dừng, phát hiện drift, chiến lược thích ứng
