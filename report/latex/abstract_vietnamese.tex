\chapter*{Tóm tắt luận văn}
\addcontentsline{toc}{chapter}{Tóm tắt luận văn}

Hiện nay, trong lĩnh vực học máy và khai thác dữ liệu, việc xử lý các luồng dữ liệu (data stream) đang trở thành một thách thức quan trọng do tính chất liên tục và biến đổi của dữ liệu theo thời gian. Một trong những vấn đề nổi bật là sự trôi dạt khái niệm (concept drift), khi phân phối dữ liệu thay đổi theo thời gian, dẫn đến suy giảm hiệu suất của các mô hình học máy được huấn luyện trên dữ liệu lịch sử. Điều này đặc biệt quan trọng trong các ứng dụng thực tế như phát hiện gian lận, dự báo tài chính, và giám sát hệ thống, nơi mà dữ liệu liên tục thay đổi và mô hình cần phải thích nghi nhanh chóng để duy trì độ chính xác.

Đề tài này tập trung tìm hiểu về việc nghiên cứu giải thuật dùng để phát hiện sự trôi dạt, điều thường xảy ra nhiều trong các ứng dụng học máy trong cuộc sống và công nghiệp cùng với những vấn đề gặp phải trong thực tế của các mô hình học máy khi gặp phải concept drift – yếu tố làm ảnh hưởng đến độ chính xác và hiệu năng của các mạng nơ-ron khi áp dụng trong môi trường có dữ liệu biến đổi liên tục về thời gian.
Mục tiêu của đề tài là ứng dụng một giải thuật về phát hiện trôi dạt, tìm hiểu cơ sở lý thuyết, cách hoạt động cũng như ứng dụng lên một số tập dữ liệu bao gồm cả thực tế và dữ liệu synthetic để đánh giá về mức độ hiệu quả và ứng dụng.
