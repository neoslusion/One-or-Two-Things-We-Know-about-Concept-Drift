\chapter*{Tóm tắt luận văn}
\addcontentsline{toc}{chapter}{Tóm tắt luận văn}

Trôi dạt khái niệm (concept drift) là một thách thức cơ bản trong lĩnh vực học máy, xảy ra khi phân phối dữ liệu cơ bản thay đổi theo thời gian, gây suy giảm hiệu suất của mô hình. Luận văn này nghiên cứu phát hiện trôi dạt khái niệm sử dụng phương pháp Shape Drift Detector (ShapeDD), với trọng tâm đặc biệt vào nền tảng lý thuyết, triển khai thực tế, và hiệu quả của phương pháp này trên các kịch bản drift khác nhau. Chúng tôi trình bày một nghiên cứu toàn diện về thuật toán ShapeDD, sử dụng Maximum Mean Discrepancy (MMD) trong không gian Reproducing Kernel Hilbert Space (RKHS) để phát hiện sự thay đổi phân phối trong luồng dữ liệu. Nghiên cứu bao gồm phân tích chi tiết về nền tảng lý thuyết của MMD, quy trình phát hiện đa giai đoạn của ShapeDD, và đặc điểm hiệu suất của nó trên nhiều dạng drift khác nhau.

Thông qua đánh giá thực nghiệm trên tập dữ liệu tổng hợp được kiểm soát, chúng tôi chứng minh hiệu quả của ShapeDD trong việc phát hiện sudden drift (drift đột ngột). Luận văn phân tích tác động của các tham số quan trọng như kích thước cửa sổ, lựa chọn kernel, và ngưỡng ý nghĩa thống kê đến hiệu suất phát hiện. Ngoài ra, chúng tôi tích hợp ShapeDD với phương pháp CDT\_MSW để tự động phân loại năm loại drift (sudden, gradual, incremental, recurrent, blip) và áp dụng chiến lược thích ứng phù hợp cho từng loại. Các đóng góp chính của nghiên cứu bao gồm: (1) phân tích lý thuyết chi tiết về Shape Drift Detector và nền tảng toán học của nó, (2) đánh giá thực nghiệm toàn diện trên tập dữ liệu tổng hợp với các đặc tính drift được kiểm soát, (3) phát triển hệ thống đầu-cuối kết hợp phát hiện drift, phân loại loại drift, và thích ứng mô hình tự động, và (4) hướng dẫn thực tế về triển khai hệ thống phát hiện drift trong các ứng dụng thực tế sử dụng Apache Kafka.

Kết quả nghiên cứu cho thấy ShapeDD đạt hiệu suất xuất sắc trong các kịch bản sudden drift với F1-score = 1.0 và độ trễ phát hiện chỉ 4 mẫu dữ liệu. Hệ thống thích ứng đạt tỷ lệ phục hồi 82.8\% sau khi xảy ra drift. Nghiên cứu cũng chỉ ra rằng ShapeDD với adaptive windowing vượt trội hơn so với các phương pháp streaming truyền thống về độ chính xác phát hiện, mặc dù có chi phí tính toán cao hơn. Chúng tôi đề xuất các chiến lược adaptive windowing và ensemble methods như các cải tiến tiềm năng để tăng cường độ bền vững trên nhiều dạng drift đa dạng.

\textbf{Từ khóa:} trôi dạt khái niệm, học máy, hệ thống thích ứng, khai thác luồng dữ liệu, môi trường không dừng, phát hiện drift, chiến lược thích ứng
