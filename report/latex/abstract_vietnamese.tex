\chapter*{Tóm tắt luận văn}
\addcontentsline{toc}{chapter}{Tóm tắt luận văn}

Trôi dạt khái niệm (concept drift) là một thách thức cơ bản trong lĩnh vực học máy, xảy ra khi phân phối dữ liệu cơ bản thay đổi theo thời gian, gây suy giảm hiệu suất của mô hình. Luận văn này nghiên cứu phát hiện trôi dạt khái niệm sử dụng phương pháp Shape Drift Detector (ShapeDD), với trọng tâm đặc biệt vào nền tảng lý thuyết, triển khai thực tế, và hiệu quả của phương pháp này trên các kịch bản drift khác nhau. Nghiên cứu trình bày một phân tích toàn diện về thuật toán ShapeDD, sử dụng Maximum Mean Discrepancy (MMD) trong không gian Reproducing Kernel Hilbert Space (RKHS) để phát hiện sự thay đổi phân phối trong luồng dữ liệu. Luận văn bao gồm phân tích chi tiết về nền tảng lý thuyết của MMD, quy trình phát hiện đa giai đoạn của ShapeDD, và đặc điểm hiệu suất của nó trên nhiều dạng drift khác nhau.

Thông qua đánh giá thực nghiệm toàn diện trên 8 tập dữ liệu tổng hợp được kiểm soát, luận văn chứng minh hiệu quả của phương pháp ShapeDD SNR-Adaptive - một detector hybrid tự động chọn chiến lược dựa trên tỷ lệ tín hiệu trên nhiễu (SNR). Luận văn phân tích tác động của các tham số quan trọng như kích thước cửa sổ, lựa chọn kernel, và ngưỡng ý nghĩa thống kê đến hiệu suất phát hiện. Nghiên cứu tích hợp ShapeDD với phương pháp CDT\_MSW để tự động phân loại năm loại drift (sudden, gradual, incremental, recurrent, blip) và áp dụng chiến lược thích ứng phù hợp cho từng loại. Các đóng góp chính bao gồm: (1) phân tích lý thuyết chi tiết về Shape Drift Detector và nền tảng toán học, (2) phát triển phương pháp SNR-Adaptive với tối ưu hóa theo tiêu chuẩn Neyman-Pearson, (3) đánh giá thực nghiệm toàn diện trên 144 experiments (18 methods × 8 datasets), (4) phát triển hệ thống đầu-cuối kết hợp phát hiện drift, phân loại loại drift, và thích ứng mô hình tự động, và (5) chứng minh hiệu ứng buffer dilution trong drift detection.

Kết quả nghiên cứu cho thấy ShapeDD SNR-Adaptive đạt hạng 2/18 methods với F1-score = 0.562 trên benchmark toàn diện, trong đó đạt hạng 1 (tie) trên dataset cường độ cao (gen\_random\_severe: F1 = 0.727) và gần hạng 1 trên stagger dataset (F1 = 0.833, gap = +0.036). Mean Time To Detection (MTTD) = 31.4 samples, nhanh hơn các methods truyền thống như ADWIN, DAWIDD và MMD. Phương pháp cũng đạt strategy balance 58.7/41.3 (aggressive/conservative), gần với mục tiêu tối ưu 50/50 theo tiêu chuẩn Neyman-Pearson. Trong thí nghiệm sudden drift riêng lẻ, ShapeDD\_Improved đạt F1-score hoàn hảo (1.0) với độ trễ phát hiện chỉ 4 mẫu và hệ thống thích ứng đạt tỷ lệ phục hồi 82.8\%. Nghiên cứu cũng chỉ ra rằng buffer-based detection giảm observed SNR khoảng 100× so với theoretical SNR, một phát hiện quan trọng cho việc calibration threshold. Luận văn đề xuất các cải tiến tiềm năng bao gồm adaptive threshold cho subtle drift và ensemble methods để tăng cường độ bền vững.

\textbf{Từ khóa:} trôi dạt khái niệm, học máy, hệ thống thích ứng, khai thác luồng dữ liệu, môi trường không dừng, phát hiện drift, chiến lược thích ứng
