\chapter*{ABSTRACT}
\addcontentsline{toc}{chapter}{ABSTRACT}

Concept drift is a fundamental challenge in machine learning where the underlying data distribution changes over time, causing model performance to degrade. This thesis investigates the detection of concept drift using the Shape Drift Detector (ShapeDD) method, with a specific focus on its theoretical foundations, practical implementation, and effectiveness across different drift scenarios. We present a comprehensive study of the ShapeDD algorithm, which employs Maximum Mean Discrepancy (MMD) in Reproducing Kernel Hilbert Space (RKHS) to detect distributional changes in data streams. Our research includes detailed analysis of the theoretical foundations of MMD, the multi-stage detection process of ShapeDD, and its performance characteristics across various drift patterns.

Through extensive experimental evaluation on 7 diverse datasets (including synthetic and real-world) with controlled drift characteristics, we demonstrate the effectiveness of the ShapeDD SNR-Adaptive method - a hybrid detector that automatically selects strategies based on Signal-to-Noise Ratio (SNR). We analyze the impact of critical parameters such as window size, kernel selection, and statistical significance thresholds on detection performance. The research integrates ShapeDD with CDT\_MSW method to automatically classify five drift types (sudden, gradual, incremental, recurrent, blip) and apply appropriate adaptive strategies for each type. The main contributions of this work include: (1) a detailed theoretical analysis of the Shape Drift Detector and its mathematical foundations, (2) development of SNR-Adaptive method with optimization based on Neyman-Pearson criterion, (3) development of an end-to-end system combining drift detection, drift type classification, and automatic model adaptation, and (4) demonstration of buffer dilution effect in drift detection.

\textbf{Keywords:} concept drift, machine learning, adaptive systems, data stream mining, non-stationary environments, drift detection, adaptive strategies 
