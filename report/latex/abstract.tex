\chapter*{ABSTRACT}
\addcontentsline{toc}{chapter}{ABSTRACT}

Concept drift is a fundamental challenge in machine learning where the underlying data distribution changes over time, causing model performance to degrade. This thesis investigates the detection of concept drift using the Shape Drift Detector (ShapeDD) method, with a specific focus on its theoretical foundations, practical implementation, and effectiveness across different drift scenarios.

We present a comprehensive study of the ShapeDD algorithm, which employs Maximum Mean Discrepancy (MMD) in Reproducing Kernel Hilbert Space (RKHS) to detect distributional changes in data streams. Our research includes detailed analysis of the theoretical foundations of MMD, the multi-stage detection process of ShapeDD, and its performance characteristics across various drift patterns.

Through extensive experimental evaluation on synthetic datasets with controlled drift characteristics, we demonstrate ShapeDD's effectiveness in detecting concept drift with focus on abrupt drift scenarios. We analyze the impact of critical parameters such as window size, kernel selection, and statistical significance thresholds on detection performance. The experimental evaluation includes comprehensive comparison with 17 baseline methods across 8 synthetic datasets.

The main contributions of this work include: (1) a detailed theoretical analysis of the Shape Drift Detector and its mathematical foundations including MMD in RKHS and triangle pattern detection, (2) development of SNR-Adaptive method - a novel hybrid drift detector that automatically adapts strategy based on signal-to-noise ratio, (3) comprehensive experimental evaluation on synthetic datasets demonstrating the effectiveness of SNR-Adaptive (F1=0.697, ranked 4th/18 methods), and (4) discovery of buffer dilution effect where observed SNR is ~100× lower than theoretical SNR in rolling buffer scenarios.

Our findings reveal that ShapeDD performs exceptionally well for abrupt drift scenarios but requires careful parameter tuning for incremental drift detection. We propose ensemble methods and adaptive windowing strategies as potential improvements for enhanced robustness across diverse drift patterns.

\textbf{Keywords:} concept drift, machine learning, adaptive systems, data stream mining, non-stationary environments 
