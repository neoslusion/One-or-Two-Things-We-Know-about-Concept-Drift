\chapter*{Abstract}
\addcontentsline{toc}{chapter}{Abstract}

Concept drift is a fundamental challenge in machine learning where the underlying data distribution changes over time, causing model performance to degrade. This thesis investigates the current state of knowledge about concept drift, focusing on detection methods, adaptation strategies, and evaluation metrics.

We present a comprehensive analysis of existing concept drift detection algorithms, categorizing them into statistical methods, model-based approaches, and ensemble techniques. Our research identifies key limitations in current methodologies and proposes novel approaches for handling different types of drift patterns.

Through extensive experimental evaluation on both synthetic and real-world datasets, we demonstrate that no single approach works universally across all drift scenarios. We introduce a framework for characterizing drift patterns and matching them with appropriate detection and adaptation strategies.

The main contributions of this work include: (1) a taxonomy of concept drift types and their characteristics, (2) a comparative analysis of state-of-the-art detection methods, (3) novel metrics for evaluating drift detection performance, and (4) guidelines for practitioners on selecting appropriate methods for specific application domains.

Our findings suggest that future research should focus on developing adaptive meta-learning approaches that can automatically select and configure drift handling methods based on the observed data characteristics.

\textbf{Keywords:} concept drift, machine learning, adaptive systems, data stream mining, non-stationary environments 
