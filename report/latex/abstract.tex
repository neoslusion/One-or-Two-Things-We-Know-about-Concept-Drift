\chapter*{ABSTRACT}
\addcontentsline{toc}{chapter}{ABSTRACT}

Concept drift is a fundamental challenge in machine learning, occurring when the distribution of underlying data changes over time, leading to model performance degradation. This thesis investigates drift detection methods, focusing primarily on the Shape Drift Detector (ShapeDD), with particular emphasis on its theoretical foundation, practical implementation, and effectiveness across various drift scenarios. The study presents a comprehensive analysis of the ShapeDD algorithm, utilizing Maximum Mean Discrepancy (MMD) in the Reproducing Kernel Hilbert Space (RKHS) to detect changes in data flow distribution. The thesis includes a detailed analysis of the MMD's theoretical basis, ShapeDD's multi-stage detection process, and its performance characteristics across various drift scenarios.

Through a comprehensive experimental evaluation on 10 controlled synthetic datasets, this thesis demonstrates the effectiveness of the ShapeDD\_OW\_MMD method — integrating Optimally\_Weighted MMD — to achieve high computational performance while maintaining high detection accuracy. The thesis analyzes the impact of key parameters such as window size, kernel selection, and statistical significance threshold on detection performance. The thesis proposes integrating ShapeDD with the CDT\_MSW method to automatically classify five types of drift (sudden, gradual, incremental, recurrent, blip) and apply appropriate adaptive strategies for each type. Key contributions include: (1) a detailed theoretical analysis of the Shape Drift Detector and its mathematical foundation, (2) proposals for improvements to the original ShapeDD by improving the MMD platform, (3) development of drift type adaptation strategies with model caching mechanisms for repeating patterns, and (4) development of a Kafka-based end-to-end system combining drift detection, drift type classification, and automatic model adaptation.

\textbf{Keywords:} concept drift, machine learning, adaptive systems, data stream mining, non-stationary environments, drift detection, adaptive strategies 
