% ==============================================================================
% NEW SLIDES TO ADD TO presentation.tex
% Replace the old results section (around line 522-606) with these updated slides
% ==============================================================================

\begin{frame}{Kết quả: Comprehensive Benchmark (18 methods × 8 datasets)}
  \begin{table}
    \centering
    \footnotesize
    \begin{tabular}{lccc}
      \toprule
      \textbf{Method} & \textbf{F1-Score} & \textbf{Std} & \textbf{Rank} \\
      \midrule
      ShapeDD\_Adaptive\_None & 0.571 & 0.286 & 1/18 \\
      \textbf{ShapeDD\_SNR\_Adaptive} & \textbf{0.562} & \textbf{0.254} & \textbf{2/18} \\
      ShapeDD\_Adaptive\_v2\_None & 0.557 & 0.281 & 3/18 \\
      ShapeDD (Original) & 0.544 & 0.224 & 4/18 \\
      DAWIDD & 0.515 & 0.132 & 5/18 \\
      ADWIN & 0.507 & 0.345 & 6/18 \\
      MMD & 0.500 & 0.195 & 7/18 \\
      \midrule
      \multicolumn{4}{l}{\textit{(11 phương pháp khác có F1 < 0.50)}} \\
      \bottomrule
    \end{tabular}
  \end{table}

  \vspace{0.5em}
  \begin{block}{Điểm nổi bật}
    \begin{itemize}
      \item \textbf{SNR-Adaptive xếp hạng 2/18} - phương pháp hybrid đầu tiên sử dụng SNR
      \item Std = 0.254 thấp hơn các adaptive methods khác (0.286, 0.281)
      \item Tổng hợp tốt trên nhiều loại drift khác nhau
    \end{itemize}
  \end{block}
\end{frame}

\begin{frame}{Kiến trúc SNR-Adaptive Method}
  \begin{figure}
    \centering
    \includegraphics[width=0.85\textwidth]{image/snr_adaptive_architecture.png}
    \caption{Hệ thống SNR-Adaptive tự động lựa chọn strategy (aggressive/conservative) dựa trên SNR}
  \end{figure}
\end{frame}

\begin{frame}{Kết quả: Performance Comparison Top 10}
  \begin{figure}
    \centering
    \includegraphics[width=0.95\textwidth]{image/f1_comparison.png}
    \caption{So sánh F1-score của top 10 methods. SNR-Adaptive xếp hạng 2/18.}
  \end{figure}
\end{frame}

\begin{frame}{Kết quả: Strategy Selection Distribution}
  \begin{figure}
    \centering
    \includegraphics[width=0.75\textwidth]{image/strategy_selection.png}
    \caption{Phân bố lựa chọn strategy: 58.7\% aggressive, 41.3\% conservative (gần 50/50 theo Neyman-Pearson optimization)}
  \end{figure}
\end{frame}

\begin{frame}{Kết quả: Per-Dataset Performance (Best Cases)}
  \textbf{TOP TIER (F1 > 0.70):}

  \begin{table}
    \centering
    \small
    \begin{tabular}{lcccl}
      \toprule
      \textbf{Dataset} & \textbf{F1} & \textbf{Recall} & \textbf{MTTD} & \textbf{Kết quả} \\
      \midrule
      stagger & \textbf{0.833} & 1.0 & 4.8 & Gần hạng 1 (+0.036) \\
      enhanced\_sea & \textbf{0.818} & 0.9 & 7.6 & Mạnh (+0.134) \\
      gen\_random\_severe & \textbf{0.727} & 0.8 & 14.5 & \textcolor{red}{\textbf{Hạng 1 (tie)}} \\
      \bottomrule
    \end{tabular}
  \end{table}

  \vspace{0.5em}
  \begin{block}{Phân tích}
    \begin{itemize}
      \item \textbf{Đạt hạng 1 (tie)} trên gen\_random\_severe → xác nhận thiết kế
      \item \textbf{Gần hạng 1} trên stagger (gap chỉ 3.6\%)
      \item \textbf{Perfect recall} (1.0) trên stagger → phát hiện tất cả 10 drifts
      \item Hiệu quả trên \textbf{high-SNR drifts} (cường độ cao)
    \end{itemize}
  \end{block}
\end{frame}

\begin{frame}{Kết quả: Per-Dataset Performance (Worst Cases)}
  \textbf{BOTTOM TIER (F1 < 0.30):}

  \begin{table}
    \centering
    \small
    \begin{tabular}{lcccl}
      \toprule
      \textbf{Dataset} & \textbf{F1} & \textbf{Recall} & \textbf{MTTD} & \textbf{Gap to Winner} \\
      \midrule
      hyperplane & 0.267 & 0.2 & 43.0 & +0.289 \\
      standard\_sea & \textcolor{red}{\textbf{0.143}} & 0.1 & 145.0 & +0.429 \\
      \bottomrule
    \end{tabular}
  \end{table}

  \vspace{0.5em}
  \begin{alertblock}{Nguyên nhân và giải thích}
    \begin{itemize}
      \item \textbf{standard\_sea:} Subtle drift + buffer dilution → SNR quá thấp
      \begin{itemize}
        \item Chỉ phát hiện 1/10 drifts (10\% recall)
        \item ADWIN (streaming) tốt hơn 42.9\% (F1=0.571)
      \end{itemize}
      \item \textbf{hyperplane:} Gradual rotation → window-based detector khó phát hiện
      \item \textbf{Kết luận:} Buffer-based approach không phù hợp với subtle gradual drifts
    \end{itemize}
  \end{alertblock}
\end{frame}

\begin{frame}{Phân tích: Performance by Drift Intensity}
  \begin{columns}[T]
    \begin{column}{0.5\textwidth}
      \textbf{Xu hướng quan sát:}
      \begin{itemize}
        \item \textbf{High} (intensity $\geq$ 1.0):\\
        F1 = 0.667-0.727 ✅
        \item \textbf{Medium} (0.25-1.0):\\
        F1 = 0.455-0.583 ⚠️
        \item \textbf{Low} (intensity $\leq$ 0.25):\\
        F1 = 0.143-0.455 ❌
      \end{itemize}
    \end{column}
    \begin{column}{0.5\textwidth}
      \textbf{Giải thích:}
      \begin{itemize}
        \item High intensity → high SNR → aggressive strategy → detection tốt
        \item Low intensity → low SNR → conservative strategy → missed detections
        \item Buffer dilution: observed SNR $\sim$ 100× thấp hơn theoretical SNR
      \end{itemize}
    \end{column}
  \end{columns}

  \vspace{1em}
  \begin{block}{Threshold Optimization}
    Threshold tối ưu = 0.010 (qua Neyman-Pearson criterion)\\
    Cân bằng precision-recall tại strategy balance $\sim$ 50/50
  \end{block}
\end{frame}

\begin{frame}{Buffer Dilution Effect}
  \begin{figure}
    \centering
    \includegraphics[width=0.8\textwidth]{image/buffer_dilution.png}
    \caption{Buffer dilution làm giảm observed SNR khoảng 100× so với theoretical SNR}
  \end{figure}

  \vspace{0.5em}
  \textbf{Giải thích:}
  \begin{itemize}
    \item Rolling buffer (750 samples) chứa cả stable data và drift data
    \item Stable: 90\%, Drift: 10\% → dilution ratio = 9:1
    \item Observed SNR = 0.005-0.020 (so với theoretical 0.4-4.0)
    \item \textbf{Hệ quả:} Cần recalibrate threshold từ lý thuyết sang thực nghiệm
  \end{itemize}
\end{frame}

\begin{frame}{Threshold Sensitivity Analysis}
  \begin{figure}
    \centering
    \includegraphics[width=0.85\textwidth]{image/threshold_sensitivity.png}
    \caption{Phân tích độ nhạy của SNR threshold. Optimal = 0.010 cân bằng precision-recall.}
  \end{figure}
\end{frame}

\begin{frame}{So sánh với Baseline Methods}
  \begin{table}
    \centering
    \footnotesize
    \begin{tabular}{lccl}
      \toprule
      \textbf{Strategy} & \textbf{F1} & \textbf{Std} & \textbf{Đặc điểm} \\
      \midrule
      Adaptive\_None (conservative) & 0.571 & 0.286 & Chiến lược đơn \\
      \textbf{SNR-Adaptive (hybrid)} & \textbf{0.562} & \textbf{0.254} & \textbf{Auto-selection} \\
      ShapeDD (original) & 0.544 & 0.224 & Baseline \\
      Adaptive\_v2\_High (aggressive) & 0.464 & 0.292 & Chiến lược đơn \\
      \bottomrule
    \end{tabular}
  \end{table}

  \vspace{0.5em}
  \begin{block}{Nhận xét}
    \begin{itemize}
      \item SNR-Adaptive xếp hạng 2, cao hơn ShapeDD gốc (rank 4)
      \item Std thấp hơn conservative methods → ổn định hơn
      \item Hybrid approach giúp tổng quát hóa tốt hơn pure strategies
      \item Trade-off: Không thắng trên nhiều datasets, nhưng không bao giờ thất bại hoàn toàn (trừ 2 outliers)
    \end{itemize}
  \end{block}
\end{frame}

\begin{frame}{Method Ranking Visualization}
  \begin{figure}
    \centering
    \includegraphics[width=0.95\textwidth]{image/method_ranking.png}
    \caption{Bảng xếp hạng 18 methods. SNR-Adaptive rank 2/18 với F1=0.562.}
  \end{figure}
\end{frame}

\begin{frame}{Optimization Comparison (v1 vs v2)}
  \begin{figure}
    \centering
    \includegraphics[width=0.85\textwidth]{image/optimization_comparison.png}
    \caption{So sánh parameter optimization. Version 2 sửa inverted threshold bug, cải thiện hiệu suất.}
  \end{figure}
\end{frame}

\begin{frame}{Gap Analysis: Distance to Winners}
  \textbf{Khoảng cách so với method tốt nhất mỗi dataset:}

  \begin{columns}[T]
    \begin{column}{0.5\textwidth}
      \textbf{Excellent gaps ($<$ 0.10):}
      \begin{itemize}
        \item gen\_random\_severe: \textbf{0.000} ✨
        \item stagger: 0.036 ✨
        \item gen\_random\_moderate: 0.083 ✅
        \item gen\_random\_ultra\_severe: 0.070 ✅
      \end{itemize}
    \end{column}
    \begin{column}{0.5\textwidth}
      \textbf{Large gaps ($>$ 0.20):}
      \begin{itemize}
        \item hyperplane: 0.289 🔴
        \item standard\_sea: 0.429 🔴
      \end{itemize}

      \vspace{1em}
      \textbf{Average gap:} 0.153\\
      (15.3\% kém hơn winner trung bình)
    \end{column}
  \end{columns}

  \vspace{1em}
  \begin{block}{Kết luận}
    Competitive trên 6/8 datasets (75\%), thất bại trên 2 outliers
  \end{block}
\end{frame}

\begin{frame}{Đóng góp nghiên cứu}
  \begin{enumerate}
    \item \textbf{Phương pháp hybrid đầu tiên} sử dụng SNR để auto-select strategy
    \begin{itemize}
      \item Aggressive khi SNR > 0.010
      \item Conservative khi SNR $\leq$ 0.010
      \item Strategy balance $\sim$ 50/50 theo Neyman-Pearson criterion
    \end{itemize}

    \item \textbf{Chứng minh buffer dilution effect}
    \begin{itemize}
      \item Observed SNR $\sim$ 100× thấp hơn theoretical
      \item Cần recalibrate threshold: 0.4-4.0 → 0.010
    \end{itemize}

    \item \textbf{Đánh giá toàn diện}
    \begin{itemize}
      \item 18 competing methods
      \item 8 diverse datasets (sudden, gradual, high/low intensity)
      \item Per-dataset analysis → điểm mạnh/yếu rõ ràng
    \end{itemize}

    \item \textbf{Đạt rank 2/18} với F1=0.562
    \begin{itemize}
      \item Ties 1st trên gen\_random\_severe
      \item Near-win trên stagger (gap = 0.036)
    \end{itemize}
  \end{enumerate}
\end{frame}

\begin{frame}{Hạn chế và hướng phát triển}
  \textbf{Hạn chế đã phát hiện:}
  \begin{itemize}
    \item \textbf{Buffer dilution:} Giảm SNR $\sim$ 100×, ảnh hưởng subtle drift detection
    \item \textbf{Gradual drift:} Window-based approach khó phát hiện continuous changes
    \item \textbf{Subtle drift:} Thất bại trên standard\_sea (F1 = 0.143)
    \item \textbf{High variance:} Std = 0.254 (performance dao động theo drift type)
  \end{itemize}

  \vspace{1em}
  \textbf{Hướng phát triển:}
  \begin{enumerate}
    \item \textbf{Adaptive threshold:} Điều chỉnh threshold theo dataset characteristics
    \item \textbf{Hybrid buffer:} Kết hợp buffer-based và streaming approaches
    \item \textbf{Multi-scale SNR:} Ước lượng SNR ở nhiều time scales khác nhau
    \item \textbf{Gradual drift detection:} Thêm module phát hiện slow drifts
    \item \textbf{Real-world validation:} Test trên Electricity, Weather, sensor data
  \end{enumerate}
\end{frame}

\begin{frame}{Khuyến nghị sử dụng}
  \begin{block}{Nên sử dụng SNR-Adaptive khi:}
    \begin{itemize}
      \item \textbf{Abrupt drifts} với cường độ cao (intensity $\geq$ 1.0)
      \item \textbf{Concept-based drifts} với concept shifts rõ ràng
      \item \textbf{Cần tổng quát hóa} trên nhiều loại drift khác nhau
      \item \textbf{Có tài nguyên} tính toán (buffer storage, permutation tests)
    \end{itemize}
  \end{block}

  \vspace{0.5em}
  \begin{alertblock}{Không nên sử dụng khi:}
    \begin{itemize}
      \item \textbf{Subtle gradual drifts} (intensity $\leq$ 0.25) → Chọn ADWIN hoặc MMD
      \item \textbf{Continuous rotation} (hyperplane-like) → Chọn HDDM\_W
      \item \textbf{Real-time constraints} rất chặt → Chọn ADWIN (5× faster)
      \item \textbf{Memory constraints} → Chọn streaming methods (DDM, EDDM)
    \end{itemize}
  \end{alertblock}
\end{frame}
