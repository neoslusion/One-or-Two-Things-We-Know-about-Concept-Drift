\begin{tabular}{|l|c|c|c|c|}
\hline
\textbf{Method} & \textbf{Detection F1} & \textbf{CAT Acc (\%)} & \textbf{SUB Acc (\%)} & \textbf{Incremental Acc (\%)} \\
\hline
\multicolumn{5}{|c|}{\textbf{SE-CDT Progression}} \\
\hline
Baseline SE-CDT & 0.441 & 85.8\% & 46.6\% & 8.5\% \\
+ Phase 2.1 (Blip) & 0.441 & 86.0\% & 46.8\% & 8.5\% \\
+ Phase 2.2 (Temporal) & \textbf{0.481} & \textbf{86.5\%} & \textbf{48.0\%} & \textbf{40.0\%} \\
\hline
\multicolumn{5}{|c|}{\textbf{Improvement Summary}} \\
\hline
Absolute Gain & +0.040 & +0.7\% & +1.4\% & +31.5\% \\
Relative Gain & +9.1\% & +0.8\% & +3.0\% & +370\% \\
\hline
\multicolumn{5}{|c|}{\textbf{Comparison with Other Methods}} \\
\hline
CDT\_MSW (Paper) & N/A & 87.5\% & 42.0\% & N/A \\
MMD\_ADW (Best) & \textbf{0.548} & N/A & N/A & N/A \\
SE-CDT (Ours) & 0.481 & 86.5\% & 48.0\% & 40.0\% \\
\hline
\multicolumn{5}{|l|}{\footnotesize Phase 2.1: PPR/DPAR features for Blip detection} \\
\multicolumn{5}{|l|}{\footnotesize Phase 2.2: LTS/SDS/MS temporal features for Incremental detection} \\
\multicolumn{5}{|l|}{\footnotesize Detection F1: Overall drift detection performance (30 runs, 12 datasets)} \\
\multicolumn{5}{|l|}{\footnotesize CAT: Categorical accuracy (TCD vs PCD classification)} \\
\multicolumn{5}{|l|}{\footnotesize SUB: Subcategory accuracy (5-class: Sudden, Blip, Gradual, Incremental, Recurrent)} \\
\hline
\end{tabular}
